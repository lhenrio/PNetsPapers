\theoremstyle{plain}
\newtheorem{thm}{Theorem}
\newtheorem{lem}{Lemma}
\newtheorem{cor}{Corollary}
\newtheorem{prop}{Proposition}

\theoremstyle{definition}
\newtheorem{defi}{Definition}
\newtheorem*{noti}{Notations}
\newtheorem{exi}{Example}

\newcommand\comment[3]{\colorbox{#1}{#2}\marginpar{#3}}
\newcommand\Rabea{\comment{yellow}}
\newcommand\Ludo{\comment{green}}
\newcommand\Eric{\comment{cyan}}
\newcommand\Quentin{\comment{pink}}

\newcommand\nmm[1]{\(\displaystyle #1\)} % nmm for Nice Math Mode
\newcommand\hyp[1]{\TextOrMath{\eqref{hyp:\thesubsection #1}}{\label{hyp:\thesubsection #1}\tag{H#1}}}
\newcommand\goal[1]{\TextOrMath{\eqref{goal:\thesubsection #1}}{\label{goal:\thesubsection #1}\tag{G#1}}}
\newcommand\defitem{\item[\bullet]}

\newcommand\setR{\mathbb{R}}
\newcommand\setZ{\mathbb{Z}}
\newcommand\setN{\mathbb{N}}

\newcommand\choice[1]{\left\{\everymath{\displaystyle}%
	\begin{array}{lr}#1\end{array}\right.}
\newcommand\subbox[1]{{\makebox[.5\width]{\(\scriptstyle #1\)}}}
\newcommand\bigsymb[2][\Large]{\text{#1\nmm{#2}}} % DeclareMathDelimiter?
\newcommand\defnotation{\DOTSB\;{\Colon=}\;}
\newcommand\defobject{\DOTSB\;{\coloneq}\;}
\newcommand\nwedge{\DOTSB\;{\wedge}\;} % when you want to force space
\newcommand\qwedge{\DOTSB\quad{\wedge}\quad}
\newcommand\wrel[4][]{#2 \overset{#1}\leq_{#4} #3}
\newcommand\fvars[1]{\mathit{vars}\mpar{#1}}
\newcommand\fguard[1]{\mathit{guard}\mpar{#1}}
\newcommand\fOT[1]{\mathrm{OT}\mpar{#1}}
\newcommand\fIT[1]{\mathrm{IT}\mpar{#1}}
\newcommand\terms{{\mathcal{T}}}
\newcommand\formulas{{\mathcal{F}}}
\newcommand\actions{\mathcal{A}}
\newcommand\rver[3][\!\!]{#2_{#1#3}}
\newcommand\rterms[1][\emptyset]{\rver{\terms}{#1}}
\newcommand\rformulas[1][\emptyset]{\rver{\formulas}{#1}}
\newcommand\ractions[1][\emptyset]{\rver[]{\actions}{#1}}
\newcommand\values{\mathcal{P}}
\newcommand\reach[1]{\checkmark_{\!\! #1}}
