%\theoremstyle{plain}
\newtheorem{thm}{Theorem}
\newtheorem{lem}{Lemma}
\newtheorem{cor}{Corollary}
\newtheorem{prop}{Proposition}

\newcounter{pc}
%\theoremstyle{definition}
%\newtheorem{defi}{Definition}
%\newtheorem*{noti}{Notations}
%\newtheorem{exi}{Example}

\let\FMproof\proof
\let\endFMproof\endproof
\newenvironment{proofsketch}{\newcommand\oldproofname\proofname \renewcommand\proofname{Proof sketch} \begin{proof}}{\qed\end{proof}\renewcommand\proofname{\oldproofname}}
%\renewenvironment{proof}{\FMproof\stepcounter{pc}}{\endFMproof}

\newcommand\nmm[1]{\(\displaystyle #1\)} % nmm for Nice Math Mode
\newcommand\hyp[1]{\TextOrMath{\eqref{hyp:\thepc #1}}{\label{hyp:\thepc #1}\tag{H#1}}}
\newcommand\goal[1]{\TextOrMath{\eqref{goal:\thepc #1}}{\label{goal:\thepc #1}\tag{G#1}}}
\newcommand\defitem{\item[\bullet]}

%\newcommand\setR{\mathbb{R}}
%\newcommand\setZ{\mathbb{Z}}
%\newcommand\setN{\mathbb{N}}

\newcommand\choice[1]{\left\{\everymath{\displaystyle}%
	\begin{array}{lr}#1\end{array}\right.}
\newcommand\subbox[1]{{\makebox[.5\width]{\(\scriptstyle #1\)}}}
\newcommand\bigsymb[2][\Large]{\text{#1\nmm{#2}}} % DeclareMathDelimiter?
\newcommand\defnotation{\DOTSB\;{\Colon=}\;}
\newcommand\defobject{\DOTSB\;{\coloneq}\;}
\newcommand\nwedge{\DOTSB{\wedge}\;} % when you want to force space
\newcommand\qwedge{\DOTSB\quad{\wedge}\quad}
\newcommand\wrel[4][]{#2 \overset{#1}\leq_{#4} #3}
\newcommand\OA[6]{\left<#1,#2,#3,#4,#5,#6\right>}
\newcommand\OAg[1][]{\OA{S_{#1}}{s_{0#1}}{V_{#1}}{\sigma_{0#1}}{J_{#1}}{T_{#1}}}
\newcommand\fvars[1]{\mathit{vars}\mpar{#1}}
\newcommand\fguard[1]{\mathit{guard}\mpar{#1}}
\newcommand\fOT[2][]{\mathrm{OT}_{#1}\mpar{#2}}
\newcommand\fIT[2][]{\mathrm{IT}_{#1}\mpar{#2}}
\newcommand\terms{{\mathcal{T}}}
\newcommand\formulas{{\mathcal{F}}}
\newcommand\actions{\mathcal{A}}
\newcommand\rver[3][\!\!]{#2_{#1#3}}
\newcommand\rterms[1][\emptyset]{\rver{\terms}{#1}}
\newcommand\rformulas[1][\emptyset]{\rver{\formulas}{#1}}
\newcommand\ractions[1][\emptyset]{\rver[]{\actions}{#1}}
\newcommand\values{\mathcal{P}}
\newcommand\reach[1]{\checkmark_{\!\! #1}}



\usepackage{epsfig,color,subfigure,enumitem,soul}
\newcommand{\TODO}[1]{\textcolor{red}{\textbf{[TODO:#1]}}}
\newcommand{\NOTE}[1]{\textcolor{blue}{\textbf{[NOTE:#1]}}}
\newcommand{\ERIC}[1]{\textcolor{blue}{#1}}
\definecolor{darkgreen}{rgb}{0.1, 0.5, 0.1}
\newcommand{\LUDO}[1]{\textbf{\textcolor{purple}{#1}}}
\newcommand{\RAB}[1]{\textcolor{magenta}{#1}}
