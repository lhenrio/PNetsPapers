\documentclass[runningheads]{llncs}

\usepackage{tikz}

\usepackage{amsmath}
%\usepackage{amsthm}
%\usepackage{unicode-math}
\usepackage{tikz}
\usetikzlibrary{external}
\usepackage{hyperref}
\AtBeginDocument{\renewcommand\setminus{\smallsetminus}}


\newcommand{\dbllbrack}{\{\hspace{-0.88ex}\{}
\newcommand{\dblrbrack}{\}\hspace{-0.87ex}\}}

\newcommand\mpar[1]{{\left(#1\right)}}
\newcommand\mbrk[1]{{\left[#1\right]}}
\newcommand\mbrc[1]{{\left\{#1\right\}}}
\newcommand\mdbrk[1]{\left\ldbrack#1\right\rdbrack}
\newcommand\card[1]{{\left|#1\right|}}
\newcommand\psubst[1]{\mbrc{\!\!\mbrc{#1}\!\!}}
\newcommand\subst[2]{\mbrk{#1\middle/#2}}
\newcommand\midbar{\,\middle|\,}
\newcommand\mset[2]{\mbrc{#1\midbar #2}}

\newcommand\act{\mathrm}
%\newcommand\OTbase[4]{\text{\small\(%
%	\setlength\arraycolsep{2pt}\everymath{\displaystyle}%
%	\renewcommand\arraystretch{1}%
%	\begin{array}{c}%
%	#2,#3,#4 \\\noalign{\color{red}\hrule\vspace{1pt}\hrule}
%	#1
%	\end{array}\)}}
\newcommand\OTbase[4]{\text{\small\(%
	\setlength\arraycolsep{2pt}\everymath{\displaystyle}%
	\renewcommand\arraystretch{1}%
	\begin{array}{c}%
	#2,#3,#4 \\\noalign{\color{red}\hrule}
	#1
	\end{array}\)}}

\newcommand\OThelperdonotuse[2][\top]{\OTtemporary{#1}{\mbrc{#2}}}
\newcommand\OTd[2]{\newcommand\OTtemporary{\OTbase{#1}{\mbrc{#2}}}\OThelperdonotuse}

\newcommand\OT[3]{\OTbase{#1\xrightarrow{\raisebox{-1.5pt}[.8\height][0pt]{\makebox[1.4\width]{\(\scriptstyle #3\)}}}#2}}
\newcommand\OTx[4]{\OT{s_{#1}}{s'_{#1 #2}}{\alpha_{#1 #2}}{\beta_{#3 j}^{j \in J'_{#4}}}{g_{#1 #2}}{\psi_{#1 #2}}}
\newcommand\OTg{\OTx{}{}{}{}}

\theoremstyle{plain}
\newtheorem{thm}{Theorem}
\newtheorem{lem}{Lemma}
\newtheorem{cor}{Corollary}
\newtheorem{prop}{Proposition}

\theoremstyle{definition}
\newtheorem{defi}{Definition}
\newtheorem{exi}{Example}

\newcommand\comment[3]{\colorbox{#1}{#2}\marginpar{#3}}
\newcommand\Rabea{\comment{yellow}}
\newcommand\Ludo{\comment{green}}
\newcommand\Eric{\comment{cyan}}
\newcommand\Quentin{\comment{pink}}

\newcommand\nmm[1]{\(\displaystyle #1\)} % nmm for Nice Math Mode
\newcommand\hyp[1]{\TextOrMath{(H#1)}{\tag{H#1}}}
\newcommand\goal[1]{\TextOrMath{(G#1)}{\tag{G#1}}}
\newcommand\defitem{\item[\bullet]}

\newcommand\setR{\mathbb{R}}
\newcommand\setZ{\mathbb{Z}}
\newcommand\setN{\mathbb{N}}

\newcommand\choice[1]{\left\{\everymath{\displaystyle}%
	\begin{array}{lr}#1\end{array}\right.}
\newcommand\subbox[1]{{\makebox[.5\width]{\(\scriptstyle #1\)}}}
\newcommand\bigsymb[2][\Large]{\text{#1\nmm{#2}}} % DeclareMathDelimiter?
\newcommand\defnotation{\text{ defined as }}
\newcommand\defobject{\DOTSB\;{\coloneq}\;}
\newcommand\nwedge{\DOTSB\;{\wedge}\;} % when you want to force space
\newcommand\qwedge{\DOTSB\quad{\wedge}\quad}
\newcommand\wrel[4][]{#2 \overset{#1}\leq_{#4} #3}
\newcommand\fvars[1]{\mathit{vars}\mpar{#1}}
\newcommand\fguard[1]{\mathit{guard}\mpar{#1}}
\newcommand\fOT[1]{\mathrm{OT}\mpar{#1}}
\newcommand\terms{{\mathcal{T}}}
\newcommand\formulae{{\mathcal{F}}}
\newcommand\actions{{\mathbb{A}}}
\newcommand\rver[3][\!\!]{#2_{#1#3}}
\newcommand\rterms[1][\emptyset]{\rver{\terms}{#1}}
\newcommand\rformulae[1][\emptyset]{\rver{\formulae}{#1}}
\newcommand\ractions[1][\emptyset]{\rver[]{\actions}{#1}}
\newcommand\values{\mathcal{P}}
\newcommand\labels{\mathcal{A}}
\newcommand\reach[1]{\checkmark_{\!\! #1}}

\tikzstyle{every edge} = [draw,-latex,every node/.style={auto}]
\tikzstyle{state} = [draw,circle,minimum size=1cm,inner sep=1mm]
\tikzstyle{initial} = [double,double distance=1mm,minimum size=.95cm,inner sep=.5mm,outer sep=.5mm]
\tikzstyle{init} = [double,double distance=1.5pt,edge label=\(\mbrc{#1}\)]
\tikzstyle{dir} = [out=#1+15,in=#1-15,looseness=10]


\begin{document}
%
\title{Refinements for open pNets}
%
%\titlerunning{Abbreviated paper title}
% If the paper title is too long for the running head, you can set
% an abbreviated paper title here
%
\author{
Rabéa\inst{1}\orcidID{1111-2222-3333-4444} \and
Quentin\inst{1,2}\orcidID{1111-2222-3333-4444} \and
Ludo\inst{2}\orcidID{0000-1111-2222-3333} \and
Eric\inst{3}\orcidID{2222--3333-4444-5555}}
%
\authorrunning{Rabéa Ameur-Boulifa et al.}
% First names are abbreviated in the running head.
% If there are more than two authors, 'et al.' is used.
%
\institute{blabla
\email{lncs@fff}\\
\url{http://www.springer.com/gp/computer-science/lncs} \and
blibli\\
\email{\{abc,def\}@fff}}
%
\maketitle              % typeset the header of the contribution
%
\begin{abstract}
The abstract should briefly summarize the contents of the paper in
150--250 words.

\keywords{Labelled transition systems  \and Refinement \and Composition.}
\end{abstract}
%
%
%
\section{Introduction}
\TODO{1.5 pages}


\section{Related Work}
\label{sec:sota}

\TODO{1 page}

Previous work on open automata focused on equivalence relations compatible with composition.
In an article by Hou, Zechen and Madelaine \cite{10.1145/3372884.3373161}, a computable bisimulation is introduced and proved equivalent to the previous bisimulation already introduced.
In a more recent work by Ameur-Boulifa, Henrio and Madelaine \cite{2007.10770}, a weak version of the bisimulation on open automata is introduced.
These works differ from ours because the relation introduced in this report is a refinement relation in the form of a simulation and not a bisimulation.
Also we do not have results as strong as computability neither a weak version able to tackle silent actions.

Some related work on other models than open automata introduce refinement relations.
In a chapter by Bellegarde, Julliand and Kouchnarenko \cite{10.1007/3-540-46428-X_19}, a simulation relation on transition systems is introduced.
This simulation encompass action refinement, is able to deal with silent actions and is compatible with parallel composition.
Here the refinement relation does not consider action refinement as valid but it should be done in future work.
Also they check how LTL properties are preserved or combined using ther refinement which we do not do.
However their model is less expressive: the transition system model is less expressive than open automata and the parallel composition is less expressive than composition on open automata.
In a later report by Kouchnarenko and Lanoix \cite{10.1007/978-3-540-70881-0_26}, the refinement relation they introduce is on LTS (labelled transition systems).
Their relation additionally prevents deadlock and livelocks.
The compositon is also extended to synchronised composition which is more expressive.
In our work we also deal with deadlocks but not with livelocks since the latter arise only with silent actions.
This work is closer than the previous one to what we do here, still open automata are more expressive than LTS and composition is more general than sychronised composition.

A refinement relation on models nearer to open automata is introduced in an article by Zhang, Meng and Lo \cite{Zhang2014}.
In their article they work with transition systems with variable which makes the state space potentially infinite.
This aspect is also present in open automata.
They show how invariants, a notion near to our reachability predicates, are composed.
By relation on these invariants they introduce several refinement relations.
We could have done something similar for non-locking composition reachability predicates which are introduced in Section \ref{sec:proofelts}.

On the deadlock prevention aspect, an article by Dihego, Sampaio and Oliveira \cite{DIHEGO2020110598} present a refinement relation on process algebra (translated to LTS).
This refinement relation is a special case of inheritance and prevents the introduction of deadlocks.
Their refinement and inheritance are quite the opposite of our refinement in terms of new behaviours.
They have channels, interfaces, inputs and outputs, which in the open automata model can be compared to action labels, holes and action data for both inputs and outputs.
They have a rich composition as open automata but the introduction of deadlock is already prevented by a well chosen set of composing operations.
Also their composition is slightly different than the one on open automata because they can cause loops by linking two channels of the same process, where in open automata the composition makes an oriented tree.
In their model there is an explicit deadlock and a succesful termination where in open automata there are no explicit termination.
We define a deadlock as a configuration without possible transition and assume what is a deadlock when comparing the open automata.
To define their refinement and inheritance relation they use trace and failure semantics, which are weaker than (bi)simulations \cite{10.5555/640428.640430} and could break with open automata compositon.


\section{Background: Open Automata and their Composition}\label{sec:background}
\TODO{3.5 pages}

\section{A First Refinement Relation}\label{sec:refinement}
\TODO{4 pages}

\section{A Refinement Relation that Takes Holes into Account}\label{sec:holes}
\TODO{3.5 pages}

\section{Properties}\label{sec:prop}
\TODO{1.5 pages}

\section{Conclusion}\label{sec:ccl}
\TODO{0.5 pages}


 \bibliographystyle{splncs04}
 \bibliography{biblio}

\end{document}