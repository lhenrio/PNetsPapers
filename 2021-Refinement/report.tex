\documentclass{article}

\usepackage{xunicode}
\usepackage{fontspec}
\usepackage[hmargin=3cm]{geometry}
\usepackage{amsmath}
\usepackage{amsthm}
\usepackage{unicode-math}
\usepackage{tikz}
\usetikzlibrary{external}
\usepackage[colorlinks]{hyperref}
\AtBeginDocument{\renewcommand\setminus{\smallsetminus}}
\usepackage{polyglossia}
\setmainlanguage{english}

\newcommand{\dbllbrack}{\{\hspace{-0.88ex}\{}
\newcommand{\dblrbrack}{\}\hspace{-0.87ex}\}}

\newcommand\mpar[1]{{\left(#1\right)}}
\newcommand\mbrk[1]{{\left[#1\right]}}
\newcommand\mbrc[1]{{\left\{#1\right\}}}
\newcommand\mdbrk[1]{\left\ldbrack#1\right\rdbrack}
\newcommand\card[1]{{\left|#1\right|}}
\newcommand\psubst[1]{\mbrc{\!\!\mbrc{#1}\!\!}}
\newcommand\subst[2]{\mbrk{#1\middle/#2}}
\newcommand\midbar{\,\middle|\,}
\newcommand\mset[2]{\mbrc{#1\midbar #2}}

\newcommand\act{\mathrm}
%\newcommand\OTbase[4]{\text{\small\(%
%	\setlength\arraycolsep{2pt}\everymath{\displaystyle}%
%	\renewcommand\arraystretch{1}%
%	\begin{array}{c}%
%	#2,#3,#4 \\\noalign{\color{red}\hrule\vspace{1pt}\hrule}
%	#1
%	\end{array}\)}}
\newcommand\OTbase[4]{\text{\small\(%
	\setlength\arraycolsep{2pt}\everymath{\displaystyle}%
	\renewcommand\arraystretch{1}%
	\begin{array}{c}%
	#2,#3,#4 \\\noalign{\color{red}\hrule}
	#1
	\end{array}\)}}

\newcommand\OThelperdonotuse[2][\top]{\OTtemporary{#1}{\mbrc{#2}}}
\newcommand\OTd[2]{\newcommand\OTtemporary{\OTbase{#1}{\mbrc{#2}}}\OThelperdonotuse}

\newcommand\OT[3]{\OTbase{#1\xrightarrow{\raisebox{-1.5pt}[.8\height][0pt]{\makebox[1.4\width]{\(\scriptstyle #3\)}}}#2}}
\newcommand\OTx[4]{\OT{s_{#1}}{s'_{#1 #2}}{\alpha_{#1 #2}}{\beta_{#3 j}^{j \in J'_{#4}}}{g_{#1 #2}}{\psi_{#1 #2}}}
\newcommand\OTg{\OTx{}{}{}{}}

\theoremstyle{plain}
\newtheorem{thm}{Theorem}
\newtheorem{lem}{Lemma}
\newtheorem{cor}{Corollary}
\newtheorem{prop}{Proposition}

\theoremstyle{definition}
\newtheorem{defi}{Definition}
\newtheorem{exi}{Example}

\newcommand\comment[3]{\colorbox{#1}{#2}\marginpar{#3}}
\newcommand\Rabea{\comment{yellow}}
\newcommand\Ludo{\comment{green}}
\newcommand\Eric{\comment{cyan}}
\newcommand\Quentin{\comment{pink}}

\newcommand\nmm[1]{\(\displaystyle #1\)} % nmm for Nice Math Mode
\newcommand\hyp[1]{\TextOrMath{(H#1)}{\tag{H#1}}}
\newcommand\goal[1]{\TextOrMath{(G#1)}{\tag{G#1}}}
\newcommand\defitem{\item[\bullet]}

\newcommand\setR{\mathbb{R}}
\newcommand\setZ{\mathbb{Z}}
\newcommand\setN{\mathbb{N}}

\newcommand\choice[1]{\left\{\everymath{\displaystyle}%
	\begin{array}{lr}#1\end{array}\right.}
\newcommand\subbox[1]{{\makebox[.5\width]{\(\scriptstyle #1\)}}}
\newcommand\bigsymb[2][\Large]{\text{#1\nmm{#2}}} % DeclareMathDelimiter?
\newcommand\defnotation{\text{ defined as }}
\newcommand\defobject{\DOTSB\;{\coloneq}\;}
\newcommand\nwedge{\DOTSB\;{\wedge}\;} % when you want to force space
\newcommand\qwedge{\DOTSB\quad{\wedge}\quad}
\newcommand\wrel[4][]{#2 \overset{#1}\leq_{#4} #3}
\newcommand\fvars[1]{\mathit{vars}\mpar{#1}}
\newcommand\fguard[1]{\mathit{guard}\mpar{#1}}
\newcommand\fOT[1]{\mathrm{OT}\mpar{#1}}
\newcommand\terms{{\mathcal{T}}}
\newcommand\formulae{{\mathcal{F}}}
\newcommand\actions{{\mathbb{A}}}
\newcommand\rver[3][\!\!]{#2_{#1#3}}
\newcommand\rterms[1][\emptyset]{\rver{\terms}{#1}}
\newcommand\rformulae[1][\emptyset]{\rver{\formulae}{#1}}
\newcommand\ractions[1][\emptyset]{\rver[]{\actions}{#1}}
\newcommand\values{\mathcal{P}}
\newcommand\labels{\mathcal{A}}
\newcommand\reach[1]{\checkmark_{\!\! #1}}

\tikzstyle{every edge} = [draw,-latex,every node/.style={auto}]
\tikzstyle{state} = [draw,circle,minimum size=1cm,inner sep=1mm]
\tikzstyle{initial} = [double,double distance=1mm,minimum size=.95cm,inner sep=.5mm,outer sep=.5mm]
\tikzstyle{init} = [double,double distance=1.5pt,edge label=\(\mbrc{#1}\)]
\tikzstyle{dir} = [out=#1+15,in=#1-15,looseness=10]

% transitions between sections
% TODO: CITER LES CONFERENCES

\title{Refinement for open automata}
\author{Quentin \textsc{Corradi}}

\begin{document}
\maketitle

\section{Introduction}
The open automata model was originally designed to give the semantic of open pNets \cite{2007.10770}.
Open pNets are not petri nets but an intermediate representation used in the VerCors project to perform verification \cite{henrio:01252323}.
This model is based around a hierarchical composition of generalised automata.
Verification is the process of checking properties on a description in a specific model (for instance an automaton).
This properties can be for instance being in some type or being deadlock free (a deadlock is an unintended state where no further action is possible).
Open pNets with the open automaton semantics and equivalence relation introduced in previous work \cite{2007.10770} were sucessfully used to model and verify some properties of applications \cite{qin:01823507, ameurboulifa:01526055} based on BIP (Behaviour-Interaction-Priority) \cite{basu:hal-00722395} and GCM (Grid Component Model) \cite{baude:inria-00323919}.
However equivalence relations are not always sufficient to perform verification, refinement relations are also used.
A refinement is the verification that an implementation satisfies a specification, whereas an equivalence would check that their behaviours are the same.
For instance a usecase on open pNet where equivalence is not sufficient can be found in an article about algorithms fo weak bisimulation on open automata \cite{wang:03126313}.

The contributions of this report are:
\begin{enumerate}
\item The formalisation of composition for open automata without requiring open pNets.
\item The translation of some properties of refinement relations from LTS to open automata.
\item The characterisation of a composition that doesn't introduce deadlocks.
\item The introduciton of several refinement relations for open automata that behave nicely with composition.
\end{enumerate}
% TODO: Related work: thesis

We begin with Section \ref{sec:notations} by giving some notations used throughout this paper.
Then there are several sections dedicated to define all the interesting objects and properties:
In Section \ref{sec:def} we give a clean and up to date definition of open automata.
In Section \ref{sec:comp} we define their composition without requiring the use of open pNets \Ludo{}{does this mean that previous definitions were using open pnets? this might be clarified in contributions but when you redo a definition you must highlight why a new definition was necessary}\Quentin{}{Yes, composition of open automata was composition of the open pNets generating them, the necessity is as stated, being independant from open pNets}.
In Section \ref{sec:proofelts} we adapt to open automata several standard properties from refinement relations on LTS.
After that Section \ref{sec:prelref} is dedicated to progressively build a refinement relation and explain the elements of this construction.
Section \ref{sec:refinement} introduces and analyses the refinement relation on open automata.
Finally Section \ref{sec:ccl} recapitulates the contribution and presents the work left.


\section{Notations}\label{sec:notations}
Notations will be defined with the operator \(\defnotation\) and names are given with the operator \(\defobject\) as follows:
\begin{align*}
	\mathit{notation\_with\_variables} & \defnotation \mathit{notated\_object\_using\_the\_variables} \\
	\mathit{name} & \defobject \mathit{fully\_defined\_mathematical\_object}
\end{align*}

Throughout this paper, tuples will be noted differently depending on what they represent.
This helps distinguishing the manipulated objects.
Every such notation will be introduced in the definition of the object.

Families of values, or equivalently maps will be noted \(\mset{i \mapsto x_i}{i \in I}\), \(\mset{i \gets x_i}{i \in I}\) or \(x_i^{i \in I}\).
The latter can only be used if there is a \Ludo{generating expression}{c'est vrai pour les exemples mais pas pour les defs generiques (OA, bisim, ...): reformuler}\Quentin{}{Je pense que c'est aussi vrai pour les defs génériques, on définit une famille à l'aide d'une expression génératrice, par exemple \(\mpar{\beta_j}_{j \in J}\) est en réalité \(j \mapsto \beta_j\); une fonction est une expression génératrice}; for instance \(\mpar{ax}^{x \in \setR}\) represents a scaling function, \(c^{i \in I}\) is a constant function over \(I\).
However \(\mbrc{\alpha \mapsto 1, \beta \mapsto 2, \gamma \mapsto 3}\) has no generating expression and is represented here with the finite version of first notation.

The disjoint union on set is noted \(\uplus\).
There are several ways of ensuring a union is disjoint, we will indifferently either suppose sets are disjoint or rename conflicting object (useful for variables).
The disjoint union of two maps \(\varphi: I \to X\) and \(\psi: J \to Y\) with \(I \cap J = \emptyset\) has the following signature \(\varphi \uplus \psi: I \uplus J \to X \cup Y\).

In a formula, a quantifier followed by a finite set will be used as a shorthand for the quantification on every variable in the set:
\(\forall \mbrc{a_1, \dots, a_n}, \exists \mbrc{b_1, \dots, b_m}, P\) means \(\forall a_1, \dots, \forall a_n, \exists b_1, \dots, \exists b_m, P\).


\section{Open Automata}\label{sec:def}
To define the open automata we need some preliminary definitions.
\begin{defi}[Expression algebra, Action algebra, Formulas, Terms]
An expression algebra \(E\) is a disjoint union \(E \defobject \terms \uplus \actions \uplus \formulas\) of the terms, the actions and the formulas.

Notations are parameterised by a term algebra \(\terms\).
As any term algebra it has constant symbols with arity, variables, and a typing mechanism to distinguish well-formed and ill-formed terms.
The action algebra \(\actions\) is another term algebra.
It \Ludo{can be}{not much informative}\Quentin{}{Intended use, it is not mandatory} a subset of the terms.
The formulas \(\formulas\) are at least the first order formulas over \(\terms\) and \(\actions\).
\end{defi}
The term algebras are arbitrary.
The formulas contain at least first order logic terms with an equality relation.
This equality relation is not necessarily a syntactic equality (\(2 + 2 \neq 4\) with a syntactic equality but it is more expressive to consider that the two terms are equal).

An example of term algebra can be Peano integers (constant zero with arity 0, constant successor function with arity 1 and the variables), the formulas associated can use syntactic equality relation, the sum relation \(\mathit{sum}\mpar{a, b, c}: \text{``}a = b + c\text{"}\), and the product relation \(\mathit{prod}\mpar{a, b, c}: \text{``}a = b \times c\text{"}\).

\begin{defi}[Unbound variables, Expressions restricted to variables, Closed expressions]
\defitem \(\fvars{e}\) is the set of variables in \(e \in E\) that are not bound by any binder.
	Binders can be for instance quantifiers in formulas, or let-binders in terms if they are part of the term algebra.
\defitem The expressions restricted to variables in \(V\) are \(E_V \defnotation \mset{e \in E}{\fvars{e} \subseteq V}\); \(E_V \subset E\).
\defitem The closed expressions are expressions restricted to variables in \(\emptyset\), \(E_\emptyset\).
\end{defi}
Terms, actions and formulas restricted to variables and their closed versions are also defined by restriction.
A closed expression can contain variables under a binder, only unbound variable are forbidden.

We can use the previous example of first order formula on Peano integers to illustrate these definitions.
\(x\), \(S\mpar{y}\) are well-formed terms, \(0\) is a closed term, \(\fvars{x} = \mbrc{x}\), \(\fvars{S\mpar{y}} = \mbrc{y}\), \(\fvars{0} = \emptyset\).
\(x = 0\), \(\forall y, \exists x, \mathit{sum}\mpar{y, z, x}\) are valid formulas, \(\forall x, \neg S\mpar{x} = 0\) is a valid closed formula, \(\fvars{x = 0} = \mbrc{x}\), \(\fvars{\forall y, \exists x, \mathit{sum}\mpar{y, z, x}} = \mbrc{z}\), \(\fvars{\forall x, \neg S\mpar{x} = 0} = \emptyset\).

\begin{defi}[Values, Satisfiability, (Parallel) substitution]
We assume that the following are given:
\defitem The values \(\values\), which are interpretations of closed terms.
\defitem The satisfiability relation on closed formulas, \({\vdash} f\) where \(f \in \rformulas\).
\defitem The substitution in \(e \in E\) of \(x \in \fvars{e}\) by \(t \in \terms\), \(e\subst{t}{x}\).
\defitem The parallel substitution in \(e \in E\) of variables in \(V\) by \(\psi: V \to \terms\), \(e\psubst{\psi}\).
\end{defi}
For the parallel substitution, the set \(V\) is not required to be a subset of \(\fvars{e}\).
In the case it isn't, the variables in \(V \setminus \fvars{e}\) are not substituted.
The substitutions might give a ill-formed expression; for instance let the terms be integers and pairs with (pointwise) addition, \(\mpar{a + b}\psubst{a \mapsto 7, b \mapsto \mpar{4, 5}}\) is a ill-formed term.
This can be guarded with the check \(e\subst{t}{x} \in E\) and \(e\psubst{\psi} \in E\) and it will implicitly be the case, for instance when there is quantification on \(t\) and \(\psi\), to simplify notations.

The interpretation of terms is supposed to be decidable.
The satisfiability of formulas might not be decidable nor complete nor consistent,
but we will consider in the following that they are; this can be achieved by restricting the term algebra to a decidable subset.
For instance a formula with quantifiers on variables might not be provable even if it is true for all values of these variables.
In practice the formulas will be given to a SMT solver and we cannot always make sure they have all the previous properties but, as soon as the satisfiability is consistent our theory is consistent too.
\(\vdash\) can hence be interpreted as an indicator of what is given to the SMT; it separates the external logic and the logic on \(\formulas\).

Values will be used for keeping a variable state, and then injected in terms for substitution.
This is correct when \(\values \subseteq \rterms\) and we suppose it is the case.
Otherwise it doesn't invalidates theorems because it can stand as a shorthand for substitution with any term which has the kept value as an interpretation.
We suppose that the interpretation of terms is compatible w.r.t.\@ substitution, that is if two terms \(t\), \(t'\) are interpreted with the same value, then replacing \(t\) by \(t'\) in a well-formed expression makes an equivalent well-formed expression.
\begin{noti}[Notations for separating external logic and logic on \(\formulas\)]
\defitem The satisfiability of a formula \(f \in \formulas\) under some valuation \(\sigma: V \to \values\) is noted:
\[ \sigma \vdash f \defnotation \vdash \exists \fvars{f\psubst{\sigma}}, f\psubst{\sigma} \]
\defitem The satisfiability of a formula \(f \in \formulas\) with some variable set \(V\) as context is noted:
\[ V \vdash f \defnotation \vdash \forall V, \exists\mpar{\fvars{f} \setminus V}, f \]
\defitem The precedence of \(\vdash\) is the lowest on the right side and higher than \(\uplus\) on the left side:
\[ \forall a \, b, a \uplus b \vdash x \wedge y \implies \exists z, P\mpar{x,z} \text{ is the same as } \forall a \, b, \bigg((a \uplus b) \vdash \Big((x \wedge y) \implies \exists z, P\mpar{x,z}\Big)\bigg) \]
\end{noti}
With these common definitions and notations settled, the objects of interest can now be defined.
\begin{defi}[Open automaton]
A open automaton is a tuple \(\OAg\) with \(S\) the set of states, \(s_0 \in S\) the initial state, \(V\) the set of variable names unique to this automaton, \(\sigma_0: V' \to \values\) the initial valuation of variables where \(V' \subseteq V\), \(J\) the set of hole names and \(T\) the set of open transitions.
A pair of a state and a valuation is called a configuration.

\(S, V, J\) are arbitrary finite sets. % OR finite arbitrary sets?
\end{defi}
The variable names may clash when considering two automata, in this case we suppose that we can still distinguish the variables in the formulas.
In practice the open automata are used in a toolchain that guarantees there are no name clash.

The initial valuation is a partial valuation of the variables.
If an undefined variable is set before being read, it behaves like an initially set variable.
If an undefined variable is not set before being read then its value may be any fixed value.
When we will introduce refinements in Section \ref{sec:prelref}, setting an undefined variable will be considered a valid refinement, for instance a \(5\) bits register is a particular \(n\) bits register.
\begin{defi}[Open transition]
An open transition is a tuple \nmm{\OTg} with \(s, s' \in S\) the source and target states, \(\alpha \in \actions\) the produced action, \(J' \subseteq J\) the holes involved in the transition, \(\beta_j \in \actions\) the actions of the holes, \(g \in \formulas\) the guard and \(\psi: V \to \terms\) the variable assignments.
\end{defi}
An open transition can have many unbound variables.
Actually an effective transition of the automaton is any well-formed substitution of the unbound variables of the transition minus the automaton variables.

The intuition of an open automaton is a partially defined LTS with variables, guards on transitions, and parametrised actions. The LTS is partial in the sense that one cn plug an LTS in the hole and the surrounding LTS will react to the actions emitted by the sub-LTS put in the hole.
Initially the automaton is in the initial state with an extension on all the variables of the initial valuation.
In any configuration, it can perform effective transitions which source state is the current state and which guard is satisfiable in the current valuation if the holes emit the indicated actions.
An effective transition is a transition where every variable that is not an automaton variable is instantiated with a value.
By performing the effective transition, the automaton emits the indicated action and updates its configuration according to the target state and variable assignments.

\begin{noti}[FH-bisimulation]
The FH-bisimulation \cite{henrio:01055091} is noted \(\cong\).

The FH-bisimulation is currently the only equivalence relation on open automata.
The definition is not important, but the interested reader can check the article cited.
\end{noti}
To illustrate these definitions we consider two implementations (figure \ref{fig:enable}) of the LOTOS \cite{ISOLOTOS} operator enable in the open automata model.
The enable operator runs the left hand side agent until it chooses to finish, at which point it produces an action \(\delta\mpar{t}\) (with \(t\) some data) that is synchronised with the first action of the right hand side agent which must be \(\act{accept}\mpar{t}\) (here \(t\) is an input), then only the latter agent runs.
The \(\act{accept}\) action is shortened as \(\act{acc}\) in the examples.
During the synchronised action, the value returned by the first agent is passed to the second agent.
\begin{exi}[Enable, state-oriented]
Graphical convention for drawing automata are as follows.
As standard automata, circles represent states and simple arrows represent transitions.
The initial state is indicated by a double circle.
States names are indicated inside the circles and transitions labels are drawn near their corresponding arrow.
The open transitions do not indicate the source and target states since that is the role of the transitions arrows; only the emmited action is on the bottom side of open transitions.
Initial valuations are indicated near a double lined arrow pointing to the initial state.

\begin{figure}
\centering
\begin{tikzpicture}

\node[state,initial] (v1) at (0,1.5) {L};
\node[state] (v2) at (0,-1.5) {R};
\draw (v1) edge[dir=0] node {\OTd{x}{l \mapsto x}[x \neq \delta]{}} (v1);
\draw (v1) edge node {\OTd{\tau}{l \mapsto \delta\mpar{x}, r \mapsto \mathrm{acc}\mpar{y}}[x = y]{}} (v2);
\draw (v2) edge[dir=0] node {\OTd{x}{r \mapsto x}{}} (v2);

\end{tikzpicture}

\vrule
\begin{tikzpicture}

\node[state,initial] (v1) at (0,0) {};
\draw (v1.north) ++(0,0.75) edge[init={v \gets l}] (v1);
\draw (v1) edge[dir=0] node {\OTd{x}{l \mapsto x}[v = l  \wedge x \neq \delta]{}} (v1);
\draw (v1) edge[dir=270] node {\OTd{\tau}{l \mapsto \delta\mpar{x}, r \mapsto \mathrm{acc}\mpar{y}}[x = y \wedge v = l]{v \gets r}} (v1);
\draw (v1) edge[dir=180] node {\OTd{x}{r \mapsto x}[v=r]{}} (v1);

\end{tikzpicture}

\caption{Enable operator implementation with open automata, on the left state oriented, on the right data oriented}
\label{fig:enable}
\end{figure}
The automaton on the left side of the figure is \(\OA{\mbrc{L, R}}{L}{\emptyset}{\mbrc{}}{\mbrc{l, r}}{T}\) where transitions in \(T\) are:
\begin{align*}
	\OT{L}{L}{x}{\mbrc{l \mapsto x}}{\forall y, x \neq \delta\mpar{y}}{\mbrc{}} &&
	\OT{L}{R}{\tau}{\mbrc{l \mapsto \delta\mpar{x}, r \mapsto \act{acc}\mpar{y}}}{x = y}{\mbrc{}} &&
	\OT{R}{R}{x}{\mbrc{r \mapsto x}}{\top}{\mbrc{}}
\end{align*}
Note that the transition in the middle could have been expressed as \(\mbrc{l \mapsto \delta\mpar{x}, r \mapsto \act{acc}\mpar{x}}\).
It would have avoided many effective transitions with trivially false guards like the one where \(x \mapsto 1, y \mapsto 2\), which has the guard \(1 = 2\). \Ludo{}{I believe that with the usual definition of the semantics of open automata it is the same ... please check: you do all potential transitions with the conditions that x is the same in both. In any case as the matching is not linear, the simplification is not obvious}\Quentin{}{Same behaviour as here with the standard definition: RR-9177, sec 5.2}

This automaton is an implementation of the enable operator because it begins in the state \(L\), where it allows any non \(\delta\) transition from its hole \(l\), then the automaton synchronises its holes on the same data, effectively allowing a data exchange when it goes into state \(R\), and finally allows any transition from its hole \(r\).
The implementation of value passing in the open automata model doesn't distinguish input and output variables as in LOTOS.
Both holes must emit their actions with valid data to perform the transition in the enable operator.
If it is not possible the system is locked.

We use the standard convention that uses \(\tau\) as a non-observable transition, never synchronised with other actions and passed unmodified.
This allows the synchronisation of the two holes to be hidden to the exterior by sending a \(\tau\).
However here \(\tau\) is not always allowed from the holes, for instance in the state \(L\) the hole \(r\) cannot emit it.
These transitions have been omitted for the sake of simplifying the first example of open automata.
\end{exi}
Finally we can define some utilitary functions:
\begin{defi}[Guard, Out-transition, Transition variables]
Let \(V\) be the variable names of the considered automaton, \(T\) its transitions and \(r\) one of its states.
\(\fOT{r}\) are called the out-transitions of \(r\).
\(\fIT{r}\) are called the in-transitions of \(r\).
The local variables of a transition are all variables appearing in that transition except the global variables of the automaton.
\begin{align*}
	\fOT{r} & \defnotation \mset{\OTg \in T}{s = r} &
	\fIT{r} & \defnotation \mset{\OTg \in T}{s' = r} \\
\end{align*}
\vspace{-1cm}
\begin{gather*}
	\fguard{\OTg} \defnotation g \\
	\fvars{\OTg} \defnotation \mpar{\fvars{\alpha} \cup \fvars{g} \cup \bigcup_{j \in J'} \fvars{\beta_j} \cup \bigcup_{v \in V} \fvars{\psi\mpar{v}}} \setminus V
\end{gather*}
\end{defi}
The goal of the extractor \(\fvars{t}\) when \(t\) is a transition is to get variables that are only present in that transition.
Also transition variables sets are disjoint from one transition to another.
Other variables like automaton variables are already known at this point so there is no use in getting their name again.
Also the use of this extractor benefits from this exclusion: otherwise there would always be \(\setminus V\) following it to prevent variable shadowing.

\Ludo{}{is it really variable shadowing?}\Quentin{}{I don't know if the term is the same in math}

\Ludo{}{you do not necessarily need to explain everything if this is what you need, but you miss an intuition of what this set is}\Quentin{}{I don't know what else I could say to give the intuition, all elements are here, in my opinion, the only thing left to build the intuition is an example}

\begin{exi}[Enable, variable-oriented]
The automaton drawn on the right side of Figure \ref{fig:enable} is \(\OA{\mbrc{»}}{»}{\mbrc{v}}{\mbrc{v \mapsto l}}{\mbrc{l, r}}{T}\) where transitions in \(T\) are:
\begin{align*}
	\OT{»}{»}{x}{\mbrc{l \mapsto x}}{v = L \wedge x \neq \delta}{\mbrc{}} &&
	\OT{»}{»}{\tau}{\mbrc{l \mapsto \delta\mpar{x}, r \mapsto \act{acc}\mpar{y}}}{v = L \wedge x = y}{\mbrc{v \gets R}} &&
	\OT{»}{»}{x}{\mbrc{r \mapsto x}}{v = R}{\mbrc{}}
\end{align*}
It is an alternative implementation of the enable operator which is \Ludo{FH-bisimilar}{makes no sense: FH bisimlar is still undefined, neither formally nor informally} to the previous one \cite{henrio:01299562}.\Quentin{}{I can remove this sentence, it adds nothing}
The variables that are said to be local to a transition are all the unbound variables in the expressions minus the automaton variables.
For instance \(v\) is a variable of the automaton (used in the guard).
It is not local to the transition.
When taking effective transitions, its value is not substituted with a closed term.
For instance \nmm{\OT{»}{»}{\tau}{\mbrc{l \mapsto \tau}}{v = L \wedge \tau \neq \delta}{\mbrc{}}} is an effective transition generated from the second transition but not \nmm{\OT{»}{»}{\tau}{\mbrc{l \mapsto \tau}}{L = L \wedge \tau \neq \delta}{\mbrc{}}} (result of the forbidden substitution \(v \mapsto L\)).

An run of this open automaton can be: The automaton in the hole \(l\) emits many actions, this synchronises with the only transition possible at that moment because \(v = L\).
Then the automaton in hole \(l\) emits a \(\delta\mpar{t}\), this is synchonised with the action that sets \(v \gets R\) and the automata in hole \(r\) must emit a \(\act{acc}\mpar{t}\).
Finally the automaton in the hole \(r\) emits any sequence of actions.

Another run can be: The automaton in the hole \(l\) emits a sequence of non \(\delta\mpar{t}\) actions so the automaton in the hole \(r\) cannot emit a \(\act{acc}\mpar{t}\) and never runs.
\end{exi}
From this point open automata and open transitions are sometimes called automata and transitions for simplicity. % NEED NORMALISATION, LEFT IN CASE OF OMISSION


\section{Composition of Open Automata}\label{sec:comp}
Open automata are partially specified automata, that partiality comes mostly from the holes.
The interpretation of a hole is an interface with another open automaton, in which we can plug an open automaton. The plugging operation is called composition.
In this report, composition will only refer to filling holes, other compositions like ``parallel composition" (pure interleaving) are not tackled (but one can define an open automaton that performs the parallel composition of two others).
The composition of open automata was already implicitly defined by the means of composition on pNets in previous work \cite{henrio:01299562} but never completely formalised on open automata.
The definition of composition below is a direct translation of what happens with pNets composition without the need of introducing pNets.
\begin{defi}[Composition of open automata]
The composition of \(A_c \defobject \OAg[c]\) in the hole \(k \in J_p\) of \(A_p \defobject \OAg[p]\) is an open automaton defined as follows.
\begin{align*}
	A_p\subst{A_c}{k} \defnotation & \OA{S_p \times S_c}{\mpar{s_{0p}, s_{0c}}}{V_p \uplus V_c}{\sigma_{0p} \uplus \sigma_{0c}}{J_c \uplus J_p \setminus \mbrc{k}}{T} \\
	\text{with } T \defobject & \mset{\OT{\mpar{s_p, s_c}}{\mpar{s'_p, s'_c}}{\alpha_p}{\beta_j^{j \in J'_c \uplus J'_p \setminus \mbrc{k}}}{g_p \wedge g_c \wedge \alpha_c = \beta_k}{\psi_p \uplus \psi_c}}{\OTx{p}{}{}{p} \in T_p, \OTx{c}{}{}{c} \in T_c} \\
	& \cup \mset{\OT{\mpar{s_p, s_c}}{\mpar{s'_p, s_c}}{\alpha_p}{\beta_j^{j \in J'_p}}{g_p}{\psi_p}}{\OTx{p}{}{}{p} \in T_p, k \notin J'_p, s_c \in S_c}
\end{align*}
\end{defi}
The action emitted when \(A_c\) makes a transition is sychronised with the action of the hole \(k\) in transitions of \(A_p\) which have it as a hole action (first transition set, \(\alpha_c = \beta_k\)).
The composition may look like a handshake, with both automata running in parallel (product of states and joint variables) however it is asymmetric because it is the first automaton that decides when the second can evolve, while the first automaton can evolve independently as soon as the hole is not involved in the open transition considered.

Composition is a complex process that generates big automata with complex and potentially simplifiable transitions.
However simplification is a hard problem that is not yet automatised on open automata, so from this point the guards and transitions will always be simplified by hand without further explanation.
An example of composition can be found in Appendix \ref{apx:composition}.
\begin{prop}[Simultaneous composition]
Let \(A_1\), \(A_2\), \(A_3\) be three open automata with respectively \(J_1\), \(J_2\) and \(J_3\) as hole names.
We have
\begin{align*}
	A_1\subst{A_2}{j}\subst{A_3}{j'} & = A_1\subst{A_3}{j'}\subst{A_2}{j} & j, j' & \in J_1 \\
	\mpar{A_1\subst{A_2}{j_1}}\subst{A_3}{j_2} & = A_1\subst{A_2\subst{A_3}{j_2}}{j_1} & \mpar{j_1, j_2} & \in J_1 \times J_2
\end{align*}
Where the equality is modulo isomorphism (state renaming, associativity, commutativity).
\end{prop}
This property is induced by the composition on open pNets.
Intuitively it holds because the cartesian product of states, the union of sets, the union of maps and the conjunction on formulas (in the guards) are commutative and associative.
This allows us to define the simultaneous composition \(A\mdbrk{A_j^{j \in J'}}\) as the result of any order of (sequential) composition where each \(A_j\) is composed in the hole \(j\).
Even if an automaton is composed in two distinct holes it does not share its variables, each composition makes the cartesian product of states and a disjoint union of the variables.
In practice, variables may have to be renamed.


\section{Properties of a refinement relation for Open Automata}\label{sec:proofelts}
We want to define a refinement relation on open automata, this refinement relation is expected to have some properties to be of any use.
In this section we list and adapt to open automata the properties expected from a refinement relation.

There are several kind of refinement relation depending on the properties we want.
For example if we are interested in producing the same sequences of actions as another automaton we may want to use trace set inclusion as a refinement.
The main expected properties here are the standard properties on relations and the ones related to composition.
Because of composition, a relation as strong as simulation must be used.

Informally, a simulation relating two automata means that every run of one automaton can be simulated by the other.
Simulations are defined by relating states of automata.
Two classical automata are called in simulation if there is a simulation relating their states.

Note that in the previous paragraph there are two relations:
The first is between states of two specific automata (\(R \subseteq S_1 \times S_2\)), the second is between automata (\(A_1 \leq A_2\)).
Generally the second relation is implicit because its definition is always \(\exists R \subseteq S_1 \times S_2, P\mpar{R}\) with \(P\) a property on the relation on states, so only talking about \(R\) by defining \(P\) is sufficient.\Ludo{}{je comprends pas cette phrase}\Quentin{}{Est-ce mieux?}
However later on in this report we will define more complex relations on automata so we clarify the distinction now.

A relation of second kind will be called ``relation on automata" whereas a relation of the first kind is called ``relation on configurations" as defined below.
In the work on FH-bisimulation \cite{henrio:01055091} such a relation \(R\) was defined as triples in \(S_1 \times S_2 \times \formulas\) where for \(\mpar{s_1, s_2} \in S_1 \times S_2\) there is a unique \(f \in \formulas\) such that \(\mpar{s_1, s_2, f} \in R\).
This is classically the definition of a function from \(S_1 \times S_2\) to \(\formulas\) so we will define it as such.
\begin{defi}[Relation on configurations]
A relation on configurations of \(\OAg[1]\) and \(\OAg[2]\) is a function \(R: S_1 \times S_2 \to \rformulas[V_1 \uplus V_2]\).

Two states \(s_1 \in S_1, s_2 \in S_2\) with their respective valuations \(\sigma_1: V_1 \to \values, \sigma_2: V_2 \to \values\) are related iff \(\sigma_1 \uplus \sigma_2 \vdash R\mpar{s_1, s_2}\).
\end{defi}
In this section we give properties on refinement relations, both on the relations on configurations and on the relation on automata.
When relating two automata \(A_1 \mathrel{\mathcal{R}} A_2\) or their configurations \(R: S_1 \times S_2 \to \rformulas[V_1 \uplus V_2]\), \(A_1\) will be called the implementation and \(A_2\) the specification.

For the relation on automata, the important properties we consider are:
\begin{defi} A relation on open automata \(\leq\) is
\defitem \textbf{reflexive} iff \(\forall a, a \leq a\);
\defitem \textbf{transitive} iff \(\forall a\, b\, c, a \leq b \wedge b \leq c \implies a \leq c\);
\defitem \textbf{a preorder} iff it is reflexive and transitive;
\defitem \textbf{correct w.r.t.\@ composition} iff \(\forall a\, b, a\subst{b}{j} \leq a\);
\defitem \textbf{complete w.r.t.\@ composition} iff \(\forall a\, b, a \leq b \implies \exists c, a \cong b\subst{c}{j}\);
\defitem \textbf{context refining for composition} iff \(\forall a\, b\, c, a \leq b \implies a\subst{c}{j} \leq b\subst{c}{j}\);
\defitem \textbf{congruent for composition} iff \(\forall a\, b\, c, a \leq b \implies c\subst{a}{j} \leq c\subst{b}{j}\);
\defitem \textbf{compatible with composition} iff \(\forall a\, b\, c\, d, a \leq b \wedge c \leq d \implies c\subst{a}{j} \leq d\subst{b}{j}\);
\defitem \textbf{compatible with FH-bisimulation} iff \(\forall a\, b\, c, d, a \cong b \wedge c \cong d \wedge a \leq c \implies b \leq d\).
\end{defi}
Reflexivity, transitivity and preorder are classical properties on relations.
Correctness w.r.t.\@ composition means that every composition is considered a refinement.
This property captures the expected behaviour of a refinement relation on a compositional structure like the open automata.
Completeness w.r.t.\@ composition means that a refinement corresponds to the left automaton being equivalent to some composition of the right automaton.
This property is useful if we want to characterise the kind of operations that can generate any refined automaton starting from the specification.
We won't have it because extending the domain of definition of the initial valuation will be a refinement which can't be expressed with composition.
Yet a ``completeness w.r.t.\@ some set of operations" is still a strong and interesting property.
Context refinement is the refinement being compatible with composition of the same automaton.
Congruence is the refinement being compatible with being composed in the same automaton.
Compatibility with composition is the conjunction of the two latter (assuming the relation is a preorder).
This property is required on compositional structures to be of any use in verification, proofs and for model checking.
Finally compatibility with FH-bisimulation is the refinement not being able to distinguish two FH-bisimilar automata.
Since FH-bisimulation is the pre-existing equivalence relation, we should not introduce an incompatible proof strategy.

\Ludo{}{I do not understand the english + too many important properties you should give a hierarchy or at least find a better transition}\Quentin{}{Important properties not on the same object, this is now underlined by moving the object at the start of the sentences}
For the relation on states, the important properties we consider are being a pre-simulation and prevent the introduction of deadlocks as defined in the following.
\begin{defi}[Pre-simulation]
Let \(A_1 \defobject \OAg[1]\) and \(A_2 \defobject \OAg[2]\).
A relation on configurations \(R: S_1 \times S_2 \to \rformulas[V_1 \uplus V_2]\) is a pre-simulation if it satisfies both
\defitem Initial configurations are related: \(\sigma_{01} \uplus \sigma_{02} \vdash R\mpar{s_{01}, s_{02}}\);
\defitem From related states, all out-transitions from \(A_1\) can be simulated in \(A_2\) and their target states are related:
\begin{multline*}
	\forall \mpar{s_1, s_2} \in S_1 \times S_2, \bigsymb{\forall} t_1 \defobject \OTx{1}{}{1}{1} \in \fOT{s_1}, \forall \sigma: V_1 \uplus V_2 \uplus \fvars{t_1} \to \values, \\
	\mpar{\sigma \vdash R\mpar{s_1, s_2} \wedge g_1} \implies \bigsymb{\exists} t_2 \defobject \OTx{2}{}{2}{2} \in \fOT{s_2}, \exists \nu: \fvars{t_2} \to \values, \\
	\sigma \uplus \nu \vdash \alpha_1 = \alpha_2 \wedge \bigwedge_\subbox{j \in J'_1 \cap J'_2} \beta_{1j} = \beta_{2j} \wedge g_2 \wedge R\mpar{s'_1, s'_2}\psubst{\psi_1 \uplus \psi_2}
\end{multline*}
\end{defi}
This definition translated to open automata the classical definition of simulation \cite{10.1007/3-540-54430-5_78}.
The last formula means that for every pair of related configurations of the automaton and every possible transitions (\(\vdash g_1\)) from the first automaton, there is a possible transition (\(\vdash g_2\)) such that the produced action matches (\(\alpha_1 = \alpha_2\)), the same holes' action (\(J'_1 \cap J'_2\)) matches (\(\beta_{1j} = \beta_{2j}\)) and the target states are related after variable update. % OR variableS
Apart from holes, this is a natural extension of the notion of simulation on LTS, which is the same definition without variables nor holes nor guards.
In FH-bisimulation, holes with the same name will receive the same open automaton when composed, and the resulting open automaton will still be equivalent.
Here we assume that one of the related open automata can be obtained by composing an open automaton in a hole, open automaton which can have holes itself.
This implies that there is no relation between the related automata's holes, for instance composing \(A_i\) in \(j_i\) (\(1 \leq i \leq n\)) of \(A_0\) gives the set of holes \(J \defobject \mpar{\biguplus_{0 \leq i \leq n} J_i} \setminus \mset{j_i}{1 \leq i \leq n}\) that is neither subset nor superset of \(J_0\).
Yet we still assume that in holes with the same name, the same open automaton will be composed.
Hence only the holes in common have to match their actions.
However it is not sufficient to imply properties on the relation on automatons like being a preorder.
It is only a necessary condition that we put on any relation that we call a ``simulation'' in the following.

In addition to the above properties, we expect from a refinement relation to prevent the introduction of deadlocks.
Deadlocks are reachable configurations from which no further transition is possible.
They can arise unpredictably when parallelism is involved (which is the case in our composition) and they generally break the intended behaviour of a system.
By preventing them we \Ludo{could}{This seems to say that it is useless to prevent them} \Quentin{also}{Is it better with this additionnal word?} hope that the refinement preserves some liveness properties.
\begin{defi}[Deadlock prevention, intuitive definition]
A relation on configurations \(R: S_1 \times S_2 \to \rformulas[V_1 \uplus V_2]\) is deadlock reducing if:
\defitem It is a pre-simulation;
\defitem If from related states, there is a possible transition in the specification, then there is a pair of possible matching transitions from both automata:
\begin{multline*}
	\forall \mpar{s_1, s_2} \in S_1 \times S_2, \forall \sigma: V_1 \uplus V_2 \to \values, \mpar{\sigma \vdash R\mpar{s_1, s_2}} \implies \\
	\mpar{\exists t_2 \in \fOT{s_2}, \exists \nu: \fvars{t_2} \to \values, \sigma \uplus \nu \vdash \fguard{t_2}} \implies \\
	\bigsymb{\exists} \mpar{t_1, t_2} \defobject \mpar{\OTx{1}{}{1}{1}, \OTx{2}{}{2}{2}} \in \fOT{s_1} \times \fOT{s_2}, \exists \nu: \fvars{t_1} \uplus \fvars{t_2} \to \values, \\
	\sigma \uplus \nu \vdash g_1 \wedge g_2 \wedge \alpha_1 = \alpha_2 \wedge \bigwedge_\subbox{j \in J'_1 \cap J'_2} \beta_{1j} = \beta_{2j} \wedge R\mpar{s'_1, s'_2}\psubst{\psi_1 \uplus \psi_2}
\end{multline*}
\end{defi}
Actually, this requirement prevents the presence of deadlocks in the first automaton if there is a related non-deadlock configuration in the second by \Ludo{contraposition}{The contraposition is too tricky for me: are you sure?}\Quentin{}{I am, I want to say that this property is in contraposition form because the formula is not exactly what I pretend it is right there.}.
Two reasons motivated the formulation in contraposed form:
\begin{itemize}
\item We require a pair of matching transitions from both automata because we do not want a transition in the implementation making a configuration not a deadlock but this transition cannot be taken from the related configuration of the specification.
\item The non contraposed formulation of the property would have required the unintuitive quantification on a possibly empty set of transition to state that they are not satisfied.
\end{itemize}
Unfortunately, in the general case, this property is conflicting with the other ones because every property involving composition would be applicable on composition introducing deadlocks.
A way to solve this conflict is to only consider a composition that do not introduce deadlocks.
With this new condition on validity of composition we are able to define in the next sections a refinement relation where most of the properties of interest are satisfied.

Before characterising a composition which does not introduce deadlocks we need to be able to tell which states are reachable because some deadlocks are not reachable and should not be taken into account as being introduced.
\begin{defi}[Reachability]
For any open automata \(A \defobject \OAg\), a reachability predicate \(\reach{A}: S \to \rformulas[V]\) is a predicate on states satisfying both
\defitem Inital state is reachable: \(\sigma_0 \vdash \reach{A}\mpar{s_0}\)
\defitem Reachability is preserved across transitions: \nmm{\forall t \defobject \OTg \in T, \fvars{t} \vdash \reach{A}\mpar{s} \wedge g \implies \reach{A}\mpar{s'}\psubst{\psi}}
\end{defi}
The reachability predicate is used to characterise the reachable configurations in a run of an automaton.
In fact the role of the predicate is to characterise potentially reachable configurations without having to characterise configurations as the result of a valid path in an automaton.

To do that we impose that the initial configuration is reachable and that reachability is preserved by taking valid transitions.
This effectively makes reachability take into account all paths, and potentially over-approximate the reachable configurations.
The exact reachability may not be representable in the formulas, hence the need of potentially over-approximating.

% Another way of understanding this predicate is that it is a fixpoint of the union of valuations (function \(f\) defined below) that contains the initial valuation:
% \begin{align*}
	% f\mpar{p} = \mbrc{\mbox{\nmm{s' \mapsto p\mpar{s'} \vee \bigsymb{\bigvee_\subbox{\OTg \,\in\, \fIT{s'}}} p\mpar{s}\psubst{\psi} \wedge g}}} && % TODO: fix that in some way
	% \sigma_0 \vdash \reach{A}\mpar{s_0} &&
	% f\mpar{\reach{A}} = \reach{A}
% \end{align*}
\begin{defi}[Non-locking composition, intuitive definition]
Let \(A_i \defobject \OAg[i], 0 \leq i \leq n\).
The composition \(A \defobject A_0\mdbrk{j_i \mapsto A_i \middle| 1 \leq i \leq n}\) is non-locking if \(A\) has a reachability predicate satisfying:
From reachable configurations, if there was a possible transition in \(A_0\) then there is a possible transition in \(A\).

\Ludo{a discuter}{les notations de cette formule sont vraiment pas terribles}

Formally for \(\OAg \defobject A\), it gives
\begin{multline*}
	\forall s \defobject \mpar{s_0, s'} \in S_0 \times \prod_\subbox{1 \leq i \leq n} S_i, \forall \sigma: V \to \values, \mpar{\sigma \vdash \reach{A}\mpar{s}} \implies \\
	\mpar{\exists t_0 \in \fOT{s_0}, \exists \nu_0: \fvars{t_0} \to \values, \sigma \uplus \nu_0 \vdash \fguard{t_0}} \implies \\
	\exists t \in \fOT{s}, \exists \nu: \fvars{t} \to \values, \sigma \uplus \nu \vdash \fguard{t}
\end{multline*}
\end{defi}
We can expand the alias \(V\) into \nmm{\biguplus_\subbox{1 \leq i \leq n} V_i \uplus V_0} and \Ludo{transform}{I do not understand this ``transformation''} the last transition \(t\) into a pair \(\mpar{t, t_0}\) where the second transition is the one from \(A_0\) which generated \(t\).
Doing that makes the definition become mildly similar with the deadlock prevention one. % OR slightly
It is not a coincidence because their goal is to caracterise the same kind of compatibility between automata.
Actually it is possible to simplify both definition and make them even more similar.
\begin{defi}[Deadlock prevention, working definition]
A relation on configurations \(R: S_1 \times S_2 \to \rformulas[V_1 \uplus V_2]\) is deadlock reducing if it is a pre-simulation and it satisfies the following:
\[ \forall \mpar{s_1, s_2} \in S_1 \times S_2, V_1 \uplus V_2 \uplus \biguplus_\subbox{t_2 \in \fOT{s_2}} \fvars{t_2} \vdash R\mpar{s_1, s_2} \wedge \bigvee_\subbox{t_2 \in \fOT{s_2}} \fguard{t_2} \implies \bigvee_\subbox{t_1 \in \fOT{s_1}} \fguard{t_1} \]
\end{defi}
The requirement here is that from any related configurations, if there is a possible transition in the second automaton (whatever the value of the free variables) then there is a possible transition in the first automaton (the existential quantifier is implicit in the notation \(\vdash\)).
The transition are not matched because the relation on configurations is a pre-simulation and it is sufficient to ensure it as will be shown in the proof.
Apart from the matching transitions this is exactly the same explaination as the intuitive definition.
The expected advantage of this formulation is that a SMT should behave better without the quantifiers.
Indeed the big disjunctions are not quantifiers because in practise the out-transitions are finitary, and so is the expansion of the disjunction. % Check if that construction is correct
\begin{lem}
The \Ludo{intuitive and working definition}{add the numbers} of deadlock prevention are equivalent.
\end{lem}
\begin{defi}[Non-locking composition, working definition]
\(A \defobject A_0\mdbrk{j_i \mapsto A_i \middle| 1 \leq i \leq n}\) is a non-locking composition if:
\[ \forall s \defobject \mpar{s_0, s'} \in S, V \uplus \biguplus_\subbox{t_0 \in \fOT{s_0}} \fvars{t_0} \vdash \reach{A}\mpar{s} \wedge \bigvee_\subbox{t_0 \in \fOT{s_0}} \fguard{t_0} \implies \bigvee_\subbox{t \in \fOT{s}} \fguard{t} \]
\end{defi}
\begin{lem}
The intuitive and working definition of non-locking composition are equivalent.
\end{lem}
An illustration of both the introduction of a deadlock by composition and the definition of non-locking composition is given in Appendix \ref{apx:lockcomp}.
Proof of both lemma are given in Appendix \ref{apx:lemeqd}.

This new definition of composition can replace the standard composition.
For the simultaneous composition, we may think that if the last composition of a specific composition order is non-locking, then every composition is non-locking for all composition orders.
\Ludo{confused}{ce paragraphe est tres tres dur a lire}
Unfortunately it is not true.
Actually even if the last composition of any order of composition is non-locking there is no guarantee that this order is non-locking at each intermediate step.
However the opposite way is true even if uninformative:
If there is an order of composition where every composition is non-locking then the last composition is non-locking, last composition resulting in the same automaton as the simultaneous composition.
So the condition is better checked after the simultaneous composition or the last individual composition.
\begin{exi}
\begin{figure}
\centering
\begin{tikzpicture}

\node[state,initial] (v0) at (0,0) {i};
\node[state] (vd1) at (0,2.5) {d1};
\node[state] (vd2) at (0,-2.5) {d2};
\node[state] (vop) at (6,0) {op};

\draw (v0) edge node[near end] {\OTd{\act{local}}{}{}} (vop);
\draw (v0) edge[bend left,looseness=0] node {\OTd{\act{fetch}}{net \mapsto \act{fetched}}{}} (vd1);
\draw (vd1) edge[bend left,looseness=0] node {\OTd{\act{err}}{hash \mapsto \act{invalid}}{}} (v0);
\draw (v0) edge[bend left,looseness=0] node {\OTd{\act{valid}}{hash \mapsto \act{valid}}{}} (vd2);
\draw (vd2) edge[bend left,looseness=0] node {\OTd{\act{err}}{net \mapsto \act{timeout}}{}} (v0);
\draw (vd1) edge[bend left] node[near end] {\OTd{\act{valid}}{hash \mapsto \act{valid}}{}} (vop);
\draw (vd2) edge[bend right] node[near end,'] {\OTd{\act{push}}{net \mapsto \act{pushed}}{}} (vop);

\end{tikzpicture}

\caption{A made-up protocol for file synchronisation}
\label{fig:pnls}
\end{figure}
The automaton on Figure \ref{fig:pnls} is a sample protocol for file synchronisation.
The two holes are \(net\) a process to manage network communications, and \(hash\) a process to check files integrity.
The state \(i\) is the initial state.
In that state we want to synchronise the files before working, or work on the local version as a fallback.
If the preferred version is the remote one, we go in state \(d1\) asking the netwotk manager to fetch the lastest version, then ask the integrity checker to validate the files.
If the preferred version is the local one, we go in state \(d2\) asking for the integrity checker to validate the local files, then ask the network manager to push them.
In fallback we use the local version.

We can plug an inactive process for the network manager and the file integrity checker by composing a deadlock automaton in their holes, but always both at the same time.
The reason for that is the simultaneous composition is non-locking, but any individual composition is locking because the state \(d1\) or \(d2\) (depending on the order) becomes a deadlock.
\end{exi}
From this point \Ludo{and in the previous definitions}{confusing: This means that when we check the properties for  a refinement relation, for example correctness wrt composition,we only consider non-locking compositions. OU UNE AUTRE SOLUTION, MAIS COMME C EST ECRIT C EST PERURBANT}, composition will only refer to non-locking composition.

A relation on open automata and its corresponding relation on configurations are simulations if the latter is a deadlock reducing pre-simulations and the former is a preorder.
With all these properties we can now look at some concrete refinement relations.


\section{Refinement relations in restricted cases}\label{sec:prelref}
The main goal of this section is to explain the different aspects of refinement in restricted cases so that they are not all introduced at once in the real refinement relation.
In particular if we have a family of open automata \(A_i \defobject \OAg[i]\), \(0 \leq i \leq n\), let \(A \defobject A_0\mdbrk{j_i \mapsto A_i \middle| 1 \leq i \leq n}\).
Recall that there is no relation between the holes of \(A\) and \(A_0\) because \(\mpar{\biguplus_{0 \leq i \leq n} J_i} \setminus \mset{j_i}{1 \leq i \leq n}\) is neither subset nor superset of \(J_0\).
So looking at restricted cases where holes are related in a specific manner can help understanding the general case.
Hopefully the intermediate refinement relation introduced can be used as simpler (and less expensive to compute) versions of the real refinement relation in the case the manipulated automata respect some constraints.

We restrict the holes of the related automata so that we can explain the different refinements without their interactions.
In the following definition, \(\triangle \in \mbrc{\subseteq, =, \supseteq}\).
\begin{defi}[Hole-\(\triangle\) simulation]
For two open automata \(A_1 \defobject \OAg[1]\) and \(A_2 \defobject \OAg[2]\), the relation on configurations \(R: S_1 \times S_2 \to \rformulas[V_1 \uplus V_2]\) is a hole-\(\triangle\) simulation of \(A_1\) by \(A_2\) if:
\item[1)] \(J_1 \triangle J_2\)
\item[2)] \(\sigma_{01} \uplus \sigma_{02} \vdash R\mpar{s_{01}, s_{02}}\)
\item[3)] \(\forall \mpar{s_1, s_2} \in S_1 \times S_2,\)\vspace{-8pt}
\noindent\begin{multline*}
	\mpar{\everymath{\displaystyle}\begin{array}{l}
		\bigsymb{\forall} t_1 \defobject \OTx{1}{}{1}{1} \in \fOT{s_1}, \bigsymb{\exists} \mpar{t_{2x} \defobject \OTx{2}{x}{2x}{2x} \in \fOT{s_2}}^{x \in X}, \\
		\quad \mpar{\forall x \in X, J'_{2x} \cap J_1 = J'_1 \cap J_2} \\[-10pt]
		\nwedge V_1 \uplus V_2 \uplus \fvars{t_1} \vdash R\mpar{s_1, s_2} \wedge g_1 \implies \operatorname*{\bigsymb{\bigvee}}_{x \in X} \mpar{\begin{array}{l}
			\alpha_1 = \alpha_{2x} \wedge \bigwedge_\subbox{j \in J'_{2x} \cap J_1} \beta_{1j} = \beta_{2xj} \\[12pt]
			\nwedge g_{2x} \wedge R\mpar{s'_1, s'_{2x}}\psubst{\psi_1 \uplus \psi_{2x}}
		\end{array}}
	\end{array}} \\
	\wedge \mpar{V_1 \uplus V_2 \uplus \biguplus_\subbox{t_2 \in \fOT{s_2}} \fvars{t_2} \vdash R\mpar{s_1, s_2} \wedge \bigvee_\subbox{t_2 \in \fOT{s_2}} \fguard{t_2} \implies \bigvee_\subbox{t_1 \in \fOT{s_1}} \fguard{t_1}}
\end{multline*}
\end{defi}
The third item of the definition has two parts:
The first part is the main requirement, a (\Ludo{not proven equivalent}{risky, perhaps choose one and state that you suppose that the other is equivalent, or discuss a bit more this point + why not the same? symbolic reasoning is hidden here}) reordering of the pre-simulation requirement so that it better fits SMT.
Instead of matching an effective transition to another effective transition as in the pre-simulation definition, we match an open transition to a family of covering open-transitions.
The second part is directly the deadlock prevention requirement.

The hole-equal simulation is applicable when the related automata have the same holes.
It is designed to capture all the refinements other than composition: Duplicating and merging states and transitions, changing variables, removing transitions, adding requirements to guards, specifying more initial variables, and many more.

The hole-subset simulation is applicable when the simulated automaton has less holes.
It is designed to have composition correctness when the composed automata have no holes.

The hole-superset simulation is applicable when the simulated automaton has more holes.
It is designed to capture the addition of holes involved in transitions, which will indirectly add requirements to transitions when the new holes will be composed.
It is also an attempt at giving a meaning to the new holes when composing with any open automata.
However it does not capture the removal of holes because other issues arise when trying to both add and remove holes.
\begin{exi}[Example of hole-equal simulation]
\begin{figure}
\centering
\begin{tikzpicture}

\node[state,initial] (n) {\textbar\textbar};

\draw (n) edge[dir=90] node {\OTd{x}{l \mapsto x}{}} (n);
\draw (n) edge[dir=270] node {\OTd{x}{r \mapsto x}{}} (n);

\end{tikzpicture}
\vrule
\begin{tikzpicture}

\node[state,initial] (v0) at (0,1.5) {l};
\node[state] (v1) at (0,-1.5) {r};

\draw (v0) ++(-1.75,0) edge[blue,init={n \gets 1}] (v0);
\draw (v0) edge[dir=0] node {\OTd{x}{l \mapsto x}{}} (v0);
\draw (v1) edge[bend left,looseness=0] node {\OTd{x}{l \mapsto x}[n > 0]{n \gets n - 1}} (v0);
\draw (v1) edge[dir=0] node {\OTd{x}{r \mapsto x}{}} (v1);
\draw (v0) edge[bend left,looseness=0] node {\OTd{x}{r \mapsto x}[n > 0]{n \gets n - 1}} (v1);

\end{tikzpicture}
\caption{On the left: Parallel composition operator; On the right: n control switches automaton}
\label{fig:hisim}
\end{figure}
Figure \ref{fig:hisim} introduces two related open automata with same holes \(\mbrc{l, r}\).
The automaton on the left is a parallel composition: any hole can perform actions at any time, the actions are passed unmodified and no synchronisation is performed.
The automaton on the right is a sequential composition with \(n\) control switches: a hole performs a sequence of actions then the other does, there are \(n\) seqences of a single automaton doing actions without the other performing some.

Intuitively the right automaton is a refinement of the left automaton because the left automaton is what happens when there was no bound \(n\) in the right automaton.
\[ R \defobject \mbrc{\mpar{l, ||} \mapsto \top, \mpar{r, ||} \mapsto \top} \]
\(R\) is a hole-identical simulation of \(n\) control switches by parallel composition.
There are enough examples of proofs of refinement in this report for the reader to be able to prove it as an exercise.
\end{exi}
\begin{exi}[Example of hole-subset simulation]
Example of Appendix \ref{apx:composition} can be reused here.
The automaton on Figure \ref{fig:tlf} is a refinement of the one on Figure \ref{fig:tls} by limited composition correctness (Proposition \ref{prop:cc'}).

It is a simplified version of the motivation for making a refinement relation, which is the traffic light system in an article about weak-bisimulation \cite{wang:03126313}.
It had to be modified because here the simulation is a strong-simulation (no silent actions).
\end{exi}
\begin{exi}[Example of hole-superset simulation]
\begin{figure}
\centering
\begin{tikzpicture}
\node[state,initial] (v0) at (0,0) {I};
\node[state] (v1) at (-6,0) {T};
\node[state] (v2) at (6,0) {C};
\draw (v0) ++(0,-1.5) edge[near start,init={m \gets 0}] (v0);
\draw (v0) edge[dir=90] node {\OTd{\act{money}\mpar{x}}{}{m \gets m + x}} (v0);
\draw (v0) edge[bend left,looseness=0] node {\OTd{\act{tea}\mpar{}}{}[m \geq 150]{m \gets m - 150}} (v1);
\draw (v1) edge[bend left,looseness=0] node {\OTd{\act{serve}\mpar{\mathit{tea}, x}}{}{}} (v0);
\draw (v0) edge[bend left,looseness=0] node {\OTd{\act{coffee}\mpar{}}{}[m \geq 150]{m \gets m - 150}} (v2);
\draw (v2) edge[bend left,looseness=0] node {\OTd{\act{serve}\mpar{\mathit{coffee}, x}}{}{}} (v0);
\end{tikzpicture}
\hrule
\begin{tikzpicture}
\node[state,initial] (v0) at (0,0) {I};
\node[state] (v1) at (-6,0) {T};
\node[state] (v2) at (6,0) {C};
\draw (v0) ++(0,-1.5) edge[near start,init={m \gets 0}] (v0);
\draw (v0) edge[dir=90] node {\OTd{\act{money}\mpar{x}}{mon \mapsto \act{credit}\mpar{x}}{m \gets m + x}} (v0);
\draw (v0) edge[bend left,looseness=0] node {\OTd{\act{tea}\mpar{}}{}[m \geq 150]{m \gets m - 150}} (v1);
\draw (v1) edge[bend left,looseness=0] node {\OTd{\act{serve}\mpar{\mathit{tea}, x}}{serv \mapsto \act{tea}\mpar{x}}{}} (v0);
\draw (v0) edge[bend left,looseness=0] node {\OTd{\act{coffee}\mpar{}}{}[m \geq 150]{m \gets m - 150}} (v2);
\draw (v2) edge[bend left,looseness=0] node {\OTd{\act{serve}\mpar{\mathit{coffee}, x}}{serv \mapsto \act{coffee}\mpar{x}}{}} (v0);
\end{tikzpicture}
\caption{On the top: Vending machine; On the bottom: Open vending machine}
\label{fig:ovm}
\end{figure}
Figure \ref{fig:ovm} introduces two related open automata with same states and transitions except for hole actions.
Both automata are vending machine: Money is introduced while in state \(I\), accumulating in the money variable \(m\), when there are enough money a coffe or a tea can be ordered by taking the transitions to the states \(C\) or \(T\) and money is decreased, then the drink is served by taking the transition back to state \(I\).
The holes added in the open version on the bottom are \(mon\) a hole which task is to verify the money inserted (coin vary depending on the country for instance) and \(serv\) which task is to serve the drink and report the exact amount served.

The automaton on the bottom is a refinement of the automaton on the top using hole-superset simulation.
The relation on states is the same as the one for reflexivity.
Intuitively the automaton on the bottom binds some variables to action produced by the holes, the automaton is no more free to take a transition some hole is not able to perform.
For instace if the automaton in the hole \(mon\) can only perform \(\act{credit}\mpar{100}\) then after the composition, the automaton on the bottom should be a refinement of the one on the top.
Actually after such a composition it is still in hole-superset simulation with the top automaton by Proposition \ref{prop:cc'}.
\end{exi}
\begin{thm} A hole-\(\triangle\) simulation is a pre-simulation. \end{thm}
Proof is given in Appendix \ref{apx:presim}.
\begin{thm} The hole-\(\triangle\) simulation on automata is a preorder. \end{thm}
The idea for reflexivity is that the related automata are always in the same configuration.
\[ R \defobject \mpar{s, s'} \mapsto \choice{\bigwedge_{v \in V} v_1 = v_2 & \text{if } s = s'\\ \bot & \text{otherwise}} \]
Where \(v_x\) is the renaming to distinguish conflicting automaton variables.
\(R\) is a witness of the reflexivity, the proof is mainly the same as several of the following ones so it will be omitted.

Let \(A_1 \defobject \OAg[1]\), \(A_2 \defobject \OAg[2]\) and \(A_3 \defobject \OAg[3]\).
Let \(R_{12}\) be a hole-\(\triangle\) simulation of \(A_1\) by \(A_2\) and \(R_{23}\) of \(A_2\) by \(A_3\), we want to define \(R\) by saying that there exists a state \(s_2 \in S_2\) and values for \(V_2\) such that \(R_{12}\mpar{s_1, s_2} \wedge R_{23}\mpar{s_2, s_3}\).
However we cannot talk about states in formulas.
The solution is to expand the existential quantifier on states as we did in deadlock prevention, which is possible because states are finite.
\[ R \defobject \mpar{s_1, s_3} \mapsto \exists V_2, \bigvee_\subbox{s_2 \in S_2} R_{12}\mpar{s_1, s_2} \wedge R_{23}\mpar{s_2, s_3} \]
\(R\) is a witness of the transitivity, proof is given in Appendix \ref{apx:trans}.
\begin{prop}[Limited composition correctness]\label{prop:cc'} Hole-subset simulation is correct w.r.t.\@ composition with automata without holes. \end{prop}
For two open automata \(A_1 \defobject \OAg[1]\) and \(A_2 \defobject \OAg[2]\) with \(J_2 = \emptyset\), let \(k \in J_1\) and \(A \defobject A_1\subst{A_2}{k}\).
Let \(\reach{A}\) be a witness that \(A\) is non-locking.
The idea for the relation on configurations is that the copy of \(A_1\) in \(A\) do the same things as \(A_1\), as if we were trying to prove reflexivity.
\[ R \defobject \mpar{\mpar{s_{1'}, s_2}, s_1} \mapsto \choice{\reach{A}\mpar{s_{1'}, s_2} \wedge \bigwedge_\subbox{v_1 \in V_1} v_{1'} = v_1 & \text{if } s_{1'} = s_1 \\ \bot & \text{otherwise}} \]
Where \(y_{1'}\) is the renaming of \(y_1\) from \(A_1\) in \(A\).
\(R\) is a witness of the limited composition correctness, proof is given in Appendix \ref{apx:lchs}.
\begin{prop}[Special composition correctness]
For three open automata \(A_1 \defobject \OAg[1]\), \(A_2 \defobject \OAg[2]\) and \(A_3\), let \(k \in J_1 \setminus J_2\) and \(R_{12}\) be a hole-superset simulation of \(A_1\) by \(A_2\).
There is a hole-superset simulation of \(A_1\subst{A_3}{k}\) by \(A_2\).
\end{prop}
Let \(\reach{A_1\subst{A_3}{k}}\) be a witness that \(A_1\subst{A_3}{k}\) is non-locking.
The idea for the relation on configurations is to combine the non-locking composition and the already existing realtion on configurations.
\[ R \defobject \mpar{\mpar{s_1, s_3}, s_2} \mapsto R_{12}\mpar{s_1, s_2} \wedge \reach{A_1\subst{A_3}{k}}\mpar{s_1, s_3} \]
\(R\) is a witness of the secial composition correctness, proof is given in Appendix \ref{apx:schs}.

For composition correctness, the hole-subset simulation can only tell if there is a refinement when the composed automata have no holes.
To lift this limitation we need to allow adding holes which doesn't have a direct interpretation.
At first a solution using a function mapping the action performed by the new holes in the composition to the ones performed by removed hole in the base automaton was explored.
It was abandonned because the base automaton has nothing intelligent to tell about the action required by the composed automata.
That is the reason the hole-superset simulation doesn't constrain the action performed by the added holes.

Only one of the hole-\(\triangle\) simulations cannot tell that a non-locking composition with any automaton is a refinement because we need at the same time to add and remove holes, hence the need of the special composition correctness for hole-superset.
Also an issue arises when trying to tell that a composition is a refinement of the base automaton: the refinement relation cannot compare the holes sets.
This conflicts with the transitivity propriety when a hole present in both extrema of the chain is absent in an intermediate automaton.
In that case all the constraint on holes actions are lifted even if we would like to require an equality of these actions on the transitive cloture.


\section{Refinement relation for open automata}\label{sec:refinement}
One of the solutions to keep transitivity and constraints on holes actions from same holes is to keep track of which holes are in common at each step of refinement.
The transitive refinement will have to either rename the holes without constraints or explicitly say which holes have constraints.
\begin{defi}[Open automata refinement]
For two open automata \(A_1 \defobject \OA{S_1}{s_{01}}{J_1}{V_1}{\sigma_{01}}{T_1}\) and \(A_2 \defobject \OA{S_2}{s_{02}}{J_2}{V_2}{\sigma_{02}}{T_2}\), \(A_1\) is a refinement of \(A_2\) looking at \(H\), noted \(\wrel{A_1}{A_2}{H}\), with \(H \subseteq J_1 \cap J_2\), is defined as:
\begin{multline*}
	\exists R: \mpar{S_1 \times S_2} \to \rformulas[V_1 \uplus V_2], \mpar{\sigma_{01} \uplus \sigma_{02} \vdash R\mpar{s_{01}, s_{02}}} \wedge \forall \mpar{s_1, s_2} \in S_1 \times S_2, \\
	\mpar{\everymath{\displaystyle}\begin{array}{l}
		\bigsymb{\forall} \OTx{1}{}{1}{1} \in \fOT{s_1}, \bigsymb{\exists} \mpar{\OTx{2}{x}{2x}{2x} \in \fOT{s_2}}^{x \in X}, \\[12pt]
		\quad \mpar{\forall x \in X, J'_{2x} \cap H = J'_1 \cap H} \\[-10pt]
		\nwedge V_1 \uplus V_2 \uplus \fvars{t_1} \vdash R\mpar{s_1, s_2} \wedge g_1 \implies \operatorname*{\bigsymb{\bigvee}}_{x \in X} \mpar{\begin{array}{l}
			\alpha_1 = \alpha_{2x} \wedge \bigwedge_\subbox{j \in J'_{2x} \cap H} \beta_{1j} = \beta_{2xj} \\
			\nwedge g_{2x} \wedge R\mpar{s'_1, s'_{2x}}\psubst{\psi_1 \uplus \psi_{2x}}
		\end{array}} \\
	\end{array}} \\
	\wedge V_1 \uplus V_2 \uplus \biguplus_\subbox{t_2 \in \fOT{s_2}} \fvars{t_2} \vdash R\mpar{s_1, s_2} \wedge \bigvee_\subbox{t_2 \in \fOT{s_2}} \fguard{t_2} \implies \bigvee_\subbox{t_1 \in \fOT{s_1}} \fguard{t_1}
\end{multline*}
Giving \(R\) and \(H\) is sufficient to characterise the refinement, so we call \(\mpar{R, H}\) a hole-tracking simulation of \(A_1\) by \(A_2\).
\end{defi}
It is basically the same definition as the refinement relation deduced from hole-\(\triangle\) simulations.
\(R\) is a kind of hole-\(\triangle\) simulation but without the \(\triangle\) constraint on hole sets.
On top of that every actions of the holes outside \(H\) is unconstrained in the related automata.

The transitivity property becomes \(\wrel{A}{B}{X} \wedge \wrel{B}{C}{X} \implies \wrel{A}{C}{X \cap Y}\).
It behaves the same as if the holes outside the intersection were renamed to avoid conflicts in the first place.
This makes the comination of the transitivity proofs of hole-\(\triangle\) simulation sufficient to prove transitivity.

Not renaming conflicting holes still have an interest.
If we are trying to build an automaton \(A\) with several constraints \(\wrel{A}{B_i}{J_{Bi}}, i \in I\) -method B like- and each \(B_i\) is a refinement of some automaton \(C\) then we can have each \(B_i\) constrain a different subset of \(A\)'s holes.
On each individual path \(A \leq_{J_{Bi}} B_i \leq_{J_{Bi}} C\), some holes of \(A\) are unconstrained but when considering all path they are all constrained.
However this doesn't implies \(A \leq_{\bigcup_{i \in I} J_{Bi}} C\), it has to be checked separately.
\begin{exi}
\begin{figure}
\centering
\begin{tikzpicture}
\node[state,initial] (n) {id};
\draw (n) edge[dir=90] node {\OTd{x}{h \mapsto x}{}} (n);
\end{tikzpicture}
\vrule
\begin{tikzpicture}
\node[state,initial] (n) {*};
\draw (n) edge[dir=90] node {\OTd{x}{}{}} (n);
\end{tikzpicture}
\vrule
\begin{tikzpicture}
\node[state,initial] (n) {\textbar\textbar\textbar};
\draw (n) edge[dir=90] node {\OTd{x}{l \mapsto x, r \mapsto x}{}} (n);
\end{tikzpicture}
\caption{On the left: Generic identity on holes; On the middle: Any action automaton; On the right: Handshake composition.}
\label{fig:sevpathref}
\end{figure}
We have two refinement path when \(h\) is replaced by \(l\) and by \(r\) in Figure \ref{fig:sevpathref}: the parallel composition (left of Figure \ref{fig:hisim}) is a refinement of the identity on holes (right of Figure \ref{fig:sevpathref}), which is a refinement of the handshake composition (right of Figure \ref{fig:sevpathref}).
However we do not have the parallel composition being a refinement of the handshake composition looking at \(\mbrc{l, r}\).

The example also shows a property worth mentionning.
The holes not looked at (or untracked) are equivalent to having the automaton on the middle of Figure \ref{fig:sevpathref} composed.
Actually all the previous refinement are both ways: we can compose the Any action automaton in hole \(r\) of the parallel composition to get the other way of the refinement with the identity on holes by compositon correctness, and we transform the transition from the identity on holes into the one of the handshake composition by hole-superset refinement.
\end{exi}
The other interesting properties left to check are correctness w.r.t.\@ composition, context refinement, composition congruence and compatibility with FH-bisimulation.
Proving them for the open automata refinement will also prove them for each hole-\(\triangle\) simulation as they are just restricted cases.
\begin{thm} Open automata refinement is correct w.r.t.\@ composition \end{thm}
We will decompose composition into two refinement steps, transitivity will complete the proof.
Let \(A_1 \defobject \OAg[1]\), \(A_2 \defobject \OAg[2]\) and \(k \in J_1\) such that \(A_1\subst{A_2}{k}\) is non-locking.
We pose
\begin{gather*}
	A'_2 \defobject \OA{S_2}{s_{02}}{V_2}{\sigma_{02}}{\emptyset}{T'_2} \\
	T'_2 \defobject \mset{\OT{s}{s'}{\alpha}{\mbrc{}}{g}{\psi}}{\OTg \in T_2}
\end{gather*}
By limited composition correctness we have \(\wrel{A_1\subst{A'_2}{k}}{A_1}{J_1 \setminus \mbrc{k}}\).
Adding holes and binding actions to the new holes is a refinement for hole-superset simulation so we have \(\wrel{A_1\subst{A_2}{k}}{A_1\subst{A'_2}{k}}{J_1 \setminus \mbrc{k}}\).
By transitivity we have composition correctness.
\begin{thm}[Context refinement] Open automata refinement is context refining. \end{thm}
Let \(A_1 \defobject \OAg[1]\), \(A_2 \defobject \OAg[2]\) and \(A_3 \defobject \OAg[3]\).
Let \(\mpar{R_{12}, H_{12}}\) be a hole-tracking simulation of \(A_1\) by \(A_2\), \(k \in H_{12}\) and \(\reach{A_1\subst{A_3}{k}}\) be a witness that \(A_1\subst{A_3}{k}\) is non-locking.
The idea for the relation on configurations is that \(A_2\) simulates \(A_1\) while \(A_3\) does the same things in both compositions in the hole \(k\).
\begin{gather*}
	R \defobject \mpar{\mpar{s_1, s_{31}}, \mpar{s_2, s_{32}}} \mapsto \choice{R_{12}\mpar{s_1, s_2} \wedge \reach{A_1\subst{A_3}{k}}\mpar{s_1, s_{31}} \wedge \bigwedge_\subbox{v_3 \in V_3} v_{31} = v_{32} & \text{if } s_{31} = s_{32} \\ \bot & \text{otherwise}} \\
	H \defobject J_3 \uplus H_{12} \setminus \mbrc{k}
\end{gather*}
Where \(y_{3x}\) is the renaming of \(y_3\) from \(A_3\) in \(A_x\subst{A_3}{k}\).
\(\mpar{R, H}\) is a witness of \(\wrel{A_1\subst{A_3}{k}}{A_2\subst{A_3}{k}}{H}\), proof is given in Appendix \ref{apx:cr}.
\begin{thm}[Composition congruence] Open automata refinement is congruent for composition \end{thm}
Let \(A_1 \defobject \OAg[1]\), \(A_2 \defobject \OAg[2]\) and \(A_3 \defobject \OAg[3]\).
Let \(k \in J_1\), \(\mpar{R_{23}, H_{23}}\) be a hole-tracking simulation of \(A_2\) by \(A_3\) and \(\reach{A_1\subst{A_2}{k}}\) be a witness that \(A_1\subst{A_2}{k}\) is non-locking.
The idea for the relation on configurations is that \(A_3\) simulates \(A_2\) while \(A_1\) does the same things in both compositions.
\begin{gather*}
	R \defobject \mpar{\mpar{s_{12}, s_2}, \mpar{s_{13}, s_3}} \mapsto \choice{R_{23}\mpar{s_2, s_3} \wedge \reach{A_1\subst{A_2}{k}}\mpar{s_{12}, s_2} \wedge \bigwedge_\subbox{v_1 \in V_1} v_{12} = v_{13} & \text{if } s_{12} = s_{13} \\ \bot & \text{otherwise}} \\
	H \defobject H_{23} \uplus J_1 \setminus \mbrc{k}
\end{gather*}
Where \(y_{1x}\) is the renaming of \(y_1\) from \(A_1\) in \(A_1\subst{A_x}{k}\).
\(\mpar{R, H}\) is a witness of \(\wrel{A_1\subst{A_2}{k}}{A_1\subst{A_3}{k}}{H}\), proof is given in Appendix \ref{apx:cong}.
\begin{thm}[FH-bisimulation compatibility] Open automata refinement is compatible with FH-bisimulation \end{thm}
The requirements of FH-bisimulation are strictly stronger than the open automata refinement so using transitivity is sufficient.

With that last property, every interesting property is proved for the presented refinement relation.


\section{Conclusion}\label{sec:ccl}
In this report we redefined on open automata an operation from open pNets called composition.
This operation was defined as the composition of the open pNets used to generate the open automata and not directly on open automata.
So to reason on composition directly and to better separate the models the definition had to be given without introducing open pNets.
Then we recalled the main properties expected from a refinement relation on a structure with composition and adapted them to open automata.
To do that we restricted the composition so that it doesn't introduce deadlocks.
Then we introduced several refinement relations in restricted cases to avoid dealing immediately with the hard issue of the incomparability of hole sets.
Finally we introduced a refinement relation having all the interesting properties and dealing with the aforementioned issue at the cost of a slight modification on transitivity.

However the refinement presented here still do not work on the motivating example because of silent actions.
To be able to refine in the presence of silent action, a weak variant of the simulation is needed.
This variant will also probably be able to encompass action refinement as a valid refinement because the mechanism are similar.


\pagebreak
\bibliographystyle{plain}
\bibliography{biblio}

\pagebreak
\appendix
\part*{Appendix}

\section{Example of composition}

\subsection{Complete composition}\label{apx:composition}
\begin{figure}[h]
\centering
\begin{tikzpicture}

\node[state,initial] (v1) at (150:3.5) {R};
\node[state] (v2) at (30:3.5) {G};
\node[state] (v3) at (270:2) {Y};

\draw (v1) edge[dir=220] node[below] {\OTd{\act{tick}}{R \mapsto \act{tick}}{}} (v1);
\draw (v1) edge[dir=150] node[left] {\OTd{\act{onRed}}{}{}} (v1);
\draw (v1) edge[dir=80] node[above] {\OTd{\tau}{C \mapsto \theta\mpar{x}, R \mapsto \act{set}\mpar{x}}{}} (v1);
\draw (v1) edge node {\OTd{\act{TurnGreen}}{R \mapsto \act{over}\mpar{x}, C \mapsto \delta\mpar{x}}{}} (v2);
\draw (v2) edge[dir=100] node[above] {\OTd{\act{tick}}{R \mapsto \act{tick}}{}} (v2);
\draw (v2) edge[dir=30] node[right] {\OTd{\act{onGreen}}{}{}} (v2);
\draw (v2) edge[dir=320] node[xshift=1cm,below] {\OTd{\tau}{C \mapsto \theta\mpar{x}, R \mapsto \act{set}\mpar{x}}{}} (v2);
\draw (v2) edge node[pos=0.6] {\OTd{\act{TurnYellow}}{R \mapsto \act{over}\mpar{x}, C \mapsto \delta\mpar{x}}{}} (v3);
\draw (v3) edge[dir=340] node {\OTd{\act{tick}}{R \mapsto \act{tick}}{}} (v3);
\draw (v3) edge[dir=270] node {\OTd{\act{onYellow}}{}{}} (v3);
\draw (v3) edge[dir=200] node {\OTd{\tau}{C \mapsto \theta\mpar{x}, R \mapsto \act{set}\mpar{x}}{}} (v3);
\draw (v3) edge node[pos=0.4] {\OTd{\act{TurnRed}}{R \mapsto \act{over}\mpar{x}, C \mapsto \delta\mpar{x}}{}} (v1);

\end{tikzpicture}

\caption{The specification of a traffic light system}
\label{fig:tls}
\end{figure}
\begin{figure}
\centering
\begin{tikzpicture}

\node[state,initial] (v1) at (90:3) {};
\node[state] (v2) at (30:3) {};
\node[state] (v3) at (330:3) {};
\node[state] (v4) at (270:3) {};
\node[state] (v5) at (210:3) {};
\node[state] (v6) at (150:3) {};

\draw (v1) edge node {\OTd{\theta\mpar{17}}{}{}} (v2);
\draw (v2) edge node {\OTd{\delta\mpar{x}}{}{}} (v3);
\draw (v3) edge node {\OTd{\theta\mpar{3}}{}{}} (v4);
\draw (v4) edge node {\OTd{\delta\mpar{x}}{}{}} (v5);
\draw (v5) edge node {\OTd{\theta\mpar{20}}{}{}} (v6);
\draw (v6) edge node {\OTd{\delta\mpar{x}}{}{}} (v1);

\end{tikzpicture}

\vrule
\begin{tikzpicture}

\node[state,initial] (v1) at (0,1.5) {S};
\node[state] (v2) at (0,-1.5) {C};

\draw (v1) edge[bend left,looseness=0] node {\OTd{\act{set}\mpar{x}}{}{t \gets x, c \gets 0}} (v2);
\draw (v2) edge[dir=0] node {\OTd{\act{tick}}{}{c \gets c + 1}} (v2);
\draw (v2) edge[bend left,looseness=0] node {\OTd{\act{over}\mpar{c}}{}[c \geq t]{}} (v1);

\end{tikzpicture}

\caption{On the left: An example of controller agent; On the right: An example of counter agent}
\label{fig:tlh}
\end{figure}
This example is derived and adapted from a traffic light controller in a collection of examples for pNets.
It is supposed to be a single traffic light.

Figure \ref{fig:tls} shows the light controller with some synchronisation logic.
It takes an unimplemented control circuit in the hole \(ctl\) which gives the timings and an also unimplemented counter in the hole \(cnt\) to count external \(\act{tick}\) actions.
The three states are used to remember which colored light is on and this color can be retrieved by the environment by synchronising with the actions \(\act{onXxx}\) (Xxx is either Red, Yellow or Green) if it is not stored externally.

The color switches when the counter and the control circuit agree that the time is over.
The new time limit can be set by the control circuit and the exposed action to the exterior is a \(\tau\).

The components we choose to compose in the holes are in Figure \ref{fig:tlh}.
On the left there is the controller agent which will be composed in the hole \(ctl\).
Its role is to decide the duration before switching the lights.
This implementation is simply a constant time for each color; we could imagine using a more fancy controller which decides of the time depending on the traffic of each lane.

On the right is the counter agent which will be composed in the hole \(cnt\).
Its role is to get a target, then count ticks until that target is reached, then emit an action with the elapsed time and restart.
This implementation does exactly that and forbids any extra tick.

\begin{figure}
\centering
\begin{tikzpicture}

\node[state,initial] (v1) at (90:3.5) {R1S};
\node[state] (v2) at (30:3.5) {R2C};
\node[state] (v3) at (330:3.5) {G3S};
\node[state] (v4) at (270:3.5) {G4C};
\node[state] (v5) at (210:3.5) {Y5S};
\node[state] (v6) at (150:3.5) {Y6C};

\draw (v1) edge[dir=90] node {\OTd{\act{onRed}}{}{}} (v1);
\draw (v1) edge node[very near start] {\OTd{\tau}{}{t \gets 17, c \gets 0}} (v2);
\draw (v2) edge[dir=70] node[right] {\OTd{\act{onRed}}{}{}} (v2);
\draw (v2) edge[dir=350] node[right] {\OTd{\act{tick}}{}[c < t]{c \gets c + 1}} (v2);
\draw (v2) edge node {\OTd{\act{TurnGreen}}{}[c = t]{}} (v3);
\draw (v3) edge[dir=330] node[right] {\OTd{\act{onGreen}}{}{}} (v3);
\draw (v3) edge node[near end] {\OTd{\tau}{}{t \gets 20, c \gets 0}} (v4);
\draw (v4) edge[dir=310] node {\OTd{\act{onGreen}}{}{}} (v4);
\draw (v4) edge[dir=230] node {\OTd{\act{tick}}{}[c < t]{c \gets c + 1}} (v4);
\draw (v4) edge node {\OTd{\act{TurnYellow}}{}[c = t]{}} (v5);
\draw (v5) edge[dir=210] node {\OTd{\act{onYellow}}{}{}} (v5);
\draw (v5) edge node {\OTd{\tau}{}{t \gets 3, c \gets 0}} (v6);
\draw (v6) edge[dir=190] node[left] {\OTd{\act{onYellow}}{}{}} (v6);
\draw (v6) edge[dir=110] node[left] {\OTd{\act{tick}}{}[c < t]{c \gets c + 1}} (v6);
\draw (v6) edge node[near end] {\OTd{\act{TurnRed}}{}[c = t]{}} (v1);

\end{tikzpicture}

\caption{The full traffic lights system}
\label{fig:tlf}
\end{figure}
The automaton on Figure \ref{fig:tlf} is a simplification of the composition of the specification on Figure \ref{fig:tls} and the agents on Figure \ref{fig:tlh}.
The simplification was hand made because it is a hard problem as explained at the end of Section \ref{sec:comp}.
It consisted in removing unreachable states, reducing the size of the guard and reducing the amount of variables.
Otherwise the figure would have 36 states (\(3 \times 2 \times 6\)) but only 6 reachable from the initial configuration.
The simplified transitions are the following (where \(n, n+1\) stands for numbers between 1 and 6, Xxx is either Red, Yellow or Green and \(\mpar{Y, X}\) is accordingly \(\mpar{Y, R}\), \(\mpar{R, G}\) or \(\mpar{G, Y}\)):
\begin{itemize}
\item \nmm{\OT{XnS}{X(n+1)C}{\tau}{\mbrc{}}{\top}{\mbrc{t \gets k, c \gets 0}}} which is the composition of \nmm{\OT{X}{X}{\tau}{\mbrc{ctl \mapsto \theta\mpar{x}, cnt \mapsto \act{set}\mpar{x}}}{\top}{\mbrc{}}} from the specification with \nmm{\OT{S}{C}{\act{set}\mpar{x}}{\mbrc{}}{\top}{\mbrc{t \gets x, c \gets 0}}} from the counter and \nmm{\OT{n}{n+1}{\theta\mpar{k}}{\mbrc{}}{\top}{\mbrc{}}} from the controller.
\item \nmm{\OT{XnC}{XnC}{\act{tick}}{\mbrc{}}{c < t}{\mbrc{c \gets c + 1}}} which is the composition of \nmm{\OT{X}{X}{\act{tick}}{\mbrc{cnt \mapsto \act{tick}}}{\top}{\mbrc{}}} from the specification with \nmm{\OT{C}{C}{\act{tick}}{\mbrc{}}{c < t}{\mbrc{c \gets c + 1}}} from the counter.
\item \nmm{\OT{YnC}{X(n+1)S}{\act{TurnXxx}}{\mbrc{}}{c = t}{\mbrc{}}} which is the composition of \nmm{\OT{Y}{X}{\act{TurnXxx}}{\mbrc{cnt \mapsto \act{over}\mpar{x}, ctl \mapsto \delta\mpar{x}}}{\top}{\mbrc{}}} from the specification with \nmm{\OT{C}{S}{\act{over}\mpar{c}}{\mbrc{}}{c = t}{\mbrc{}}} from the counter and \nmm{\OT{n}{n+1}{\delta\mpar{x}}{\mbrc{}}{\top}{\mbrc{}}} from the controller.
\end{itemize}

To illustrate composition, hand made simplifications and value passing, the \(\tau\) transition from the state \(R\) will be examined.
The transition in the specification is \nmm{\OT{R}{R}{\tau}{\mbrc{cnt \mapsto \act{set}\mpar{x}, ctl \mapsto \theta\mpar{x}}}{\top}{\mbrc{}}}.
The composition can produce 12 states containing the \(R\) state, these states are in \(\mbrc{R} \times \mdbrk{1; 6} \times \mbrc{S, C}\), however only \(R1S\) and \(R2C\) are reachable so we will only consider \(\tau\) transitions from these two.
The holes \(cnt\) and \(ctl\) are both involved in the \(\tau\) transition and filled with automata so the composed automata must synchronise with a transition.

In the state \(R2C\) the composition produces 2 transitions by composition out of the \(\tau\) transition but both have a false guard; It is still interesting to look at one of them to show the product of composition without simplification.

\nmm{\OT{R2C}{R3C}{\tau}{\mbrc{}}{\top \wedge \top \wedge c < t \wedge \act{set}\mpar{x} = \act{tick} \wedge \theta\mpar{x} = \delta\mpar{x_C}}{\mbrc{c \gets c + 1}}} is made with \nmm{\OT{2}{3}{\delta\mpar{x}}{\mbrc{}}{\top}{\mbrc{}}} from the controller and \nmm{\OT{C}{C}{\act{tick}}{\mbrc{}}{c < t}{\mbrc{c \gets c + 1}}} from the counter.
Obviously the equality of actions in the guard cannot be satisfied so the guard is false, it is equivalent to have no transition.
We could go on to another transition but something worth explaining happened to build its guard.
The first two \(\top\) come from the guards of the specification and the controller.
\(c < t\) comes from the counter.
The rest of the guard comes from the equality constraint between holes actions and automata actions.
When composing this expression, we can see that the \(x\) variables from the specification, the controller and the counter are in conflict.
To resolve the conflict, the conflicted variables were renamed depending on where they come from.

In the state \(R1S\) the composition produces 1 transition out of the \(\tau\) transition.
The simplified transition follows the first pattern in the transitions described before.
Examining the composition process can illustrate how value passing work.
The obtained guard is \(\top \wedge \top \wedge \top \wedge \theta\mpar{x} = \theta\mpar{17} \wedge \act{set}\mpar{x} = \act{set}\mpar{x_R}\) and the obtained variable assignment is \(\mbrc{t \gets x_R, c \gets 0}\).
We can see how value passing works in the equality constraints: the only possible value of \(x\) and \(x_R\) making the guard hold is \(17\), this value has been synchronised from the controller to \(x\) then to the second occurrence of \(x\) to the distinct \(x\) of the counter.
Note that \nmm{\OT{S}{C}{\act{set}\mpar{t}}{\mbrc{}}{\top}{\mbrc{c \gets 0}}} wouldn't have worked as a transition for the controller because \(t\) is not a transition variable, its value is fixed before the transition and cannot be set outside of the variable update, hence the use of an auxiliary variable \(x\).

\subsection{Composition introducing deadlocks and requirements of non-locking composition}\label{apx:lockcomp}
\begin{exi}[Deadlocks introduced by composition]
This example reuses the counter introduced in Figure \ref{fig:tlh}.
The behaviour of the counter is to count external \(\act{tick}\) actions up to the amount set by action \(\act{set}\mpar{x}\) then to notify the environnement that this amount of time elapsed.
In the traffic light example the clock is not specified, we will introduce deadlocks using a wisely choosen clock specification.
However the full traffic light will not be used in order to simplify the example, only the counter will be.

\begin{figure}
\begin{tikzpicture}

\node[state,initial] (v1) at (0,1.5) {0};
\node[state] (v2) at (0,-1.5) {1};

\draw (v1) edge[dir=180] node {\OTd{x}{H \mapsto x}[x \neq \act{tick}]{}} (v1);
\draw (v1) edge[bend left,looseness=0] node {\OTd{\tau}{}{}} (v2);
\draw (v2) edge[bend left,looseness=0] node {\OTd{\act{tick}}{H \mapsto \act{tick}}{}} (v1);

\end{tikzpicture}

\vrule
\begin{tikzpicture}

\node[state,initial] (v1) at (0,1.5) {S};
\node[state] (v2) at (0,-1.5) {C};
\draw (v1) edge[dir=0] node {\OTd{\act{tick}}{}{}} (v1);
\draw (v1) edge[bend left,looseness=0] node {\OTd{\act{set}\mpar{x}}{}{t \gets x, c \gets 0}} (v2);
\draw (v2) edge[dir=0] node {\OTd{\act{tick}}{}{c \gets c + 1}} (v2);
\draw (v2) edge[bend left,looseness=0] node {\OTd{\act{over}\mpar{c}}{}[c \geq h]{}} (v1);

\end{tikzpicture}
\caption{On the left: A clock which imposes a tick; On the right: A modified version of the counter at figure \ref{fig:tlh}}
\label{fig:anytick}
\end{figure}
The clock on the left side of Figure \ref{fig:anytick} transmit the actions of its hole unchanged until the clock imposes a \(\act{tick}\) on its hole.
This can model a physical clock, because physical time ticks cannot be delayed.

If the hole cannot handle a \(\act{tick}\) at any time then there is a deadlock, which is the case with our counter.
The composition of this clock with the counter is given on the left of Figure \ref{fig:deadlock}.
\begin{figure}
\begin{tikzpicture}

\end{tikzpicture}
\vrule
\begin{tikzpicture}

\node[state] (v3) at (0:4) {1C};
\node[state] (v2) at (90:4) {0C};
\node[state,initial] (v1) at (180:4) {0S};
\node[state] (v4) at (270:4) {1S};

\draw  (v1) edge[bend left,looseness=0] node {\OTd{\act{set}\mpar{x}}{}{c \gets 0, t \gets x}} (v2);
\draw  (v2) edge[bend left,looseness=0] node {\OTd{\tau}{}{}} (v3);
\draw  (v3) edge[bend left,looseness=0] node[near start] {\OTd{\act{tick}}{}{c \gets c + 1}} (v2);
\draw  (v2) edge[bend left,looseness=0] node {\OTd{\act{over}\mpar{c}}{}[c \geq t]{}} (v1);
\draw  (v1) edge[bend left,looseness=0] node {\OTd{\tau}{}{}} (v4);
\draw  (v4) edge[bend left,looseness=0] node {\OTd{\act{tick}}{}{}} (v1);

\end{tikzpicture}
\caption{On the left: Deadlocks introduced by composition; On the right: No deadlock introduced by composition}
\label{fig:deadlock}
\end{figure}
There are two deadlocks:
The first is in the state \(1S\), which correspond to the clock trying to imppose a tick when the counter is not set.
The second is in the state \(1C\), which correspond to the clock trying to impose a tick when the counter has reached the amount peviously set, but not has not yet reported it to the environment.
These deadlocks could have been in the specification in which case they were intended, but it is not the case here.

The counter on the right of Figure \ref{fig:anytick} is a modification to accept ticks at in any state.
Composing this modified counter gives the automaton on the right of Figure \ref{fig:deadlock} which has no deadlocks.

We should successfully caracterise the second composition as non-locking but fail for the first.
\begin{itemize}
\item In the state \(0S\) any valuation is reachable (no initial valuation).
	In both automata the transition \nmm{\OT{0S}{1S}{\tau}{\mbrc{}}{\top}{\mbrc{}}} has a true guard, hence the first branch of the disjunction holds.
\item In the state \(0C\), for the left automaton, valuations where \(t \geq c \geq 0 \vee \mpar{t < 0 \wedge c = 0}\) are reachable; for the right automaton, valuations where \(c \geq 0\) are reachable.
	For all these valuations the transition \nmm{\OT{0C}{1C}{\tau}{\mbrc{}}{\top}{\mbrc{}}} has a true guard, hence the first branch of the disjunction holds.
\item In the state \(1C\), the same valuations as the ones reachable in \(0C\) are also reachable.
	In the right automaton the transition \nmm{\OT{1C}{0C}{\act{tick}}{\mbrc{}}{\top}{\mbrc{c \gets c + 1}}} has a true guard.
	In the left automaton the transition \nmm{\OT{1C}{0C}{\act{tick}}{\mbrc{}}{c < t}{\mbrc{c \gets c + 1}}} covers the cases where \(c < t\).
	For the other cases (\(c = t \vee t \leq 0 \mbrk{\leq c}\)) there are no transition with a true guard, and in the clock the transition same transition as for the state \(1S\) has a true guard, hence the composition is not non-locking.
\item In the state \(1S\) any valuation is reachable.
	In the left automaton, no transition is possible, and in the parent automaton \nmm{\OT{1}{0}{\act{tick}}{\mbrc{H \mapsto \act{tick}}}{\top}{\mbrc{}}} had a true guard, so the composition is not a non-locking composition (we already know it at this point).
	In the right automaton, \nmm{\OT{1S}{0S}{\act{tick}}{\mbrc{}}{\top}{\mbrc{}}} has a true guard.
\end{itemize}
The left automaton indeed fails at being the result of a non-locking composition where the right automaton passes.

One more thing worth noting is that for the left automaton, in the state \(1C\) with a valuation such that \(t < 0\), the counter agent is in a deadlock state.
We could have defined the non-locking composition so that this is considered as an intended deadlock, making the requirement symmetric.
However the the semantics of composition in a hole is asymmetric and the considered specification -as in the deadlock prevention- is the parent automaton.
\end{exi}


\section{Proof of equivalent definitions}\label{apx:lemeqd}

\subsection{Proof of lemma 1}
Let \(R\) be a pre-simulation between \(A_1 \defobject \OA{S_1}{s_{01}}{V_1}{\sigma_{01}}{J_1}{T_1}\) and \(A_2 \defobject \OA{S_2}{s_{02}}{V_2}{\sigma_{02}}{J_2}{T_2}\).
We want to prove:
\[ \forall \mpar{s_1, s_2} \in S_1 \times S_2, V_1 \uplus V_2 \uplus \biguplus_\subbox{t_2 \in \fOT{s_2}} \fvars{t_2} \vdash R\mpar{s_1, s_2} \wedge \bigvee_\subbox{t_2 \in \fOT{s_2}} \fguard{t_2} \implies \bigvee_\subbox{t_1 \in \fOT{s_1}} \fguard{t_1} \tag{WD}\label{eq:drWD} \]
\[ \bigsymb{\iff} \]
\begin{multline}
	\forall \mpar{s_1, s_2} \in S_1 \times S_2, \forall \sigma: V_1 \uplus V_2 \to \values, \mpar{\sigma \vdash R\mpar{s_1, s_2}} \implies \\
	\mpar{\exists t_2 \in \fOT{s_2}, \exists \nu: \fvars{t_2} \to \values, \sigma \uplus \nu \vdash \fguard{t_2}} \implies \\
	\bigsymb{\exists} \mpar{t_1, t_2} \defobject \mpar{\OTx{1}{}{1}{1}, \OTx{2}{}{2}{2}} \in \fOT{s_1} \times \fOT{s_2}, \exists \nu: \fvars{t_1} \uplus \fvars{t_2} \to \values, \\
	\sigma \uplus \nu \vdash g_1 \wedge g_2 \wedge \alpha_1 = \alpha_2 \wedge \bigwedge_\subbox{j \in J'_1 \cap J'_2} \beta_{1j} = \beta_{2j} \wedge R\mpar{s'_1, s'_2}\psubst{\psi_1 \uplus \psi_2}\tag{ID}\label{eq:drID}
\end{multline}
\begin{proof}
\item[\(\eqref{eq:drWD}\Rightarrow\eqref{eq:drID}\):]
	Let \(\mpar{s_1, s_2} \in S_1 \times S_2\) and \(\sigma: V_1 \uplus V_2 \to \values\) such that:
	\[ \sigma \vdash R\mpar{s_1, s_2} \hyp{1} \]
	We admit the left side of the second implication in \eqref{eq:drID}:
	\[ \exists t_2 \in \fOT{s_2}, \exists \nu: \fvars{t_2} \to \values, \sigma \uplus \nu \vdash \fguard{t_2} \hyp{2} \]
	And get immediately a transition \(t_2 \in \fOT{s_2}\), a valuation \(\nu: \fvars{t_2} \to \values\) and the hypothesis:
	\[ \sigma \uplus \nu \vdash \fguard{t_2} \hyp{2'} \]
	\(R\) satisfies \eqref{eq:drWD} with the current value of \(\mpar{s_1, s_2}\) and \(\sigma \uplus \nu\) completed with dummy values for the other variables of \nmm{\biguplus_\subbox{t_2 \in \fOT{s_2}} \fvars{t_2}} so we have:
	\[ \sigma \uplus \nu \vdash R\mpar{s_1, s_2} \wedge \bigvee_\subbox{t_2 \in \fOT{s_2}} \fguard{t_2} \implies \bigvee_\subbox{t_1 \in \fOT{s_1}} \fguard{t_1} \hyp{3} \]
	The \(\vdash\) hides an implicit \(\exists\) that we use to get values to the variables in \nmm{\biguplus_\subbox{t_1 \in \fOT{s_1}} \fvars{t_1}} in a valuation \(\mu\).
	The left side of the implication is proved using \hyp{1} and \hyp{2'} for the branch \(t_2\) of the disjunction.
	The right side is a disjunction so there is a \(t_1 \in \fOT{s_1}\) such that:
	\[ \sigma \uplus \nu \uplus \mu \vdash \fguard{t_1} \hyp{3'} \]
	At this point we have a pair \(\mpar{t_1, t_2}\) so we may be tempted to use it to prove the goal:
	\begin{multline}
		\bigsymb{\exists} \OTx{2}{}{2}{2} \in \fOT{s_2}, \OTx{1}{}{1}{1} \in \fOT{s_1}, \\
		\sigma \vdash g_2 \wedge \alpha_1 = \alpha_2 \wedge \bigwedge_\subbox{j \in J'_1 \cap J'_2} \beta_{1j} = \beta_{2j} \wedge g_1 \wedge R\mpar{s'_1, s'_2}\psubst{\psi_1 \uplus \psi_2} \goal{1}
	\end{multline}
	However we cannot yet prove it because we don't know whether \(t_2\) and \(t_1\) match.
	In order to get a \(t_2\) matching \(t_1\) we will use the fact that \(R\) is a pre-simulation:
	\begin{multline}
		\forall \mpar{s_1, s_2} \in S_1 \times S_2, \bigsymb{\forall} t_1 \defobject \OTx{1}{}{1}{1} \in \fOT{s_1}, \forall \sigma: V_1 \uplus V_2 \uplus \fvars{t_1} \to \values, \\
		\mpar{\sigma \vdash R\mpar{s_1, s_2} \wedge g_1} \implies \bigsymb{\exists} t_2 \defobject \OTx{2}{}{2}{2} \in \fOT{s_2}, \exists \nu: \fvars{t_2} \to \values, \\
		\sigma \uplus \nu \vdash \alpha_1 = \alpha_2 \wedge \bigwedge_\subbox{j \in J'_1 \cap J'_2} \beta_{1j} = \beta_{2j} \wedge g_2 \wedge R\mpar{s'_1, s'_2}\psubst{\psi_1 \uplus \psi_2} \hyp{0}
	\end{multline}
	\hyp{0} with current value of \(\mpar{s_1, s_2}\), \(t_1\) and \(\sigma \uplus \mu\) as a valuation of \(V_1 \uplus V_2 \uplus \fvars{t_1}\) gives:
	\begin{multline}
		\mpar{\sigma \uplus \mu \vdash R\mpar{s_1, s_2} \wedge g_1} \implies \exists t_2 \defobject \OTx{2}{}{2}{} \in \fOT{s_2}, \exists \nu: \fvars{t_2} \to \values, \\
		\sigma \uplus \mu \uplus \nu \vdash \alpha_1 = \alpha_2 \wedge \bigwedge_\subbox{j \in J'_1 \cap J'_2} \beta_{1j} = \beta_{2j} \wedge g_2 \wedge R\mpar{s'_1, s'_2}\psubst{\psi_1 \uplus \psi_2} \hyp{0'}
	\end{multline}
	The left part of the implication is \hyp{1} and \hyp{3'}, so we get \(t'_2 \in \fOT{s_2}\), \(\nu': \fvars{t_2} \to \values\) and:
	\[ \sigma \uplus \mu \uplus \nu' \vdash \alpha_1 = \alpha'_2 \wedge \bigwedge_\subbox{j \in J'_1 \cap J''_2} \beta_{1j} = \beta'_{2j} \wedge g'_2 \wedge R\mpar{s'_1, s''_2}\psubst{\psi_1 \uplus \psi'_2} \hyp{4} \]
	The witnesses for \goal{1} are \(t'_2\) and \(t_1\), and the rest is proved with a combination of \hyp{4} and \hyp{3'}.
\item[\(\eqref{eq:drWD}\Leftarrow\eqref{eq:drID}\):]
	Let \(\mpar{s_1, s_2} \in S_1 \times S_2\) and \nmm{\sigma: V_1 \uplus V_2 \uplus \biguplus_\subbox{t_2 \in \fOT{s_2}} \fvars{t_2} \to \values}.
	We want to prove:
	\[ \sigma \vdash R\mpar{s_1, s_2} \wedge \bigvee_\subbox{t_2 \in \fOT{s_2}} \fguard{t_2} \implies \bigvee_\subbox{t_1 \in \fOT{s_1}} \fguard{t_1} \goal{2} \]
	We can admit the left part of the implication in which the big disjunction gives us a \(t_2 \in \fOT{s_2}\) and:
	\begin{gather}
		\sigma \vdash R\mpar{s_1, s_2} \hyp{5} \\
		\sigma \vdash \fguard{t_2} \hyp{6}
	\end{gather}
	We are left to prove:
	\[ \sigma \vdash \bigvee_\subbox{t_1 \in \fOT{s_1}} \fguard{t_1} \goal{2'} \]
	To do that we use \eqref{eq:drID} with the current value of \(\mpar{s_1, s_2}\) and \(\sigma\), and prove the premisse of the implication with \hyp{5} to get:
	\begin{multline}
		\mpar{\exists t_2 \in \fOT{s_2}, \exists \nu: \fvars{t_2} \to \values, \sigma \uplus \nu \vdash \fguard{t_2}} \implies \\
		\bigsymb{\exists} t_1 \defobject \OTx{1}{}{1}{1} \in \fOT{s_1}, t_2 \defobject \OTx{2}{}{2}{2} \in \fOT{s_2}, \exists \nu: \fvars{t_1} \uplus \fvars{t_2} \to \values, \\
		\sigma \uplus \nu \vdash g_1 \wedge g_2 \wedge \alpha_1 = \alpha_2 \wedge \bigwedge_\subbox{j \in J'_1 \cap J'_2} \beta_{1j} = \beta_{2j} \wedge R\mpar{s'_1, s'_2}\psubst{\psi_1 \uplus \psi_2} \hyp{7}
	\end{multline}
	The left part of the implication is proved for \(t_2\) and \(\sigma\) (for the values covered by \(\nu\)) with \hyp{6}.
	The right part gives \(t_1\), \(t'_2\) and \(\nu\) such that (only the interesting part of the conjunction has been extracted):
	\[ \sigma \uplus \nu \vdash \fguard{t_1} \hyp{8} \]
	Which proves \goal{2'} with the values of \(\nu\) for the variables of \(\fguard{t_1}\) and dummy values for the other guards.
\end{proof}

\subsection{Proof of lemma 2}
We want to prove
\[ \forall s \defobject \mpar{s_p, s_c} \in S, V \uplus \biguplus_\subbox{t_p \in \fOT{s_p}} \fvars{t_p} \vdash \reach{A}\mpar{s} \wedge \bigvee_\subbox{t_p \in \fOT{s_p}} \fguard{t_p} \implies \bigvee_\subbox{t \in \fOT{s}} \fguard{t} \tag{WD}\label{eq:nlcWD} \]
\[ \bigsymb{\iff} \]
\begin{multline*}
	\forall s \defobject \mpar{s_p, s_c} \in S, \forall \sigma: V \to \values, \mpar{\sigma \vdash \reach{A}\mpar{s}} \implies \\
	\mpar{\exists t_p \in \fOT{s_p}, \exists \nu: \fvars{t_p} \to \values, \sigma \uplus \nu \vdash \fguard{t_p}} \implies \\
	\mpar{\exists t \in \fOT{s}, \exists \nu: \fvars{t} \to \values, \sigma \uplus \nu \vdash \fguard{t}} \tag{ID}\label{eq:nlcID}
\end{multline*}
\begin{proof}
\item[\(\eqref{eq:nlcWD}\Rightarrow\eqref{eq:nlcID}\):]
	Let \(s \defobject \mpar{s_p, s_c} \in S\) and \(\sigma \in V \to \values\) be such that:
	\[ \sigma \vdash \reach{A}\mpar{s} \hyp{1} \]
	We admit the left side of the second implication in \eqref{eq:nlcID}:
	\[ \exists t_p \in \fOT{s_p}, \exists \nu: \fvars{t_p} \to \values, \sigma \uplus \nu \vdash \fguard{t_p} \hyp{2} \]
	We get immediately a transition \(t_p \in \fOT{s_p}\), a valuation \(\nu: \fvars{t_p} \to \values\) and the hypothesis:
	\[ \sigma \uplus \nu \vdash \fguard{t_p} \hyp{2'} \]
	\(\reach{A}\) satisfies \eqref{eq:nlcWD} with the current value of \(s\) and \(\sigma \uplus \nu\) completed with dummy values for the other variables of \nmm{\biguplus_\subbox{t_p \in \fOT{s_p}} \fvars{t_p}} so we have:
	\[ \sigma \uplus \nu \vdash \reach{A}\mpar{s} \wedge \bigvee_\subbox{t_p \in \fOT{s_p}} \fguard{t_p} \implies \bigvee_\subbox{t \in \fOT{s}} \fguard{t} \hyp{3} \]
	The \(\vdash\) hides an implicit \(\exists\) that we use to get \nmm{\mu: \biguplus_\subbox{t \in \fOT{s}} \fvars{t} \to \values}.
	The left side of the implication is proved using \hyp{1} and \hyp{2'} for the branch \(t_p\) of the disjunction.
	The right side is a big disjunction that gives us \(t \in \fOT{s}\) such that:
	\[ \sigma \uplus \nu \uplus \mu \vdash \fguard{t} \hyp{3'} \]
	At this point we have a \(t\) which is composed of a transition in \(A_p\) and one in \(A_c\) so unlike lemma 1 we do not need to check that two transitions matches.
	The current goal is:
	\[ \exists t \in \fOT{s}, \exists \nu: \fvars{t} \to \values, \sigma \uplus \nu \vdash \fguard{t} \]
	Which is proved on \(t\) with valuation \(\mu\) by \hyp{3'}.
\item[\(\eqref{eq:nlcWD}\Leftarrow\eqref{eq:nlcID}\):]
	Let \(s \defobject \mpar{s_p, s_c} \in S\), \nmm{\sigma: V \uplus \biguplus_\subbox{t_p \in \fOT{s_p}} \fvars{t_p} \to \values}.
	We want to prove:
	\[ \sigma \vdash \reach{A}\mpar{s} \wedge \bigvee_\subbox{t_p \in \fOT{s_p}} \fguard{t_p} \implies \bigvee_\subbox{t \in \fOT{s}} \fguard{t} \goal{1} \]
	We can admit the left part of the implication in which the big disjunction gives us \(t_p \in \fOT{s_p}\) and:\\
	\begin{gather}
		\sigma \vdash \reach{A}\mpar{s} \hyp{4} \\
		\sigma \vdash \fguard{t_p} \hyp{5}
	\end{gather}
	We are left to prove:
	\[ \sigma \vdash \bigvee_\subbox{t \in \fOT{s}} \fguard{t} \goal{1'} \]
	To do that we use \eqref{eq:nlcID} with the current value of \(s\) and \(\sigma\), and prove the premisse of the implication with \hyp{4} to get:
	\begin{multline}
		\mpar{\exists t_p \in \fOT{s_p}, \exists \nu: \fvars{t_p} \to \values, \sigma \uplus \nu \vdash \fguard{t_p}} \implies \\
		\exists t \in \fOT{s}, \exists \nu: \fvars{t} \to \values, \sigma \uplus \nu \vdash \fguard{t} \hyp{6}
	\end{multline}
	The left part of the implication is proved for \(t_p\) and \(\sigma\) (for the values covered by \(\nu\)) with \hyp{5}.
	The right part gives \(t\) and \(\nu\) such that:
	\[ \sigma \uplus \nu \vdash \fguard{t} \hyp{7} \]
	Which proves \goal{1'} with the values of \(\nu\) for the variables of \(\fguard{t}\) and dummy values for the other guards.
\end{proof}


\section{Proofs of properties for hole-\(\triangle\) simulation}

\subsection{Hole-\(\triangle\) simulation is a pre-simulation}\label{apx:presim}
Let \(R\) be a relation on configurations.
The condition \(\sigma_{01} \uplus \sigma_{02} \vdash R\mpar{s_{01}, s_{02}}\) is in both definitions and the deadlock prevention requirement will not be used so proving the following is sufficient:
\begin{multline}
	\forall \mpar{s_1, s_2} \in S_1 \times S_2, \\
	\mpar{\everymath{\displaystyle}\begin{array}{l}
		\bigsymb{\forall} t_1 \defobject \OTx{1}{}{1}{1} \in \fOT{s_1}, \bigsymb{\exists} \mpar{t_{2x} \defobject \OTx{2}{x}{2x}{2x} \in \fOT{s_2}}^{x \in X}, \\
		\quad \mpar{\forall x \in X, J'_{2x} \cap J_1 = J'_1 \cap J_2} \\[-10pt]
		\nwedge V_1 \uplus V_2 \uplus \fvars{t_1} \vdash R\mpar{s_1, s_2} \wedge g_1 \implies \operatorname*{\bigsymb{\bigvee}}_{x \in X} \mpar{\begin{array}{l}
			\alpha_1 = \alpha_{2x} \wedge \bigwedge_\subbox{j \in J'_{2x} \cap J_1} \beta_{1j} = \beta_{2xj} \\[12pt]
			\nwedge g_{2x} \wedge R\mpar{s'_1, s'_{2x}}\psubst{\psi_1 \uplus \psi_{2x}}
		\end{array}}
	\end{array}} \hyp{0}
\end{multline}
\[ \bigsymb{\implies} \]\vspace{-1cm}
\begin{multline}
	\bigsymb{\forall} t_1 \defobject \OTx{1}{}{1}{1} \in \fOT{s_1}, \forall \sigma: V_1 \uplus V_2 \uplus \fvars{t_1} \to \values, \\
	\mpar{\sigma \vdash R\mpar{s_1, s_2} \wedge g_1} \implies \bigsymb{\exists} t_2 \defobject \OTx{2}{}{2}{2} \in \fOT{s_2}, \exists \nu: \fvars{t_2} \to \values, \\
	\sigma \uplus \nu \vdash \alpha_1 = \alpha_2 \wedge \bigwedge_\subbox{j \in J'_1 \cap J'_2} \beta_{1j} = \beta_{2j} \wedge g_2 \wedge R\mpar{s'_1, s'_2}\psubst{\psi_1 \uplus \psi_2} \goal{0}
\end{multline}
\begin{proof} Let \(\mpar{s_1, s_2} \in S_1 \times S_2\), \nmm{t_1 \defobject \OTx{1}{}{1}{1} \in \fOT{s_1}} and \(\sigma: V_1 \uplus V_2 \uplus \fvars{t_1} \to \values\) be such that
\[ \sigma \vdash R\mpar{s_1, s_2} \wedge g_1 \hyp{1} \]
In order to get the transition required to prove
\begin{multline}
	\bigsymb{\exists} t_2 \defobject \OTx{2}{}{2}{2} \in \fOT{s_2}, \exists \nu: \fvars{t_2} \to \values, \\
	\sigma \uplus \nu \vdash \alpha_1 = \alpha_2 \wedge \bigwedge_\subbox{j \in J'_1 \cap J'_2} \beta_{1j} = \beta_{2j} \wedge g_2 \wedge R\mpar{s'_1, s'_2}\psubst{\psi_1 \uplus \psi_2} \goal{0'}
\end{multline}
we use the hypothesis \hyp{0} with the transition \(t_1\) and get a family of transitions
\[ \mpar{t_{2x} \defobject \OTx{2}{x}{2x}{2x} \in \fOT{s_2}}^{x \in X} \]
such that the two following properties hold
\begin{gather}
	\forall x \in X, J'_{2x} \cap J_1 = J'_1 \cap J_2 \hyp{2} \\
	V_1 \uplus V_2 \uplus \fvars{t_1} \vdash R\mpar{s_1, s_2} \wedge g_1 \implies \operatorname*{\bigsymb{\bigvee}}_{x \in X} \mpar{\everymath{\displaystyle}\begin{array}{l}
		\alpha_1 = \alpha_{2x} \wedge \bigwedge_\subbox{j \in J'_{2x} \cap J_1} \beta_{1j} = \beta_{2xj} \\[12pt]
		\nwedge g_{2x} \wedge R\mpar{s'_1, s'_{2x}}\psubst{\psi_1 \uplus \psi_{2x}}
	\end{array}} \hyp{3}
\end{gather}
\hyp{3} holds for every values of the variables in \(V_1 \uplus V_2 \uplus \fvars{t_1}\) so in particular with the ones given by \(\sigma\).
We gather the variables of the implicit \(\exists\) in a valuation \nmm{\nu: \biguplus_{x \in X} \fvars{t_{2x}} \to \values}.
\[ \sigma \uplus \nu \vdash R\mpar{s_1, s_2} \wedge g_1 \implies \bigvee_{x \in X} \mpar{\alpha_1 = \alpha_{2x} \wedge \bigwedge_\subbox{j \in J'_{2x} \cap J_1} \beta_{1j} = \beta_{2xj} \wedge g_{2x} \wedge R\mpar{s'_1, s'_{2x}}\psubst{\psi_1 \uplus \psi_{2x}}} \hyp{3'} \]
The left side of the implication is proved by \hyp{1}, and for the right side we get \(x \in X\) such that
\[ \sigma \uplus \nu \vdash \alpha_1 = \alpha_{2x} \wedge \bigwedge_\subbox{j \in J'_{2x} \cap J_1} \beta_{1j} = \beta_{2xj} \wedge g_{2x} \wedge R\mpar{s'_1, s'_{2x}}\psubst{\psi_1 \uplus \psi_{2x}} \hyp{4} \]
We have \(J'_{2x} \cap J_1 = J'_{2x} \cap J'_1\) in \hyp{4}:
\begin{align*}
	J'_{2x} \cap J_1 & \subseteq J'_{2x} \tag{1} \\
	J'_{2x} \cap J_1 & = J'_1 \cap J_2 & \text{by \hyp{2}} \\
	& \subseteq J'_1 \tag{2} \\
	J'_{2x} \cap J_1 & \subseteq J'_{2x} \cap J'_1 & \text{by (1) and (2)} \tag{3} \\
	J'_{2x} \cap J'_1 & \subseteq J'_{2x} \cap J_1 & \text{by } J'_1 \subseteq J_1 \tag{4} \\
	J'_{2x} \cap J_1 & = J'_{2x} \cap J'_1 & \text{by (3) and (4)}
\end{align*}
So we can prove \goal{0'} for the transition \(t_{2x}\) and the valuation \(\nu\) using \hyp{4}.
\end{proof}

\subsection{Limited composition correctness for hole-subset simulation}\label{apx:lchs}
Let \(A_1 \defobject \OAg[1]\) and \(A_2 \defobject \OAg[2]\) be open automata, where \(J_2 = \emptyset\).
Let \(k \in J_1\) and \(A \defobject A_1\subst{A_2}{k}\), with \(\OAg \defobject A\).
Let \(\reach{A}\) be a witness that \(A\) is non-locking.
\[ R \defobject \mpar{\mpar{s_{1'}, s_2}, s_1} \mapsto \choice{\reach{A}\mpar{s_{1'}, s_2} \wedge \bigwedge_\subbox{v_1 \in V_1} v_{1'} = v_1 & \text{if } s_{1'} = s_1 \\ \bot & \text{otherwise}} \]
Where \(y_{1'}\) is the renaming of \(y_1\) from \(A_1\) in \(A\).
We want to prove that \(R\) is a hole-subset simulation of \(A\) by \(A_1\).
\begin{proof}
\item[1)] \(J = J_2 \uplus J_1 \setminus \mbrc{k} = \emptyset \uplus J_1 \setminus \mbrc{k} \subseteq J_1\)
\item[2)] We use the fact that \(R\) doesn't constrain the behaviour of \(J_2\) and initial configuration is reachable:
	\begin{align*}
		\sigma_0 \uplus \sigma_{01} \vdash R\mpar{s_0, s_{01}} \iff & \mpar{\reach{A}\mpar{s_0} \wedge \bigwedge_{v_1 \in V_1} v_{1'} = v_1}\psubst{\sigma_{01'} \uplus \sigma_{02} \uplus \sigma_{01}} \\
		\iff & \reach{A}\mpar{s_0}\psubst{\sigma_0} \wedge \bigwedge_\subbox{v_1 \in V_1} \sigma_{01'}\mpar{v_{1'}} = \sigma_{01}\mpar{v_1} \\
		\iff & \sigma_0 \vdash \reach{A}\mpar{s_0}
	\end{align*}
\item[3)] Let \(s \defobject \mpar{s_{1'}, s_2} \in S\) and \(s_1 \in S_1\).
	We want to prove both
	\begin{align*}
		& \bigsymb{\forall} t \defobject \OTg \in \fOT{s}, \bigsymb{\exists} \mpar{t_{1x} \defobject \OTx{1}{x}{1x}{1x} \in \fOT{s_1}}^{x \in X}, \\
		& \quad \mpar{\forall x \in X, J'_{1x} \cap J = J' \cap J_1} \\[-10pt]
		& \nwedge V \uplus V_1 \uplus \fvars{t} \vdash R\mpar{s, s_1} \wedge g \implies \operatorname*{\bigsymb{\bigvee}}_{x \in X} \mpar{\everymath{\displaystyle}\begin{array}{l}
			\alpha = \alpha_{1x} \wedge \bigwedge_\subbox{j \in J'_{1x} \cap J} \beta_j = \beta_{1xj} \\[12pt]
			\nwedge g_{1x} \wedge R\mpar{s', s'_{1x}}\psubst{\psi \uplus \psi_{1x}}
		\end{array}} \goal{0}
	\end{align*}
	\[ V \uplus V_1 \uplus \biguplus_\subbox{t_1 \in \fOT{s_1}} \fvars{t_1} \vdash R\mpar{s, s_1} \wedge \bigvee_\subbox{t_1 \in \fOT{s_1}} \fguard{t_1} \implies \bigvee_\subbox{t \in \fOT{s}} \fguard{t} \goal{1} \]
	Let \nmm{t \defobject \OTg \in \fOT{s}}, we know that \(t\) is obtained by composition so there is a generating transition \(t_1 \in \fOT{s_1}\) for which we will prove \goal{0}.
	If \(s_{1'} \neq s_1\) we instead prove \goal{0} with the empty family.
	\begin{gather}
		\forall x \in X, J'_1 \cap J = J' \cap J_1 \goal{0'} \\
		V \uplus V_1 \uplus \fvars{t} \vdash R\mpar{s, s_1} \wedge g \implies \alpha = \alpha_1 \wedge \bigwedge_\subbox{j \in J'_1 \cap J} \beta_j = \beta_{1j} \wedge g_1 \wedge R\mpar{s', s'_1}\psubst{\psi \uplus \psi_1} \goal{2}
	\end{gather}
\item[\goal{0'}:] \(t\) is a transition built from \(t_1\) so it has its holes.
	Additionnally either \(k \notin J'_1\) so \(J' = J'_1\) or \(k \in J'_1\) so \(J' = \emptyset \uplus J'_1 \setminus \mbrc{k}\) because \(A_2\) has no holes.
	Either case \(J'_1 \cap J = J'_1 \setminus \mbrc{k} = \mpar{J'_1 \setminus \mbrc{k}} \cap J_1 = J' \cap J_1\).
\item[\goal{2}:] Let \(\sigma: V \uplus V_1 \uplus \fvars{t} \to \values\).
	We rename some variables in the domain of definition of \(\sigma\) to get \(\nu: \fvars{t_1} \to \values\).
	We use \(\nu\) to prove \goal{2}, get the left side of the implication as an hypothesis and break the right side of the implication:
	\begin{gather}
		\sigma \vdash R\mpar{s, s_1} \wedge g \hyp{1} \\
		\sigma \uplus \nu \vdash \alpha = \alpha_1 \goal{2a} \\
		\sigma \uplus \nu \vdash \bigwedge_\subbox{j \in J'_1 \cap J} \beta_j = \beta_{1j} \goal{2b} \\
		\sigma \uplus \nu \vdash g_1 \goal{2c} \\
		\sigma \uplus \nu \vdash R\mpar{s', s'_1}\psubst{\psi \uplus \psi_1} \goal{2d}
	\end{gather}
	If \(s_{1'} \neq s_1\) then \hyp{1} is \(\sigma \vdash \bot\) so the proof is finished and the empty family worked.
	Otherwise we can assume \(s_{1'} = s_1\) and break \hyp{1} into
	\begin{gather}
		\sigma \vdash \reach{A}\mpar{s} \hyp{1a} \\
		\sigma \vdash \bigwedge_\subbox{v_1 \in V_1} v_{1'} = v_1 \hyp{1b} \\
		\sigma \vdash g \hyp{1c}
	\end{gather}
	\goal{2a} holds because composition doesn't change the produced action and the valuation coincide on the renamed variables.
	\goal{2b} also holds because \(J'_1 \cap J = J'_1 \setminus \mbrc{k} = J'\), composition doesn't change the actions of the holes involved and the valuation coincide on the renamed variables.
	\goal{2c} is proved using \hyp{1c} (\(g \equiv g_1\) or \(g \equiv g_1 \wedge \dots\)).
	For \goal{2d} we use \hyp{1a} with the preservation of reachability across transitions, \hyp{1b} and the fact that the valuations coincide on the renamed values:
	\begin{align*}
		& \sigma \uplus \nu \vdash R\mpar{s', s'_1}\psubst{\psi \uplus \psi_1} \\
		\iff & \sigma \uplus \nu \vdash \mpar{\reach{A}\mpar{s} \wedge \bigwedge_{v_1 \in V_1} v_{1'} = v_1}\psubst{\psi \uplus \psi_1} \\
		\iff & \sigma \uplus \nu \vdash \reach{A}\mpar{s}\psubst{\psi} \wedge \bigwedge_\subbox{v_1 \in V_1} \psi_{1'}\mpar{v_{1'}} = \psi_1\mpar{v_1} \\
		\impliedby & \sigma \uplus \nu \vdash \reach{A}\mpar{s}\psubst{\psi} \wedge \bigwedge_\subbox{v_1 \in V_1 \uplus \fvars{t_1}} v_{1'} = v_1 \\
		\iff & \mpar{\sigma \vdash \reach{A}\mpar{s}\psubst{\psi}} \wedge \mpar{\sigma \uplus \nu \vdash \bigwedge_\subbox{v_1 \in V_1} v_{1'} = v_1}
	\end{align*}
\item[\goal{1}:] Let \nmm{\sigma: V \uplus V_1 \uplus \biguplus_\subbox{t_1 \in \fOT{s_1}} \fvars{t_1} \to \values}.
	We use dummy values for the variables in \nmm{\biguplus_\subbox{t \in \fOT{s}} \fvars{t}} to get the left side of the implication (independant from these values), immediately broken into
	\begin{gather}
		\sigma \vdash \reach{A}\mpar{s} \hyp{2a} \\
		\sigma \vdash \bigwedge_\subbox{v_1 \in V_1} v_{1'} = v_1 \hyp{2b} \\
		\sigma \vdash \bigvee_\subbox{t_1 \in \fOT{s_1}} \fguard{t_1} \hyp{2c}
	\end{gather}
	\(\reach{A}\) is a witness of the non-locking composition so we have the hypothesis:
	\[ V \uplus \biguplus_\subbox{t_1 \in \fOT{s_1}} \fvars{t_1} \vdash \reach{A}\mpar{s} \wedge \bigvee_\subbox{t_1 \in \fOT{s_1}} \fguard{t_1} \implies \bigvee_\subbox{t \in \fOT{s}} \fguard{t} \hyp{3} \]
	We use \hyp{3} with valuation \(\sigma\) to get \nmm{\nu: \biguplus_\subbox{t \in \fOT{s}} \fvars{t} \to \values}.
	The left side of the implication is proved with \hyp{2a} and \hyp{2c}, the right side gives the hypothesis:
	\[ \sigma \uplus \nu \vdash \bigvee_\subbox{t \in \fOT{s}} \fguard{t} \hyp{3'} \]
	We prove \goal{1} with valuation \(\nu\) by admitting the left side of the implication and using \hyp{3'}.
\end{proof}

\subsection{Special composition correctness for hole-superset simulation}\label{apx:schs}
Let \(A_1 \defobject \OAg[1]\), \(A_2 \defobject \OAg[2]\) and \(A_3 \defobject \OAg[3]\) be open automata.
Let \(k \in J_1 \setminus J_2\) and \(A \defobject A_1\subst{A_3}{k}\), with \(\OAg \defobject A\).
Let \(R_{12}\) be a hole-superset simulation of \(A_1\) by \(A_2\) and \(\reach{A}\) be a witness that \(A\) is non-locking.
\[ R \defobject \mpar{\mpar{s_1, s_3}, s_2} \mapsto R_{12}\mpar{s_1, s_2} \wedge \reach{A}\mpar{s_1, s_3} \]
We want to prove that \(R\) is a hole-superset simulation of \(A\) by \(A_2\).
\begin{proof}
\item[1)] \(J = J_3 \uplus J_1 \setminus \mbrc{k} \supseteq J_3 \uplus J_2 \setminus \mbrc{k} \supseteq J_2 \setminus \mbrc{k} = J_2\)
\item[2)] We use the fact that \(R_{12}\) relates initial configurations and that initial configuration is reachable:
	\begin{align*}
		\sigma_0 \uplus \sigma_{02} \vdash R\mpar{s_0, s_{02}} \iff & \sigma_{01} \uplus \sigma_{02} \uplus \sigma_{03} \vdash R_{12}\mpar{s_{01}, s_{02}} \wedge \reach{A}\mpar{s_0} \\
		\iff & \mpar{\sigma_{01} \uplus \sigma_{02} \vdash R_{12}\mpar{s_{01}, s_{02}}} \wedge \mpar{\sigma_0 \vdash \reach{A}\mpar{s_0}}
	\end{align*}
\item[3)] Let \(s \defobject \mpar{s_1, s_3} \in S\) and \(s_2 \in S_2\).
	We want to prove both
	\begin{align*}
		& \bigsymb{\forall} t \defobject \OTg \in \fOT{s}, \bigsymb{\exists} \mpar{t_{2x} \defobject \OTx{2}{x}{2x}{2x} \in \fOT{s_2}}^{x \in X}, \\
		& \quad \mpar{\forall x \in X, J'_{2x} \cap J = J' \cap J_2} \\[-10pt]
		& \nwedge V \uplus V_2 \uplus \fvars{t} \vdash R\mpar{s, s_2} \wedge g \implies \operatorname*{\bigsymb{\bigvee}}_{x \in X} \mpar{\everymath{\displaystyle}\begin{array}{l}
			\alpha = \alpha_{2x} \wedge \bigwedge_\subbox{j \in J'_{2x} \cap J} \beta_j = \beta_{2xj} \\[12pt]
			\nwedge g_{2x} \wedge R\mpar{s', s'_{2x}}\psubst{\psi \uplus \psi_{2x}}
		\end{array}} \goal{0}
	\end{align*}
	\[ V \uplus V_2 \uplus \biguplus_\subbox{t_2 \in \fOT{s_2}} \fvars{t_2} \vdash R\mpar{s, s_2} \wedge \bigvee_\subbox{t_2 \in \fOT{s_2}} \fguard{t_2} \implies \bigvee_\subbox{t \in \fOT{s}} \fguard{t} \goal{1} \]
	Let \nmm{t \defobject \OTg \in \fOT{s}}, we use the fact that \(R_{12}\) is a simulation from the states \(\mpar{s_1, s_2}\) to get the hypothesis:
	\begin{align*}
		& \bigsymb{\forall} t_1 \defobject \OTx{1}{}{1}{1} \in \fOT{s_1}, \bigsymb{\exists} \mpar{t_{2x} \defobject \OTx{2}{x}{2x}{2x} \in \fOT{s_2}}^{x \in X}, \\
		& \quad \mpar{\forall x \in X, J'_{2x} \cap J_1 = J'_1 \cap J_2} \\[-10pt]
		& \nwedge V_1 \uplus V_2 \uplus \fvars{t_1} \vdash R_{12}\mpar{s_1, s_2} \wedge g_1 \implies \operatorname*{\bigsymb{\bigvee}}_{x \in X} \mpar{\everymath{\displaystyle}\begin{array}{l}
			\alpha_1 = \alpha_{2x} \wedge \bigwedge_\subbox{j \in J'_{2x} \cap J_1} \beta_{1j} = \beta_{2xj} \\[12pt]
			\nwedge g_{2x} \wedge R_{12}\mpar{s'_1, s'_{2x}}\psubst{\psi_1 \uplus \psi_{2x}}
		\end{array}} \hyp{1}
	\end{align*}
	Transition \(t\) is obtained by composition of \nmm{t_1 \defobject \OTx{1}{}{1}{1} \in \fOT{s_1}} and \nmm{t_3 \defobject \OTx{3}{}{3}{3} \in \fOT{s_3}} if \(k \in J'_1\) or by a the first transition and the state \(s_3\) if \(k \notin J'_1\).
	We use \hyp{1} with the transition \(t_1\) to get
	\begin{gather*}
		\mpar{t_{2x} \defobject \OTx{2}{x}{2x}{2x} \in \fOT{s_2}}^{x \in X} \\
		\forall x \in X, J'_{2x} \cap J_1 = J'_1 \cap J_2 \hyp{1'} \\
		V_1 \uplus V_2 \uplus \fvars{t_1} \vdash R_{12}\mpar{s_1, s_2} \wedge g_1 \implies \operatorname*{\bigsymb{\bigvee}}_{x \in X} \mpar{\everymath{\displaystyle}\begin{array}{l}
			\alpha_1 = \alpha_{2x} \wedge \bigwedge_\subbox{j \in J'_{2x} \cap J_1} \beta_{1j} = \beta_{2xj} \\[12pt]
			\nwedge g_{2x} \wedge R_{12}\mpar{s'_1, s'_{2x}}\psubst{\psi_1 \uplus \psi_{2x}}
		\end{array}} \hyp{1"}
	\end{gather*}
	We will prove \goal{0} for the family \(t_{2x}\), the new goals are
	\begin{gather}
		\forall x \in X, J'_{2x} \cap J = J' \cap J_2 \goal{0'} \\
		V \uplus V_2 \uplus \fvars{t} \vdash R\mpar{s, s_2} \wedge g \implies \operatorname*{\bigsymb{\bigvee}}_{x \in X} \mpar{\everymath{\displaystyle}\begin{array}{l}
			\alpha = \alpha_{2x} \wedge \bigwedge_\subbox{j \in J'_{2x} \cap J} \beta_j = \beta_{2xj} \\[12pt]
			\nwedge g_{2x} \wedge R\mpar{s', s'_{2x}}\psubst{\psi \uplus \psi_{2x}}
		\end{array}} \goal{2}
	\end{gather}
\item[\goal{0'}:] Let \(x \in X\)
	\begin{align*}
		J'_{2x} \cap J & = J'_{2x} \cap \mpar{J_3 \uplus J_1 \setminus \mbrc{k}} \\
		& = J'_{2x} \cap J_1 & \text{by } k \in J_1 \setminus J_2 \text{ and } \emptyset = J_3 \cap J_1 \supseteq J_3 \cap J_2 \hyp{2} \\
		& = J'_1 \cap J_2 & \text{by \hyp{1'}} \\
		& = \mpar{J'_1 \uplus J'_3} \cap J_2 \\
		& = J' \cap J_2
	\end{align*}
\item[\goal{2}:] Let \nmm{\sigma: V \uplus V_2 \uplus \fvars{t} \to \values}.
	We use dummy values for the variables in \nmm{\biguplus_\subbox{t_{2x} \in \fOT{s_2}} \fvars{t_{2x}}} to get the left side of the implication (independant from these values), immediately broken into
	\begin{gather}
		\sigma \vdash R_{12}\mpar{s_1, s_2} \hyp{3a} \\
		\sigma \vdash \reach{A}\mpar{s} \hyp{3b} \\
		\sigma \vdash g \hyp{3c}
	\end{gather}
	We use \hyp{1"} with valuation \(\sigma\) on variables \(V_1 \uplus V_2 \uplus \fvars{t_1}\), which is possible because the composition of transitions doesn't remove any variable.
	Then we get the values of the implicit \(\exists\) in \nmm{\nu: \biguplus_{x \in X} \fvars{t_{2x}}}.
	The left side of the implication is proved with \hyp{3a} and \hyp{3c}.
	So we get a \(x \in X\) such that
	\begin{gather}
		\sigma \uplus \nu \vdash \alpha_1 = \alpha_{2x} \hyp{4a} \\
		\sigma \uplus \nu \vdash \bigwedge_\subbox{j \in J'_{2x} \cap J_1} \beta_{1j} = \beta_{2xj} \hyp{4b} \\
		\sigma \uplus \nu \vdash g_{2x} \hyp{4c} \\
		\sigma \uplus \nu \vdash R_{12}\mpar{s'_1, s'_{2x}}\psubst{\psi_1 \uplus \psi_{2x}} \hyp{4d}
	\end{gather}
	In order to progress on \goal{2} we give valuation \(\nu\) for the implicit \(\exists\), admit the left side of the implication and choose to prove the branch \(x\):
	\begin{gather}
		\sigma \uplus \nu \vdash \alpha = \alpha_{2x} \goal{2a} \\
		\sigma \uplus \nu \vdash \bigwedge_\subbox{j \in J'_{2x} \cap J} \beta_j = \beta_{2xj} \goal{2b} \\
		\sigma \uplus \nu \vdash g_{2x} \goal{2c} \\
		\sigma \uplus \nu \vdash R\mpar{s', s'_{2x}}\psubst{\psi \uplus \psi_{2x}} \goal{2d}
	\end{gather}
	\goal{2a} is proved by \hyp{4a} (\(\alpha = \alpha_1\)).
	\goal{2b} is proved using \hyp{4b} because \(J'_{2x} \cap J = J'_{2x} \cap J_1 = J'_1 \cap J_2\) (\hyp{2} and \hyp{1'}) and \(\forall j \in J'_1, \beta_{1j} = \beta_j\) (transition comes from composition).
	\goal{2c} is proved by \hyp{4c}.
	For \goal{2d} we use \hyp{4d} and \hyp{3b} with the preservation of reachability across transitions:
	\begin{align*}
		& \sigma \uplus \nu \vdash R\mpar{s', s'_{2x}}\psubst{\psi \uplus \psi_{2x}} \\
		\iff & \sigma \uplus \nu \vdash \mpar{R_{12}\mpar{s'_1, s'_{2x}} \wedge \reach{A}\mpar{s}}\psubst{\psi_1 \uplus \psi_{2x} \uplus \psi_3} \\
		\iff & \mpar{\sigma \uplus \nu \vdash R_{12}\mpar{s'_1, s'_{2x}}\psubst{\psi_1 \uplus \psi_{2x}}} \wedge \mpar{\sigma \vdash \reach{A}\mpar{s}\psubst{\psi}}
	\end{align*}
\item[\goal{1}:]
	Let \nmm{\sigma: V \uplus V_2 \uplus \biguplus_\subbox{t_2 \in \fOT{s_2}} \fvars{t_2} \to \values}.
	We use dummy values for the variables in \nmm{\biguplus_\subbox{t \in \fOT{s}} \fvars{t}} to get the left side of the implication (independant from these values), immediately broken into
	\begin{gather}
		\sigma \vdash R_{12}\mpar{s_1, s_2} \hyp{5a} \\
		\sigma \vdash \reach{A}\mpar{s} \hyp{5b} \\
		\sigma \vdash \bigvee_\subbox{t_2 \in \fOT{s_2}} \fguard{t_2} \hyp{5c}
	\end{gather}
	\(R_{12}\) is a simulation and \(\reach{A}\) is a witness of the non-locking composition so we have the hypotheses:
	\begin{gather}
		V_1 \uplus V_2 \uplus \biguplus_\subbox{t_2 \in \fOT{s_2}} \fvars{t_2} \vdash R_{12}\mpar{s_1, s_2} \wedge \bigvee_\subbox{t_2 \in \fOT{s_2}} \fguard{t_2} \implies \bigvee_\subbox{t_1 \in \fOT{s_1}} \fguard{t_1} \hyp{6} \\
		V \uplus \biguplus_\subbox{t_1 \in \fOT{s_1}} \fvars{t_1} \vdash \reach{A}\mpar{s} \wedge \bigvee_\subbox{t_1 \in \fOT{s_1}} \fguard{t_1} \implies \bigvee_\subbox{t \in \fOT{s}} \fguard{t} \hyp{7}
	\end{gather}
	We use \hyp{6} with valuation \(\sigma\) to get \nmm{\nu: \biguplus_\subbox{t_1 \in \fOT{s_1}} \fvars{t_1} \to \values}.
	The left side of the implication is proved with \hyp{5a} and \hyp{5c}, the right side gives the hypothesis:
	\[ \sigma \uplus \nu \vdash \bigvee_\subbox{t_2 \in \fOT{s_2}} \fguard{t_2} \hyp{6'} \]
	We then use \hyp{7} with valuation \(\sigma \uplus \nu\) to get \nmm{\nu': \biguplus_\subbox{t \in \fOT{s}} \fvars{t}}.
	The left side of the implication is proved using \hyp{5b} and \hyp{6'}, the right side gives:
	\[ \sigma \uplus \nu' \vdash \bigvee_\subbox{t \in \fOT{s}} \fguard{t} \hyp{7'} \]
	We prove \goal{1} with valuation \(\nu'\) by admitting the left side of the implication and using \hyp{7'}.
\end{proof}

\section{Proof of transitivity}\label{apx:trans}
Let \(A_1 \defobject \OAg[1]\), \(A_2 \defobject \OAg[2]\) and \(A_3 \defobject \OAg[3]\) be open automata.
Let \(R_{12}\) be a hole-\(\triangle\) simulation of \(A_1\) by \(A_2\) and \(R_{23}\) of \(A_2\) by \(A_3\).
\[ R \defobject \mpar{s_1, s_3} \mapsto \exists V_2, \bigvee_\subbox{s_2 \in S_2} R_{12}\mpar{s_1, s_2} \wedge R_{23}\mpar{s_2, s_3} \]
We want to prove that \(R\) is a hole-\(\triangle\) simulation of \(A_1\) by \(A_3\).
\begin{proof}
\item[1)] By transitivity of \(\triangle\), \(J_1 \triangle J_3\)
\item[2)] We prove that initial configurations are related using \(\sigma_{02}\) as the values of the variables in \(V_2\), and \(s_{02}\) to choose the branch of the disjunction.
\begin{align*}
	\sigma_{01} \uplus \sigma_{03} \vdash R\mpar{s_{01}, s_{03}} \iff & \sigma_{01} \uplus \sigma_{03} \vdash \exists V_2, \bigvee_\subbox{s_2 \in S_2} R_{12}\mpar{s_{01}, s_2} \wedge R_{23}\mpar{s_2, s_{03}} \\
	\impliedby & \sigma_{01} \uplus \sigma_{02} \uplus \sigma_{03} \vdash R_{12}\mpar{s_{01}, s_{02}} \wedge R_{23}\mpar{s_{02}, s_{03}} \\
	\impliedby & \mpar{\sigma_{01} \uplus \sigma_{02} \vdash R_{12}\mpar{s_{01}, s_{02}}} \wedge \mpar{\sigma_{02} \uplus \sigma_{03} \vdash R_{23}\mpar{s_{02}, s_{03}}}
\end{align*}
\item[3)] Let \(\mpar{s_1, s_3} \in S_1 \times S_3\), we want to prove both
	\begin{align*}
		& \bigsymb{\forall} t_1 \defobject \OTx{1}{}{1}{1} \in \fOT{s_1}, \bigsymb{\exists} \mpar{t_{3x} \defobject \OTx{3}{x}{3x}{3x} \in \fOT{s_3}}^{x \in X}, \\
		& \mpar{\forall x \in X, J'_{3x} \cap J_1 = J'_1 \cap J_3} \\[-10pt]
		& \wedge V_1 \uplus V_3 \uplus \fvars{t_1} \vdash R\mpar{s_1, s_3} \wedge g_1 \implies \operatorname*{\bigsymb{\bigvee}}_{x \in X} \mpar{\everymath{\displaystyle}\begin{array}{l}
			\alpha_1 = \alpha_{3x} \wedge \bigwedge_\subbox{j \in J'_{3x} \cap J_1} \beta_{1j} = \beta_{3xj} \\[12pt]
			\nwedge g_{3x} \wedge R\mpar{s'_1, s'_{3x}}\psubst{\psi_1 \uplus \psi_{3x}}
		\end{array}} \goal{0}
	\end{align*}
	\[ V_1 \uplus V_3 \uplus \biguplus_\subbox{t_3 \in \fOT{s_3}} \fvars{t_3} \vdash R\mpar{s_1, s_3} \wedge \bigvee_\subbox{t_3 \in \fOT{s_3}} \fguard{t_3} \implies \bigvee_\subbox{t_1 \in \fOT{s_1}} \fguard{t_1} \goal{1} \]
\item[\goal{0}:] Let \nmm{t_1 \defobject \OTx{1}{}{1}{1} \in \fOT{s_1}}.
	For every \(s_2 \in S_2\) we have the hypotheses on \(R_{12}\) and \(R_{23}\):
	\begin{gather}
		\everymath{\displaystyle}\begin{array}{l}
		\bigsymb{\forall} t_1 \defobject \OTx{1}{}{1}{1} \in \fOT{s_1}, \bigsymb{\exists} \mpar{t_{s2x} \defobject \OTx{2}{x}{2x}{2x} \in \fOT{s_2}}^{x \in X}, \\
		\quad \mpar{\forall x \in X, J'_{s2x} \cap J_1 = J'_1 \cap J_2} \\[-10pt]
		\nwedge V_1 \uplus V_2 \uplus \fvars{t_1} \vdash R_{12}\mpar{s_1, s_2} \wedge g_1 \implies \operatorname*{\bigsymb{\bigvee}}_{x \in X} \mpar{\begin{array}{l}
			\alpha_1 = \alpha_{2x} \wedge \bigwedge_\subbox{j \in J'_{s2x} \cap J_1} \beta_{1j} = \beta_{2xj} \\[12pt]
			\nwedge g_{2x} \wedge R_{12}\mpar{s'_1, s'_{2x}}\psubst{\psi_1 \uplus \psi_{2x}}
		\end{array}}
		\end{array} \hyp{0} \\
		\everymath{\displaystyle}\begin{array}{l}
		\bigsymb{\forall} t_2 \defobject \OTx{2}{}{2}{2} \in \fOT{s_2}, \bigsymb{\exists} \mpar{t_{3x} \defobject \OTx{3}{x}{3x}{3x} \in \fOT{s_3}}^{x \in X}, \\
		\quad \mpar{\forall x \in X, J'_{3x} \cap J_2 = J'_2 \cap J_3} \\[-10pt]
		\nwedge V_2 \uplus V_3 \uplus \fvars{t_{s2}} \vdash R_{23}\mpar{s_2, s_3} \wedge g_2 \implies \operatorname*{\bigsymb{\bigvee}}_{x \in X} \mpar{\begin{array}{l}
			\alpha_2 = \alpha_{3x} \wedge \bigwedge_\subbox{j \in J'_{3x} \cap J_2} \beta_{2j} = \beta_{3xj} \\[12pt]
			\nwedge g_{3x} \wedge R_{23}\mpar{s'_2, s'_{3x}}\psubst{\psi_2 \uplus \psi_{3x}}
		\end{array}}
		\end{array} \hyp{1}
	\end{gather}
	We use hypothesis \hyp{0} with \(t_1\) for every \(s_2 \in S_2\) to get
	\begin{gather*}
		\mpar{t_{s2x} \defobject \OT{s_2}{s'_{s2x}}{\alpha_{s2x}}{\beta_{s2xj}^{j \in J'_{s2x}}}{g_{s2x}}{\psi_{s2x}} \in \fOT{s_2}}^{x \in X_{s2}} \\
		\forall x \in X, J'_{s2x} \cap J_1 = J'_1 \cap J_2 \hyp{0'} \\
		V_1 \uplus V_2 \uplus \fvars{t_1} \vdash R_{12}\mpar{s_1, s_2} \wedge g_1 \implies \operatorname*{\bigsymb{\bigvee}}_{x \in X_{s2}} \mpar{\everymath{\displaystyle}\begin{array}{l}
			\alpha_1 = \alpha_{s2x} \wedge \bigwedge_\subbox{j \in J'_{s2x} \cap J_1} \beta_{1j} = \beta_{s2xj} \\[12pt]
			\nwedge g_{s2x} \wedge R_{12}\mpar{s'_1, s'_{s2x}}\psubst{\psi_1 \uplus \psi_{s2x}}
		\end{array}} \hyp{0"}
	\end{gather*}
	Then use \hyp{1} with every \(t_{s2x}\) to get
	\begin{gather*}
		Z_{s2x} \defobject \mbrc{s_2} \times \mbrc{x} \times Y_{s2x} \\
		\mpar{t_{3z} \defobject \OTx{3}{z}{3z}{3z} \in \fOT{s_3}}^{z \in Z_{s2x}} \\
		\forall z \in Z_{s2x}, J'_{3z} \cap J_2 = J'_{s2x} \cap J_3 \hyp{1x} \\
		V_2 \uplus V_3 \uplus \fvars{t_{s2x}} \vdash R_{23}\mpar{s_2, s_3} \wedge g_{s2x} \implies \operatorname*{\bigsymb{\bigvee}}_\subbox{z \in Z_{s2x}} \mpar{\everymath{\displaystyle}\begin{array}{l}
			\alpha_{s2x} = \alpha_{3z} \wedge \bigwedge_\subbox{j \in J'_{3z} \cap J_2} \beta_{s2xj} = \beta_{3zj} \\[12pt]
			\nwedge g_{3z} \wedge R_{23}\mpar{s'_{s2x}, s'_{3z}}\psubst{\psi_{s2x} \uplus \psi_{3z}}
		\end{array}} \hyp{1x'}
	\end{gather*}
	We pose \nmm{Z \defobject \biguplus_{s_2 \in S_2} \biguplus_{x \in X_{s2}} Z_{s2x}}.
	We will prove \goal{0} for the family \(t_{3z}^{z \in Z}\).
	The new goals are:
	\begin{gather}
		\forall z \in Z, J'_{3z} \cap J_1 = J'_1 \cap J_3 \goal{0'} \\
		V_1 \uplus V_3 \uplus \fvars{t_1} \vdash R\mpar{s_1, s_3} \wedge g_1 \implies \operatorname*{\bigsymb{\bigvee}}_{z \in Z} \mpar{\everymath{\displaystyle}\begin{array}{l}
			\alpha_1 = \alpha_{3z} \wedge \bigwedge_\subbox{j \in J'_{3z} \cap J_1} \beta_{1j} = \beta_{3zj} \\[12pt]
			\nwedge g_{3z} \wedge R\mpar{s'_1, s'_{3z}}\psubst{\psi_1 \uplus \psi_{3z}}
		\end{array}} \goal{2}
	\end{gather}
\item[\goal{0'}:] By case on \(\triangle\):
	\begin{itemize}
	\item Let \(\triangle\) be \(\subseteq\).
		\begin{align*}
		J'_{3z} \cap J_1 & \subseteq J'_{3z} \cap J_2 & \text{by } & J_1 \subseteq J_2 \\
			& \subseteq J'_{2x} \cap J_3 & \text{by } & \text{\hyp{1x}} \hyp{2} \\
			& \subseteq J'_{2x} \cap J_3 \cap J_1 & \text{by } & J'_{3z} \cap J_1 \subseteq J_1 \\
			& \subseteq J'_{2x} \cap J_1 && \\
			& \subseteq J'_1 \cap J_2 & \text{by } & \text{\hyp{0'}} \hyp{2'} \\
			& \subseteq J'_1 \cap J_3 & \text{by } & J_2 \subseteq J_3
		\end{align*}
		\begin{align*}
		J'_1 \cap J_3 & = J'_1 = J'_1 \cap J_2 & \text{by } & J'_1 \subseteq J_1 \subseteq J_2 \subseteq J_3 \\
			& = J'_{2x} \cap J_1 & \text{by } & \text{\hyp{0'}} \\
			& \subseteq J'_{2x} \cap J_3 & \text{by } & J_1 \subseteq J_3 \\
			& \subseteq J'_{3z} \cap J_2 & \text{by } & \text{\hyp{1x}} \\
			& \subseteq J'_{3z} \cap J_2 \cap J_1 & \text{by } & J'_1 \subseteq J_1 \\
			& \subseteq J'_{3z} \cap J_1 &&
		\end{align*}
	\item Let \(\triangle\) be \(\supseteq\).
		\begin{align*}
		J'_1 \cap J_3 & \subseteq J'_1 \cap J_2 & \text{by } & J_3 \supseteq J_2 \\
			& \subseteq J'_{2x} \cap J_1 & \text{by } & \text{\hyp{0'}} \\
			& \subseteq J'_{2x} \cap J_1 \cap J_3 & \text{by } & J'_1 \cap J_3 \subseteq J_3 \\
			& \subseteq J'_{2x} \cap J_3 && \\
			& \subseteq J'_{3z} \cap J_2 & \text{by } & \text{\hyp{1x}} \\
			& \subseteq J'_{3z} \cap J_1 & \text{by } & J_1 \supseteq J_2
		\end{align*}
		\begin{align*}
		J'_{3z} \cap J_1 & = J'_{3z} = J'_{3z} \cap J_2 & \text{by } & J_1 \supseteq J_2 \supseteq J_3 \supseteq J'_{3z} \\
			& = J'_{2x} \cap J_3 & \text{by } & \text{\hyp{1x}} \hyp{3} \\
			& \subseteq J'_{2x} \cap J_1 & \text{by } & J_1 \supseteq J_3 \\
			& \subseteq J'_1 \cap J_2 & \text{by } & \text{\hyp{0'}} \hyp{3'} \\
			& \subseteq J'_1 \cap J_2 \cap J_3 & \text{by } & J'_{3z} \subseteq J_3 \\
			& \subseteq J'_1 \cap J_3 &&
		\end{align*}
	\item Let \(\triangle\) be \(=\).
		We combine the proofs for two above cases and get both inclusions so \goal{0'} holds.
	\end{itemize}
\item[\goal{2}:] Let \(\sigma_1: V_1 \uplus \fvars{t_1} \to \values\) and \(\sigma_3: V_3 \to \values\).
	We use dummy values for variables in \nmm{\biguplus_\subbox{z \in Z} \fvars{t_{3z}}} to get the left side of this implication as an hypothesis (independant from these values):
	\[ \sigma_1 \uplus \sigma_3 \vdash R\mpar{s_1, s_3} \wedge g_1 \hyp{4} \]
	Hypothesis immediately broken down into
	\begin{gather*}
		\sigma_2: V_2 \to \values \quad s_2 \in S_2 \\
		\sigma_1 \uplus \sigma_2 \uplus \sigma_3 \vdash R_{12}\mpar{s_1, s_2} \wedge R_{23}\mpar{s_2, s_3} \hyp{4'} \\
		\sigma_1 \vdash g_1 \hyp{4"}
	\end{gather*}
	We can use immediately \hyp{0"} with valuation \(\sigma_1 \uplus \sigma_2\) to get \nmm{\nu_2: \biguplus_{x \in X} \fvars{t_{2x}} \to \values}.
	The left side of the implication is proved with the right side of \hyp{4'} and \hyp{4"}, so we get \(x \in X_{s2}\) and the hypotheses
	\begin{align}
		\sigma_1 \uplus \sigma_2 \uplus \nu_2 & \vdash \alpha_1 = \alpha_{s2x} \hyp{5a} \\
		\sigma_1 \uplus \sigma_2 \uplus \nu_2 & \vdash \bigwedge_\subbox{j \in J'_1 \cap J_{s2x}} \beta_{1j} = \beta_{s2xj} \hyp{5b} \\
		\sigma_2 \uplus \nu_2 & \vdash g_{s2x} \hyp{5c} \\
		\sigma_1 \uplus \sigma_2 \uplus \nu_2 & \vdash R\mpar{s'_1, s'_{s2x}}\psubst{\psi_1 \uplus \psi_{s2x}} \hyp{5d}
	\end{align}
	We can now use \hyp{1x'} with valuation \(\sigma_2 \uplus \nu_2 \uplus \sigma_3\) to get \nmm{\nu_3: \biguplus_\subbox{Z \in Z_{s2x}} \fvars{t_{3z}} \to \values}.
	The left side of the implication is proved with the left side of \hyp{4'} and \hyp{4"}, so we get \(z \in Z_{s2x}\) and the hypotheses
	\begin{align}
		\sigma_2 \uplus \nu_2 \uplus \sigma_3 \uplus \nu_3 & \vdash \alpha_{s2x} = \alpha_{3z} \hyp{6a} \\
		\sigma_2 \uplus \nu_2 \uplus \sigma_3 \uplus \nu_3 & \vdash \bigwedge_\subbox{j \in J'_{s2x} \cap J_{3z}} \beta_{s2xj} = \beta_{3zj} \hyp{6b} \\
		\sigma_3 \uplus \nu_3 & \vdash g_{3z} \hyp{6c} \\
		\sigma_2 \uplus \nu_2 \uplus \sigma_3 \uplus \nu_3 & \vdash R\mpar{s'_{s2x}, s'_{3z}}\psubst{\psi_{s2x} \uplus \psi_{3z}} \hyp{6d}
	\end{align}
	In order to progress on \goal{2} we give valuation \(\nu_3\) for the implicit \(\exists\), admit the left side of the implication and choose to prove the branch \(z\):
	\begin{align}
		\sigma_1 \uplus \sigma_3 \uplus \nu_3 & \vdash \alpha_1 = \alpha_{3z} \goal{2a} \\
		\sigma_1 \uplus \sigma_3 \uplus \nu_3 & \vdash \bigwedge_\subbox{j \in J'_{3z} \cap J_1} \beta_{1j} = \beta_{3zj} \goal{2b} \\
		\sigma_3 \uplus \nu_3 & \vdash g_{3z} \goal{2c} \\
		\sigma_1 \uplus \sigma_3 \uplus \nu_3 &\vdash R\mpar{s'_1, s'_{3z}}\psubst{\psi_1 \uplus \psi_{3z}} \goal{2d}
	\end{align}
	\goal{2a} is proved by transitivity using \hyp{5a} and \hyp{5a}.
	\goal{2b} is proved using \hyp{5b} and \hyp{6b}, but depending on \(\triangle\) the equality are obtained for the interesting values \(j\) using either \hyp{2} and \hyp{2'} or \hyp{3} and \hyp{3'}.
	\goal{2c} is exactly \hyp{6c}.
	We prove \goal{2d} with valuation \(\psi_{s2x}\psubst{\sigma_2 \uplus \nu_2}\) for the branch \(s_2\) using \hyp{5d} and \hyp{6d}:
	\begin{align*}
		& \sigma_1 \uplus \sigma_3 \uplus \nu_3 \vdash R\mpar{s'_1, s'_{3z}}\psubst{\psi_1 \uplus \psi_{3z}} \\
		\iff & \mpar{R_{12}\mpar{s_1, s_2} \wedge R_{23}\mpar{s_2, s_3}}\psubst{\psi_1 \uplus \psi_{3z}}\psubst{\psi_{s2x}\psubst{\sigma_2 \uplus \nu_2}}\psubst{\sigma_1 \uplus \sigma_3 \uplus \nu_3} \\
		\iff & R_{12}\mpar{s_1, s_2}\psubst{\psi_1}\psubst{\psi_{s2x}\psubst{\sigma_2 \uplus \nu_2}}\psubst{\sigma_1} \wedge R_{23}\mpar{s_2, s_3}\psubst{\psi_{3z}}\psubst{\psi_{s2x}\psubst{\sigma_2 \uplus \nu_2}}\psubst{\sigma_3 \uplus \nu_3} \\
		\iff & R_{12}\mpar{s_1, s_2}\psubst{\psi_1 \uplus \psi_{s2x}}\psubst{\sigma_2 \uplus \nu_2 \uplus \sigma_1} \wedge R_{23}\mpar{s_2, s_3}\psubst{\psi_{3z} \uplus \psi_{s2x}}\psubst{\sigma_2 \uplus \nu_2 \uplus \sigma_3 \uplus \nu_3} \\
		\iff & \mpar{\sigma_2 \uplus \nu_2 \uplus \sigma_1 \vdash R_{12}\mpar{s_1, s_2}\psubst{\psi_1 \uplus \psi_{s2x}}} \wedge \mpar{\sigma_2 \uplus \nu_2 \uplus \sigma_3 \uplus \nu_3 \vdash R_{23}\mpar{s_2, s_3}\psubst{\psi_{3z} \uplus \psi_{s2x}}}
	\end{align*}
\item[\goal{1}:] Let \(\sigma_1: V_1 \to \values\) and \nmm{\sigma_3: V_3 \uplus \biguplus_\subbox{t_3 \in \fOT{s_3}} \fvars{t_3} \to \values}.
	We have the hypotheses on \(R_{12}\) and \(R_{23}\) for the states \(s_1\) and \(s_3\), then the current goal:
	\begin{gather}
		\forall s_2 \in S_2, V_1 \uplus V_2 \uplus \biguplus_\subbox{t_2 \in \fOT{s_2}} \fvars{t_2} \vdash R_{12}\mpar{s_1, s_2} \wedge \bigvee_\subbox{t_2 \in \fOT{s_2}} \fguard{t_2} \implies \bigvee_\subbox{t_1 \in \fOT{s_1}} \fguard{t_1} \hyp{7} \\
		\forall s_2 \in S_2, V_2 \uplus V_3 \uplus \biguplus_\subbox{t_3 \in \fOT{s_3}} \fvars{t_3} \vdash R_{23}\mpar{s_2, s_3} \wedge \bigvee_\subbox{t_3 \in \fOT{s_3}} \fguard{t_3} \implies \bigvee_\subbox{t_2 \in \fOT{s_2}} \fguard{t_2} \hyp{8} \\
		\sigma_1 \uplus \sigma_3 \vdash R\mpar{s_1, s_3} \wedge \bigvee_\subbox{t_3 \in \fOT{s_3}} \fguard{t_3} \implies \bigvee_\subbox{t_1 \in \fOT{s_1}} \fguard{t_1} \goal{1'}
	\end{gather}
	We use dummy values for variables in \nmm{\biguplus_\subbox{t_1 \in \fOT{s_1}} \fvars{t_1}} to get the left side of this implication as an hypothesis (independant from these values):
	\[ \sigma_1 \uplus \sigma_3 \vdash R\mpar{s_1, s_3} \wedge \bigvee_\subbox{t_3 \in \fOT{s_3}} \fguard{t_3} \hyp{9} \]
	Hypothesis that we immediately break down into:
	\begin{gather*}
		\sigma_2: V_2 \to \values \quad s_2 \in S_2 \\
		\sigma_1 \uplus \sigma_2 \uplus \sigma_3 \vdash R_{12}\mpar{s_1, s_2} \wedge R_{23}\mpar{s_2, s_3} \hyp{9'} \\
		\sigma_1 \uplus \sigma_3 \vdash \bigvee_\subbox{t_3 \in \fOT{s_3}} \fguard{t_3} \hyp{9"}
	\end{gather*}
	We can use \hyp{8} with state \(s_2\) and valuation \(\sigma_2 \uplus \sigma_3\) to get \nmm{\nu_2: \biguplus_\subbox{t_2 \in \fOT{s_2}} \fvars{t_2} \to \values}.
	The left side of the implication is proved with the right side of \hyp{9'} and \hyp{9"}, so we get
	\[ \sigma_3 \uplus \nu_2 \vdash \bigvee_\subbox{t_2 \in \fOT{s_2}} \fguard{t_2} \hyp{8'} \]
	We can now use \hyp{7} with state \(s_2\) and valuation \(\sigma_1 \uplus \sigma_2 \uplus \nu_2\) to get \nmm{\nu_1: \biguplus_\subbox{t_1 \in \fOT{s_1}} \fvars{t_1} \to \values}.
	The left side of the implication is proved with the left side of \hyp{9'} and \hyp{8'}, so we get
	\[ \sigma_1 \uplus \nu_1 \vdash \bigvee_\subbox{t_1 \in \fOT{s_1}} \fguard{t_1} \hyp{7'} \]
	Finally we can prove \goal{1'} with valuation \(\nu_1\) for the implicit \(\exists\) by admitting the left side of the implication and using \hyp{7'} for the right side.
\end{proof}


\section{Proof or properties for open automata refinement}

\subsection{Context refinement}\label{apx:cr}
Let \(A_1 \defobject \OAg[1]\), \(A_2 \defobject \OAg[2]\) and \(A_3 \defobject \OAg[3]\) and \(\mpar{R_{12}, H_{12}}\) be a hole-tracking simulation of \(A_1\) by \(A_2\).
Let \(k \in H_{12}\), \(A_{13} \defobject A_1\subst{A_3}{k}\), \(A_{23} \defobject A_2\subst{A_3}{k}\) and \(\reach{A_{13}}\) be a witness that \(A_{13}\) is non-locking.
We pose \(\OAg[13] \defobject A_{13}\) and \(\OAg[23] \defobject A_{23}\).
\begin{gather*}
	R \defobject \mpar{\mpar{s_1, s_{31}}, \mpar{s_2, s_{32}}} \mapsto \choice{R_{12}\mpar{s_1, s_2} \wedge \reach{A_{13}}\mpar{s_1, s_{31}} \wedge \bigwedge_\subbox{v_3 \in V_3} v_{31} = v_{32} & \text{if } s_{31} = s_{32} \\ \bot & \text{otherwise}} \\
	H \defobject J_3 \uplus H_{12} \setminus \mbrc{k}
\end{gather*}
Where \(y_{3x}\) is the renaming of \(y_3\) from \(A_3\) in \(A_{x3}\).
We want to prove that \(\mpar{R, H}\) is a witness of \(\wrel{A_{13}}{A_{23}}{H}\).
\begin{proof}
\item[1)] \(H = J_3 \uplus H_{12} \setminus \mbrc{k} \subseteq J_3 \uplus \mpar{J_1 \cap J_2} \setminus \mbrc{k} = \mpar{J_3 \uplus J_1 \setminus \mbrc{k}} \cap \mpar{J_3 \uplus J_2 \setminus \mbrc{k}} = J_{13} \cap J_{23}\)
\item[2)] We use the fact that \(R_{12}\) relates inital configurations, initial configuration is reachable, and \(\sigma_{031}\) and \(\sigma_{032}\) are renaming of \(\sigma_{03}\):
	\begin{align*}
		& \sigma_{013} \uplus \sigma_{023} \vdash R\mpar{s_{013}, s_{023}} \\
		\iff & R\mpar{\mpar{s_{01}, s_{031}}, \mpar{s_{02}, s_{031}}}\psubst{\sigma_{01} \uplus \sigma_{02} \uplus \sigma_{031} \uplus \sigma_{032}} \\
		\iff & R_{12}\mpar{s_{01}, s_{02}}\psubst{\sigma_{01} \uplus \sigma_{02}} \wedge \reach{A_{13}}\mpar{s_{013}}\psubst{\sigma_{013}} \wedge \mpar{\bigwedge_{v_3 \in V_3} v_{31} = v_{32}}\psubst{\sigma_{031} \uplus \sigma_{032}} \\
		\iff & \mpar{\sigma_{01} \uplus \sigma_{02} \vdash R_{12}\mpar{s_{01}, s_{02}}} \wedge \mpar{\sigma_{013} \vdash \reach{A_{13}}\mpar{s_{013}}} \wedge \bigwedge_\subbox{v_3 \in V_3} \sigma_{031}\mpar{v_{31}} = \sigma_{032}\mpar{v_{32}} \\
		\iff & \mpar{\sigma_{01} \uplus \sigma_{02} \vdash R_{12}\mpar{s_{01}, s_{02}}} \wedge \mpar{\sigma_{013} \vdash \reach{A_{13}}\mpar{s_{013}}} \wedge \top
	\end{align*}
\item[3)] Let \(s_{13} \defobject \mpar{s_1, s_{31}} \in S_{13}\) and \(s_{23} \defobject \mpar{s_2, s_{32}} \in S_{23}\).
	We want to prove both
	\begin{align*}
		& \bigsymb{\forall} t_{13} \defobject \OTx{13}{}{13}{13} \in \fOT{s_{13}}, \bigsymb{\exists} \mpar{t_{23x} \defobject \OTx{23}{x}{23x}{23x} \in \fOT{s_{23}}}^{x \in X}, \\
		& \quad \mpar{\forall x \in X, J'_{23x} \cap H = J'_{13} \cap H} \\[-10pt]
		& \nwedge V_{13} \uplus V_{23} \uplus \fvars{t_{13}} \vdash R\mpar{s_{13}, s_{23}} \wedge g_{13} \implies \operatorname*{\bigsymb{\bigvee}}_{x \in X} \mpar{\everymath{\displaystyle}\begin{array}{l}
			\alpha_{13} = \alpha_{23x} \wedge \bigwedge_\subbox{j \in J'_{23x} \cap H} \beta_{13j} = \beta_{23xj} \\[12pt]
			\nwedge g_{23x} \wedge R\mpar{s'_{13}, s'_{23x}}\psubst{\psi_{13} \uplus \psi_{23x}}
		\end{array}} \goal{0}
	\end{align*}
	\[ V_{13} \uplus V_{23} \uplus \biguplus_\subbox{t_{23} \in \fOT{s_{23}}} \fvars{t_{23}} \vdash R\mpar{s_{13}, s_{23}} \wedge \bigvee_\subbox{t_{23} \in \fOT{s_{23}}} \fguard{t_{23}} \implies \bigvee_\subbox{t_{13} \in \fOT{s_{13}}} \fguard{t_{13}} \goal{1} \]
	Let \nmm{t_{13} \defobject \OTx{13}{}{13}{13} \in \fOT{s_{13}}}, we use the fact that \(R_{12}\) is a simulation from the states \(\mpar{s_1, s_2}\) to get the hypothesis:
	\begin{align*}
		& \bigsymb{\forall} t_1 \defobject \OTx{1}{}{1}{1} \in \fOT{s_1}, \bigsymb{\exists} \mpar{t_{2x} \defobject \OTx{2}{x}{2x}{2x} \in \fOT{s_2}}^{x \in X}, \\
		& \quad \mpar{\forall x \in X, J'_{2x} \cap H_{12} = J'_1 \cap H_{12}} \\[-10pt]
		& \nwedge V_1 \uplus V_2 \uplus \fvars{t_1} \vdash R_{12}\mpar{s_1, s_2} \wedge g_1 \implies \operatorname*{\bigsymb{\bigvee}}_{x \in X} \mpar{\everymath{\displaystyle}\begin{array}{l}
			\alpha_1 = \alpha_{2x} \wedge \bigwedge_\subbox{j \in J'_{2x} \cap H_{12}} \beta_{1j} = \beta_{2xj} \\[12pt]
			\nwedge g_{2x} \wedge R_{12}\mpar{s'_1, s'_{2x}}\psubst{\psi_1 \uplus \psi_{2x}}
		\end{array}} \hyp{1}
	\end{align*}
	Transition \(t_{13}\) is obtained in \(A_{13}\) by composition of two transitions \nmm{t_1 \defobject \OTx{1}{}{1}{1} \in \fOT{s_1}} and \nmm{t_{31} \defobject \OTx{31}{}{31}{31} \in \fOT{s_{31}}} if \(k \in J'_1\) or by a the first transition and the state \(s_{31}\) if \(k \notin J'_1\).
	We use \hyp{1} with the transition \(t_1\) to get
	\begin{gather*}
		\mpar{t_{2x} \defobject \OTx{2}{x}{2x}{2x} \in \fOT{s_2}}^{x \in X} \\
		\forall x \in X, J'_{2x} \cap H_{12} = J'_1 \cap H_{12} \hyp{1'} \\
		V_1 \uplus V_2 \uplus \fvars{t_1} \vdash R_{12}\mpar{s_1, s_2} \wedge g_1 \implies \operatorname*{\bigsymb{\bigvee}}_{x \in X} \mpar{\everymath{\displaystyle}\begin{array}{l}
			\alpha_1 = \alpha_{2x} \wedge \bigwedge_\subbox{j \in J'_{2x} \cap H_{12}} \beta_{1j} = \beta_{2xj} \\[12pt]
			\nwedge g_{2x} \wedge R_{12}\mpar{s'_1, s'_{2x}}\psubst{\psi_1 \uplus \psi_{2x}}
		\end{array}} \hyp{1"}
	\end{gather*}
	To build the family of transitions we need to know if \(k \in J'_{2x}\) for a every \(x \in X\).
	We have \(k \in H_{12}\) so by \hyp{1'} \(k \in J'_{2x} \iff k \in J'_1\).
	If \(s_{31} = s_{32}\) we can copy \(t_{31}\) into \nmm{t_{32} = \OTx{32}{}{32}{32}} to avoid conflicts.
	We also name \(t_3\) the original transition from \(T_3\).
	We build the familly of transition generated by the family \(t_{2x}^{x \in X}\):
	\[ t_{23x}^{x \in X} \defobject x \mapsto \choice{
		\OT{\mpar{s_2, s_{32}}}{\mpar{s'_{2x}, s'_{32}}}{\alpha_{2x}}{\beta_{2xj}^{j \in J'_{2x} \setminus \mbrc{k}} \uplus \beta_{3j}^{j \in J'_{32}}}{g_{2x} \wedge g_{32} \wedge \alpha_{32} = \beta_{2xk}}{\psi_{2x} \uplus \psi_{32}} & \text{if } k \in J'_{2x} \\
		\OT{\mpar{s_2, s_{32}}}{\mpar{s'_{2x}, s_{32}}}{\alpha_{2x}}{\beta_{2xj}^{j \in J'_2}}{g_2}{\psi_2} & \text{otherwise}
	} \]
	Otherwise \(s_{31} \neq s_{32}\) and we can give the empty family, it won't cause any issue.
	We will prove \goal{0} for the family \(t_{23x}^{x \in X}\), the new goals are
	\begin{gather}
		\forall x \in X, J'_{23x} \cap H = J'_{13} \cap H \goal{0'} \\
		V_{13} \uplus V_{23} \uplus \fvars{t_{13}} \vdash R\mpar{s_{13}, s_{23}} \wedge g_{13} \implies \operatorname*{\bigsymb{\bigvee}}_{x \in X} \mpar{\everymath{\displaystyle}\begin{array}{l}
			\alpha_{13} = \alpha_{23x} \wedge \bigwedge_\subbox{j \in J'_{13} \cap H} \beta_{13j} = \beta_{23xj} \\[12pt]
			\nwedge g_{23x} \wedge R\mpar{s'_{13}, s'_{23x}}\psubst{\psi_{13} \uplus \psi_{23x}}
		\end{array}} \goal{2}
	\end{gather}
\item[\goal{0'}:] Let \(x \in X\), by case on \(k \in J'_1\).
	\begin{align*}
		\text{If } k \in J'_1: && J'_{23x} \cap H & = \mpar{J'_{32} \uplus J'_{2x} \setminus \mbrc{k}} \cap \mpar{J_3 \uplus H_{12} \setminus \mbrc{k}} \\
		&&& = \mpar{J'_{31} \cap J_3} \uplus \mpar{J'_{2x} \cap H_{12}} \setminus \mbrc{k} & \text{by } J'_{32} = J'_{31} \\
		&&& = \mpar{J'_{31} \cap J_3} \uplus \mpar{J'_1 \cap H_{12}} \setminus \mbrc{k} & \text{by \hyp{1'}} \\
		&&& = \mpar{J'_{31} \uplus J'_1 \setminus \mbrc{k}} \cap \mpar{J_3 \uplus H_{12} \setminus \mbrc{k}} \\
		&&& = J'_{13} \cap H \\
		\text{If } k \notin J'_1: && J'_{23x} \cap H & = \mpar{J'_{2x} \setminus \mbrc{k}} \cap \mpar{J_3 \uplus H_{12} \setminus \mbrc{k}} \\
		&&& = \mpar{J'_{2x} \cap H_{12}} \setminus \mbrc{k} \\
		&&& = \mpar{J'_1 \cap H_{12}} \setminus \mbrc{k} & \text{by \hyp{1'}} \\
		&&& = \mpar{J'_1 \setminus \mbrc{k}} \cap \mpar{J_3 \uplus H_{12} \setminus \mbrc{k}} \\
		&&& = J'_{13} \cap H
	\end{align*}
\item[\goal{2}:] Let \(\sigma: V_{13} \uplus V_{23} \uplus \fvars{t_{13}} \to \values\).
	We use dummy values for the variables in \nmm{\biguplus_{x \in X} \fvars{t_{23x}}} to get the left side of the implication as an hypothesis (independant from these values):
	\[ \sigma \vdash R\mpar{s_{13}, s_{23}} \wedge g_{13} \hyp{2} \]
	If \(s_{31} \neq s_{32}\) then this hypothesis is \(\sigma \vdash \bot\), the empty family works and the proof is finished.
	We can assume \(s_{31} = s_{32}\) and break \hyp{2} into
	\begin{gather}
		\sigma \vdash R_{12}\mpar{s_1, s_2} \hyp{2a} \\
		\sigma \vdash \reach{A_{13}}\mpar{s_{13}} \hyp{2b} \\
		\sigma \vdash \bigwedge_\subbox{v_3 \in V_3} v_{31} = v_{32} \hyp{2c} \\
		\sigma \vdash g_{13} \hyp{2d}
	\end{gather}
	We use \hyp{1"} with valuation \(\sigma\) on variables \(V_1 \uplus V_2 \uplus \fvars{t_1}\), which is possible because the composition of transitions doesn't remove any variable (the simplification steps do, they give bisimilar automata).
	Then we get the values of the implicit \(\exists\) in \nmm{\nu: \biguplus_{x \in X} \fvars{t_{2x}}}.
	The left side of the implication is proved with \hyp{2a} and \hyp{2d}.
	So we get a \(x \in X\) such that
	\begin{gather}
		\sigma \uplus \nu \vdash \alpha_1 = \alpha_{2x} \hyp{3a} \\
		\sigma \uplus \nu \vdash \bigwedge_\subbox{j \in J'_{2x} \cap H} \beta_{1j} = \beta_{2xj} \hyp{3b} \\
		\sigma \uplus \nu \vdash g_{2x} \hyp{3c} \\
		\sigma \uplus \nu \vdash R_{12}\mpar{s'_1, s'_{2x}}\psubst{\psi_1 \uplus \psi_{2x}} \hyp{3d}
	\end{gather}
	If \(k \in J'_1\), we extend the valuation \(\nu\) to cover the variables of \(t_3\) in \(t_{23x}\) using their value in \(t_{13}\), \(\nu' \defobject \nu \uplus \mset{v_{32} \mapsto \sigma\mpar{v_{31}}}{v_3 \in \fvars{t_3}}\).
	Otherwise \(\nu' \defobject \nu\).
	We progress on \goal{2} by giving the valuation \(\sigma \uplus \nu'\) to the implicit \(\exists\), admitting the left side of the implication and choosing to prove the branch \(x\) of the disjunction:
	\begin{gather}
		\sigma \uplus \nu' \vdash \alpha_{13} = \alpha_{23x} \goal{3a} \\
		\sigma \uplus \nu' \vdash \bigwedge_\subbox{j \in J'_{23x} \cap H} \beta_{13j} = \beta_{23xj} \goal{3b} \\
		\sigma \uplus \nu' \vdash g_{23x} \goal{3c} \\
		\sigma \uplus \nu' \vdash R\mpar{s'_{13}, s'_{23x}}\psubst{\psi_{13} \uplus \psi_{23x}} \goal{3d}
	\end{gather}
	Because composition doesn't change the produced actions nor its variables, \goal{3a} is proved by \hyp{3a}.
	\goal{3b} holds because the valuations for the actions of the holes in \(J_3\) coincide by definition of \(\nu'\) and by \hyp{3b} for the others.
	For \goal{3c} we have two cases, if \(k \notin J'_1\) then \(g_{23x} \equiv g_{2x}\) and \hyp{3c} is sufficient, if \(k \in J'_1\) then \(g_{23x} \equiv g_{2x} \wedge g_3 \wedge \alpha_3 = \beta_{2xk}\) so we also need \hyp{2d} (\(g_{13} \equiv g_1 \wedge g_3 \wedge \alpha_3 = \beta_{1k}\)) and \hyp{3b} for the value \(k\) (\(\beta_{1k} = \beta_{2xk}\)).
	\goal{3d} is proved using \hyp{3d}, \hyp{2b} with the preservation of reachability across transitions (+ \hyp{2d}), \hyp{2c} and the fact that \(A_3\) performs the same transition with same values in both composed automata:
	\begin{align*}
		& \sigma \uplus \nu' \vdash R\mpar{s'_{13}, s'_{23x}}\psubst{\psi_{13} \uplus \psi_{23x}} \\
		\iff & \sigma \uplus \nu' \vdash \mpar{R_{12}\mpar{s'_1, s'_{2x}} \wedge \reach{A_{13}}\mpar{s'_{13}} \wedge \bigwedge_\subbox{v_3 \in V_3} v_{31} = v_{32}}\psubst{\psi_1 \uplus \psi_{31} \uplus \psi_{2x} \uplus \psi_{32}} \\
		\iff & \sigma \uplus \nu' \vdash R_{12}\mpar{s'_1, s'_{2x}}\psubst{\psi_1 \uplus \psi_{2x}} \wedge \reach{A_{13}}\mpar{s'_{13}}\psubst{\psi_{13}} \wedge \bigwedge_\subbox{v_3 \in V_3} \psi_{31}\mpar{v_{31}} = \psi_{32}\mpar{v_{32}} \\
		\impliedby & \sigma \uplus \nu' \vdash R_{12}\mpar{s'_1, s'_{2x}}\psubst{\psi_1 \uplus \psi_{2x}} \wedge \reach{A_{13}}\mpar{s'_{13}}\psubst{\psi_{13}} \wedge \bigwedge_\subbox{v_3 \in V_3 \uplus \fvars{t_3}} v_{31} = v_{32} \\
		\iff & \mpar{\sigma \uplus \nu \vdash R_{12}\mpar{s'_1, s'_{2x}}\psubst{\psi_1 \uplus \psi_{2x}}} \wedge \mpar{\sigma \vdash \reach{A_{13}}\mpar{s'_{13}}\psubst{\psi_{13}}} \wedge \mpar{\sigma \vdash \bigwedge_\subbox{v_3 \in V_3} v_{31} = v_{32}}
	\end{align*}
\item[\goal{1}:] Let \nmm{\sigma: V_{13} \uplus V_{23} \uplus \biguplus_\subbox{t_{23} \in \fOT{s_{23}}} \fvars{t_{23}} \to \values}.
	We use dummy values for the variables in \nmm{\biguplus_\subbox{t_{13} \in \fOT{s_{13}}} \fvars{t_{13}}} to get the left side of the implication (independant from these values), immediately broken into
	\begin{gather}
		\sigma \vdash R_{12}\mpar{s_1, s_2} \hyp{4a} \\
		\sigma \vdash \reach{A_{13}}\mpar{s_{13}} \hyp{4b} \\
		\sigma \vdash \bigvee_\subbox{t_{23} \in \fOT{s_{23}}} \fguard{t_{23}} \hyp{4c}
	\end{gather}
	\(R_{12}\) is a simulation and \(\reach{A_{13}}\) is a witness of the non-locking composition so we have the hypotheses:
	\begin{gather}
		V_1 \uplus V_2 \uplus \biguplus_\subbox{t_2 \in \fOT{s_2}} \fvars{t_2} \vdash R_{12}\mpar{s_1, s_2} \wedge \bigvee_\subbox{t_2 \in \fOT{s_2}} \fguard{t_2} \implies \bigvee_\subbox{t_1 \in \fOT{s_1}} \fguard{t_1} \hyp{5} \\
		V_{13} \uplus \biguplus_\subbox{t_1 \in \fOT{s_1}} \fvars{t_1} \vdash \reach{A_{13}}\mpar{s_{13}} \wedge \bigvee_\subbox{t_1 \in \fOT{s_1}} \fguard{t_1} \implies \bigvee_\subbox{t_{13} \in \fOT{s_{13}}} \fguard{t_{13}} \hyp{6}
	\end{gather}
	We use \hyp{5} with valuation \(\sigma\) projected on the variables from generating transitions in \(A_2\) to get \nmm{\nu: \biguplus_\subbox{t_1 \in \fOT{s_1}} \fvars{t_1} \to \values}.
	The left side of the implication is proved using \hyp{4a} and \hyp{4c} because the guards from transitions of \(A_2\) are still in transitions of \(A_{23}\), the right side gives:
	\[ \sigma \uplus \nu \vdash \bigvee_\subbox{t_1 \in \fOT{s_1}} \fguard{t_1} \hyp{5'} \]
	We then use \hyp{6} with valuation \(\sigma \uplus \nu\) to get \nmm{\nu': \biguplus_\subbox{t_{13} \in \fOT{s_{13}}} \fvars{t_{13}}}.
	The left side of the implication is proved using \hyp{4b} and \hyp{5'}, the right side gives:
	\[ \sigma \uplus \nu' \vdash \bigvee_\subbox{t_{13} \in \fOT{s_{13}}} \fguard{t_{13}} \hyp{6'} \]
	\goal{1} is proved for valuation \(\nu'\) by admitting the left side and using \hyp{6'}.
\end{proof}

\subsection{Congruence for composition}\label{apx:cong}
Let \(A_1 \defobject \OAg[1]\), \(A_2 \defobject \OAg[2]\) and \(A_3 \defobject \OAg[3]\) be three open automata.
Let \(k \in J_1\), \(\mpar{R_{23}, H_{23}}\) be a hole-tracking simulation of \(A_2\) by \(A_3\) and \(\reach{A_1\subst{A_2}{k}}\) be a witness that \(A_1\subst{A_2}{k}\) is non-locking.
We pose \(\OAg[21] \defobject A_{21}\) and \(\OAg[31] \defobject A_{31}\).
\begin{gather*}
	R \defobject \mpar{\mpar{s_{12}, s_2}, \mpar{s_{13}, s_3}} \mapsto \choice{R_{23}\mpar{s_2, s_3} \wedge \reach{A_{21}}\mpar{s_{12}, s_2} \wedge \bigwedge_\subbox{v_1 \in V_1} v_{12} = v_{13} & \text{if } s_{12} = s_{13} \\ \bot & \text{otherwise}} \\
	H \defobject H_{23} \uplus J_1 \setminus \mbrc{k}
\end{gather*}
Where \(y_{1x}\) is the renaming of \(y_1\) from \(A_1\) in \(A_{x1}\).
We want to prove that \(\mpar{R, H}\) is a witness of \(\wrel{A_{21}}{A_{31}}{H}\).
\begin{proof}
\item[1)] \(H = H_{23} \uplus J_1 \setminus \mbrc{k} \subseteq \mpar{J_2 \cap J_3} \uplus J_1 \setminus \mbrc{k} = \mpar{J_2 \uplus J_1 \setminus \mbrc{k}} \cap \mpar{J_3 \uplus J_1 \setminus \mbrc{k}} = J_{21} \cap J_{31}\)
\item[2)] We use the fact that \(R_{23}\) relates inital configurations, initial configuration is reachable, and \(\sigma_{012}\) and \(\sigma_{013}\) are renaming of \(\sigma_{01}\):
	\begin{align*}
		& \sigma_{021} \uplus \sigma_{031} \vdash R\mpar{s_{021}, s_{031}} \\
		\iff & R\mpar{\mpar{s_{012}, s_{02}}, \mpar{s_{013}, s_{03}}}\psubst{\sigma_{012} \uplus \sigma_{02} \uplus \sigma_{013} \uplus \sigma_{03}} \\
		\iff & R_{23}\mpar{s_{02}, s_{03}}\psubst{\sigma_{02} \uplus \sigma_{03}} \wedge \reach{A_{21}}\mpar{s_{021}}\psubst{\sigma_{021}} \wedge \mpar{\bigwedge_{v_1 \in V_1} v_{12} = v_{13}}\psubst{\sigma_{012} \uplus \sigma_{013}} \\
		\iff & \mpar{\sigma_{02} \uplus \sigma_{03} \vdash R_{23}\mpar{s_{02}, s_{03}}} \wedge \mpar{\sigma_{021} \vdash \reach{A_{21}}\mpar{s_{021}}} \wedge \bigwedge_\subbox{v_1 \in V_1} \sigma_{012}\mpar{v_{12}} = \sigma_{013}\mpar{v_{13}} \\
		\iff & \mpar{\sigma_{02} \uplus \sigma_{03} \vdash R_{23}\mpar{s_{02}, s_{03}}} \wedge \mpar{\sigma_{021} \vdash \reach{A_{21}}\mpar{s_{021}}} \wedge \top
	\end{align*}
\item[3)] Let \(s_{21} \defobject \mpar{s_{12}, s_2} \in S_{21}\) and \(s_{31} \defobject \mpar{s_{13}, s_3} \in S_{31}\).
	We want to prove both
	\begin{align*}
		& \bigsymb{\forall} t_{21} \defobject \OTx{21}{}{21}{21} \in \fOT{s_{21}}, \bigsymb{\exists} \mpar{t_{31x} \defobject \OTx{31}{x}{31x}{31x} \in \fOT{s_{31}}}^{x \in X}, \\
		& \quad \mpar{\forall x \in X, J'_{31x} \cap H = J'_{21} \cap H} \\[-10pt]
		& \nwedge V_{21} \uplus V_{31} \uplus \fvars{t_{21}} \vdash R\mpar{s_{21}, s_{31}} \wedge g_{21} \implies \operatorname*{\bigsymb{\bigvee}}_{x \in X} \mpar{\everymath{\displaystyle}\begin{array}{l}
			\alpha_{21} = \alpha_{31x} \wedge \bigwedge_\subbox{j \in J'_{31x} \cap H} \beta_{21j} = \beta_{31xj} \\[12pt]
			\nwedge g_{31x} \wedge R\mpar{s'_{21}, s'_{31x}}\psubst{\psi_{21} \uplus \psi_{31x}}
		\end{array}} \goal{0}
	\end{align*}
	\[ V_{21} \uplus V_{31} \uplus \biguplus_\subbox{t_{31} \in \fOT{s_{31}}} \fvars{t_{31}} \vdash R\mpar{s_{21}, s_{31}} \wedge \bigvee_\subbox{t_{31} \in \fOT{s_{31}}} \fguard{t_{31}} \implies \bigvee_\subbox{t_{21} \in \fOT{s_{21}}} \fguard{t_{21}} \goal{1} \]
	Let \nmm{t_{21} \defobject \OTx{21}{}{21}{21} \in \fOT{s_{21}}}, it is obtained by composition of \nmm{t_{12} \defobject \OTx{12}{}{12}{12} \in \fOT{s_{12}}} and \nmm{t_2 \defobject \OTx{2}{}{2}{2} \in \fOT{s_2}} if \(k \in J'_{12}\) or by a the first transition and the state \(s_2\) if \(k \notin J'_{12}\).
	If \(k \in J'_{12}\), we use the fact that \(R_{23}\) is a simulation from the states \(\mpar{s_2, s_3}\) to get the hypothesis:
	\begin{align*}
		& \bigsymb{\forall} t_2 \defobject \OTx{2}{}{2}{2} \in \fOT{s_2}, \bigsymb{\exists} \mpar{t_{3x} \defobject \OTx{3}{x}{3x}{3x} \in \fOT{s_3}}^{x \in X}, \\
		& \quad \mpar{\forall x \in X, J'_{3x} \cap H_{23} = J'_2 \cap H_{23}} \\[-10pt]
		& \nwedge V_2 \uplus V_3 \uplus \fvars{t_2} \vdash R_{23}\mpar{s_2, s_3} \wedge g_2 \implies \operatorname*{\bigsymb{\bigvee}}_{x \in X} \mpar{\everymath{\displaystyle}\begin{array}{l}
			\alpha_2 = \alpha_{3x} \wedge \bigwedge_\subbox{j \in J'_{3x} \cap H_{23}} \beta_{2j} = \beta_{3xj} \\[12pt]
			\nwedge g_{3x} \wedge R_{23}\mpar{s'_2, s'_{3x}}\psubst{\psi_2 \uplus \psi_{3x}}
		\end{array}} \hyp{1}
	\end{align*}
	hypothesis that we use with the transition \(t_2\) to get
	\begin{gather*}
		\mpar{t_{3x} \defobject \OTx{3}{x}{3x}{3x} \in \fOT{s_3}}^{x \in X} \\
		\forall x \in X, J'_{3x} \cap H_{23} = J'_2 \cap H_{23} \hyp{1'} \\
		V_2 \uplus V_3 \uplus \fvars{t_2} \vdash R_{23}\mpar{s_2, s_3} \wedge g_2 \implies \operatorname*{\bigsymb{\bigvee}}_{x \in X} \mpar{\everymath{\displaystyle}\begin{array}{l}
			\alpha_2 = \alpha_{3x} \wedge \bigwedge_\subbox{j \in J'_{3x} \cap H_{23}} \beta_{2j} = \beta_{3xj} \\[12pt]
			\nwedge g_{3x} \wedge R_{23}\mpar{s'_2, s'_{3x}}\psubst{\psi_2 \uplus \psi_{3x}}
		\end{array}} \hyp{1"}
	\end{gather*}
	In the case \(s_{12} \neq s_{13}\) we can give the empty family, it won't cause any issue.
	Otherwise \(s_{12} = s_{13}\) so we can copy \(t_{12}\) into \nmm{t_{13} = \OTx{13}{}{13}{13}} to avoid conflicts.
	We also name \(t_1\) the original transition from \(T_1\).
	We build the familly of transition generated by \(t_{13}\):
	\[ t_{31x}^{x \in X} \defobject \choice{
		x \mapsto \OT{\mpar{s_{13}, s_3}}{\mpar{s'_{13}, s'_{3x}}}{\alpha_{13}}{\beta_{13j}^{j \in J'_{13x} \setminus \mbrc{k}} \uplus \beta_{3xj}^{j \in J'_{3x}}}{g_{13} \wedge g_{3x} \wedge \alpha_{3x} = \beta_{13k}}{\psi_{13} \uplus \psi_{3x}} & \text{if } k \in J'_{13} \\
		x \mapsto \OT{\mpar{s_{13}, s_3}}{\mpar{s'_{13}, s_3}}{\alpha_{13}}{\beta_{13j}^{j \in J'_{13}}}{g_{13}}{\psi_{13}} & \text{otherwise}
	} \]
	We will prove \goal{0} for the family \(t_{31x}^{x \in X}\), the new goals are
	\begin{gather}
		\forall x \in X, J'_{31x} \cap H = J'_{21} \cap H \goal{0'} \\
		V_{21} \uplus V_{31} \uplus \fvars{t_{21}} \vdash R\mpar{s_{21}, s_{31}} \wedge g_{21} \implies \operatorname*{\bigsymb{\bigvee}}_{x \in X} \mpar{\everymath{\displaystyle}\begin{array}{l}
			\alpha_{21} = \alpha_{31x} \wedge \bigwedge_\subbox{j \in J'_{31x} \cap H} \beta_{21j} = \beta_{31xj} \\[12pt]
			\nwedge g_{31x} \wedge R\mpar{s'_{21}, s'_{31x}}\psubst{\psi_{21} \uplus \psi_{31x}}
		\end{array}} \goal{2}
	\end{gather}
\item[\goal{0'}:] Let \(x \in X\), by case on \(k \in J'_{12}\).
	\begin{align*}
		\text{If } k \in J'_{12}: && J'_{31x} \cap H & = \mpar{J'_{3x} \uplus J'_{13} \setminus \mbrc{k}} \cap \mpar{H_{23} \uplus J_1 \setminus \mbrc{k}} \\
		&&& = \mpar{J'_{3x} \cap H_{23}} \uplus \mpar{J'_{12} \cap J_1} \setminus \mbrc{k} & \text{by } J'_{13} = J'_{12} \\
		&&& = \mpar{J'_2 \cap H_{23}} \uplus \mpar{J'_{12} \cap J_1} \setminus \mbrc{k} & \text{by \hyp{1'}} \\
		&&& = \mpar{J'_2 \uplus J'_{12} \setminus \mbrc{k}} \cap \mpar{H_{23} \uplus J_1 \setminus \mbrc{k}} \\
		&&& = J'_{32} \cap H \\
		\text{If } k \notin J'_{12}: && J'_{31x} \cap H & = \mpar{J'_{13} \setminus \mbrc{k}} \cap \mpar{H_{23} \uplus J_1 \setminus \mbrc{k}} \\
		&&& = \mpar{J'_{12} \setminus \mbrc{k}} \cap \mpar{H_{23} \uplus J_1 \setminus \mbrc{k}} \\
		&&& = J'_{32} \cap H
	\end{align*}
\item[\goal{2}:] Let \(\sigma: V_{21} \uplus V_{31} \uplus \fvars{t_{21}} \to \values\).
	We use dummy values for the variables in \nmm{\biguplus_{x \in X} \fvars{t_{31x}}} to get the left side of the implication as an hypothesis (independant from these values):
	\[ \sigma \vdash R\mpar{s_{21}, s_{31}} \wedge g_{21} \hyp{2} \]
	If \(s_{12} \neq s_{13}\) then this hypothesis is \(\sigma \vdash \bot\), the empty family works and the proof is finished.
	We can assume \(s_{12} = s_{13}\) and break \hyp{2} into
	\begin{gather}
		\sigma \vdash R_{23}\mpar{s_2, s_3} \hyp{2a} \\
		\sigma \vdash \reach{A_{21}}\mpar{s_{21}} \hyp{2b} \\
		\sigma \vdash \bigwedge_\subbox{v_1 \in V_1} v_{12} = v_{13} \hyp{2c} \\
		\sigma \vdash g_{21} \hyp{2d}
	\end{gather}
	If \(k \in J'_{12}\), we use \hyp{1"} with valuation \(\sigma\) on variables \(V_2 \uplus V_3 \uplus \fvars{t_2}\), then we get the values of the implicit \(\exists\) in \nmm{\nu: \biguplus_{x \in X} \fvars{t_{3x}}}.
	The left side of the implication is proved with \hyp{2a} and \hyp{2d}.
	So we get a \(x \in X\) such that
	\begin{gather}
		\sigma \uplus \nu \vdash \alpha_2 = \alpha_{3x} \hyp{3a} \\
		\sigma \uplus \nu \vdash \bigwedge_\subbox{j \in J'_{3x} \cap H} \beta_{2j} = \beta_{3xj} \hyp{3b} \\
		\sigma \uplus \nu \vdash g_{3x} \hyp{3c} \\
		\sigma \uplus \nu \vdash R_{23}\mpar{s'_2, s'_{3x}}\psubst{\psi_2 \uplus \psi_{3x}} \hyp{3d}
	\end{gather}
	if \(k \notin J'_{12}\) we pose \(\nu \defobject \mbrc{}\) the empty valuation.
	We extend the valuation \(\nu\) to cover the variables of \(t_{13}\) in \(t_{31x}\) using their value in \(t_{12}\), \(\nu' \defobject \nu \uplus \mset{v_{13} \mapsto \sigma\mpar{v_{12}}}{v_1 \in \fvars{t_1}}\)
	We progress on \goal{2} by giving the valuation \(\sigma \uplus \nu'\) to the implicit \(\exists\), admitting the left side of the implication and choosing to prove the branch \(x\) of the disjunction:
	\begin{gather}
		\sigma \uplus \nu' \vdash \alpha_{21} = \alpha_{31x} \goal{3a} \\
		\sigma \uplus \nu' \vdash \bigwedge_\subbox{j \in J'_{31x} \cap H} \beta_{21j} = \beta_{31xj} \goal{3b} \\
		\sigma \uplus \nu' \vdash g_{31x} \goal{3c} \\
		\sigma \uplus \nu' \vdash R\mpar{s'_{21}, s'_{31x}}\psubst{\psi_{21} \uplus \psi_{31x}} \goal{3d}
	\end{gather}
	Because composition doesn't change the produced actions nor its variables, \goal{3a} is proved by reflexivity.
	\goal{3b} holds because the valuations for the actions of the holes in \(J_1\) coincide by definition of \(\nu'\) and by \hyp{3b} for the others (\(k \notin H\)).
	For \goal{3c} we have two cases, if \(k \notin J'_{12}\) then \(g_{31x} \equiv g_{13}\) and \hyp{2d} is sufficient (\(g_{21} \equiv g_{13}\)), if \(k \in J'_{13}\) then \(g_{31x} \equiv g_{13} \wedge g_{3x} \wedge \alpha_{3x} = \beta_{13k}\) so we also need \hyp{3c} and \hyp{3a} in addition to \hyp{2d} (\(g_{21} \equiv g_{13} \wedge \alpha_2 = \beta_{12k}\)).
	\goal{3d} is proved using \hyp{3d}, \hyp{2b} with the preservation of reachability across transitions (+ \hyp{2d}), \hyp{2c} and the fact that \(A_1\) performs the same transition with same values in both composed automata:
	\begin{align*}
		& \sigma \uplus \nu' \vdash R\mpar{s'_{21}, s'_{31x}}\psubst{\psi_{21} \uplus \psi_{31x}} \\
		\iff & \sigma \uplus \nu' \vdash \mpar{R_{23}\mpar{s'_2, s'_{3x}} \wedge \reach{A_{21}}\mpar{s'_{21}} \wedge \bigwedge_\subbox{v_1 \in V_1} v_{12} = v_{13}}\psubst{\psi_{12} \uplus \psi_2 \uplus \psi_{13} \uplus \psi_{3x}} \\
		\iff & \sigma \uplus \nu' \vdash R_{23}\mpar{s'_2, s'_{3x}}\psubst{\psi_2 \uplus \psi_{3x}} \wedge \reach{A_{21}}\mpar{s'_{21}}\psubst{\psi_{21}} \wedge \bigwedge_\subbox{v_1 \in V_1} \psi_{12}\mpar{v_{12}} = \psi_{13}\mpar{v_{13}} \\
		\impliedby & \sigma \uplus \nu' \vdash R_{23}\mpar{s'_2, s'_{3x}}\psubst{\psi_2 \uplus \psi_{3x}} \wedge \reach{A_{21}}\mpar{s'_{21}}\psubst{\psi_{21}} \wedge \bigwedge_\subbox{v_1 \in V_1 \uplus \fvars{t_1}} v_{12} = v_{13} \\
		\iff & \mpar{\sigma \uplus \nu \vdash R_{23}\mpar{s'_2, s'_{3x}}\psubst{\psi_2 \uplus \psi_{3x}}} \wedge \mpar{\sigma \vdash \reach{A_{21}}\mpar{s'_{21}}\psubst{\psi_{21}}} \wedge \mpar{\sigma \vdash \bigwedge_\subbox{v_1 \in V_1} v_{12} = v_{13}}
	\end{align*}
\item[\goal{1}:] Let \nmm{\sigma: V_{21} \uplus V_{31} \uplus \biguplus_\subbox{t_{31} \in \fOT{s_{31}}} \fvars{t_{31}} \to \values}.
	We use dummy values for the variables in \nmm{\biguplus_\subbox{t_{21} \in \fOT{s_{21}}} \fvars{t_{21}}} to get the left side of the implication (independant from these values), immediately broken into
	\begin{gather}
		\sigma \vdash R_{23}\mpar{s_2, s_3} \hyp{4a} \\
		\sigma \vdash \reach{A_{21}}\mpar{s_{21}} \hyp{4b} \\
		\sigma \vdash \bigvee_\subbox{t_{31} \in \fOT{s_{31}}} \fguard{t_{31}} \hyp{4c}
	\end{gather}
	\(R_{23}\) is a simulation and \(\reach{A_{21}}\) is a witness of the non-locking composition so we have the hypotheses:
	\begin{gather}
		V_2 \uplus V_3 \uplus \biguplus_\subbox{t_3 \in \fOT{s_3}} \fvars{t_3} \vdash R_{23}\mpar{s_2, s_3} \wedge \bigvee_\subbox{t_3 \in \fOT{s_3}} \fguard{t_3} \implies \bigvee_\subbox{t_2 \in \fOT{s_2}} \fguard{t_2} \hyp{5} \\
		V_{21} \uplus \biguplus_\subbox{t_{12} \in \fOT{s_{12}}} \fvars{t_{12}} \vdash \reach{A_{21}}\mpar{s_{21}} \wedge \bigvee_\subbox{t_{12} \in \fOT{s_{12}}} \fguard{t_{12}} \implies \bigvee_\subbox{t_{21} \in \fOT{s_{21}}} \fguard{t_{21}} \hyp{6}
	\end{gather}
	We use \hyp{5} with valuation \(\sigma\) to get \nmm{\nu: \biguplus_\subbox{t_{12} \in \fOT{s_{12}}} \fvars{t_{12}} \to \values}.
	The left side of the implication is proved using \hyp{4a} and \hyp{4c} because the guards from transitions of \(A_3\) are still in transitions of \(A_{31}\), the right side gives:
	\[ \sigma \uplus \nu \vdash \bigvee_\subbox{t_2 \in \fOT{s_2}} \fguard{t_2} \hyp{5'} \]
	We then use \hyp{6} with valuation \(\sigma \uplus \nu\) to get \nmm{\nu': \biguplus_\subbox{t_{21} \in \fOT{s_{21}}} \fvars{t_{21}}}.
	The left side of the implication is proved using \hyp{4b} and \hyp{5'}, the right side gives:
	\[ \sigma \uplus \nu' \vdash \bigvee_\subbox{t_{21} \in \fOT{s_{21}}} \fguard{t_{21}} \hyp{6'} \]
	\goal{1} is proved for valuation \(\nu'\) by admitting the left side and using \hyp{6'}.
\end{proof}

\end{document}