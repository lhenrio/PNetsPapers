\documentclass[11pt,fleqn]{article}
\usepackage{graphicx,url}
\usepackage{subfigure}
\usepackage[T1]{fontenc}
\usepackage[utf8]{inputenc}
\usepackage{amsmath}
\usepackage{amsfonts}

\oddsidemargin 0cm
\evensidemargin 0cm
\topmargin -0.5cm
%\textheight 23cm  %\advance\textheight by \topskip 
%\textwidth 18.5cm
% current ---
\textheight 21cm  %\advance\textheight by \topskip 
\textwidth 16cm
\raggedbottom


%\usepackage{scrpage2,lastpage,tikz,
\usepackage{url,amsmath}

%\graphicspath{{../../logos/}{./figs/}}
%
%\newcommand{\logos}{%
%  \begin{tikzpicture}[remember picture,overlay]
%    \node [yshift=-1cm] at (current page.north) [below] 
%      {\includegraphics[height=15mm]{logo_lip}%\hspace*{2cm}
%%        \includegraphics[height=15mm]{citi}
%
%};
% \end{tikzpicture}}
%\newcommand{\adresse}{%
%  \begin{tikzpicture}[remember picture,overlay]
%    \node [yshift=1cm] at (current page.south) [above] 
%       {\begin{tabular}{c}
%
%        \textsf{\textup{\color{blue}
%          LIP -- UMR CNRS / ENS Lyon / UCB Lyon 1 / INRIA -- 69007
%          Lyon  %et         CEA Tech - LIST
%        }}\\
%        \textsf{\textup{\color{blue}
%         %- +33 (0)3 59 57 78 24
%         % -- Fax. : +33 (0)3 28 77 85 37
%          Contact E-mail : amaury.maille@ens-lyon.fr -- matthieu.moy@univ-lyon1.fr -- ludovic.henrio@ens-lyon.fr}}
%      \end{tabular}};
% \end{tikzpicture}}


\thispagestyle{empty}

\begin{document}
\begin{center}
\resizebox{0.5\textwidth}{!}{\bf Internship subject (Master 2 or Master 1)}
\\\bigskip
\resizebox{\textwidth}{!}{\bf Session Types meet Open Automata}
%  Systolisation d'un réseau de processus polyédrique}
\end{center}
\medskip

\begin{description}
\item[\bf Main advisor:] Rabéa Ameur-Boulifa ?
\item[\bf Co-advisors:] Ludovic Henrio and Eric Madelaine?
\item[\bf Place:] Eurecom  -- Sophia Antipolis
\\
or\\
Laboratoire de l'Informatique du Parallélisme (LIP) --
  \'Ecole Normale Supérieure de Lyon
\end{description}

\subsection*{Context}




In the previous years, we have studied theoretical foundations for open
systems and we defined  open automata \cite{arxiv-weakbisim,henrio:Forte2016,hou:hal-02406098} that can be seen as labelled transition systems (LTSs) with parameters and holes. The transitions of open
automata are much more complex than transitions of an LTS as the firing on a transition depends on parameters and actions that are symbolic.

Establishing  whether two open automata are equivalent  or not requires the designer to provide the behaviour of all holes in a context where they are used; which consists in describing all the interactions and the series of inputs and outputs of data exchanged  between the automata and its environment.  
To make those reusable, we are interested in enriching holes with description of behaviours to give them enough power and enable us to check equivalences for these open systems. The description will characterise the assumptions that the automaton makes about the behavior of its surroundings  in order to guarantee certain conditions.


To tackle this issue, we ask: { \it Can we synthesise a global type from a collection of automata so as to ensure the composability of an open automaton  with compatible  surrounding automaton? }



\subsection*{Objectives}

The main objective of the intern ship  deals with the characterization of interactions  of open automaton with its environment as behavioural types.
More specifically, the intern ship would follow the following steps:
\begin{itemize}
\item The internship will start with a familiarization with technical material: the semantics of  open automata and the notion behavioural types \cite{Hans:2016}, in particular the notion of session types that allows to describe the dynamic aspects of processes. 

\item Design of an adequate characterisation of  compatibility condition  suitable for the composition open automata, and which is sufficient to decide if two open automata are compatible or not. 
\item Examples/Algorithm




\end{itemize}
 
\begin{thebibliography}{9}

\bibitem{Hans:2016}
Hans H\"uttel, Ivan Lanese, Vasco T. Vasconcelos, Lu\`is Caires, Marco Carbone, Pierre-Malo Deni\'elou, Dimitris Mostrous, Luca Padovani, Ant\`onio Ravara, Emilio Tuosto, Hugo Torres Vieira, Gianluigi Zavattaro:
\newblock {Foundations of Session Types and Behavioural Contracts}.
\newblock ACM Comput. Surv. 49(1): 3:1-3:36 (2016)



\bibitem{henrio:Forte2016}
Henrio, L., Madelaine, E., Zhang, M.:
\newblock {A Theory for the Composition of Concurrent Processes}.
\newblock In Albert, E., Lanese, I., eds.: {36th International Conference on
  Formal Techniques for Distributed Objects, Components, and Systems (FORTE)}.
  Volume LNCS-9688 of Formal Techniques for Distributed Objects, Components,
  and Systems., Heraklion, Greece (June 2016)  175--194. 
\newblock  \url{https://hal.inria.fr/hal-01299562}

\bibitem{hou:hal-02406098}
Hou, Z., Madelaine, E.:
\newblock {Symbolic Bisimulation for Open and Parameterized Systems}.
\newblock In: {PEPM 2020 - ACM SIGPLAN Workshop on Partial Evaluation and
  Program Manipulation}, New-Orleans, United States (January 2020). 
 \newblock \url{https://hal.inria.fr/hal-02406098}

\bibitem{arxiv-weakbisim}
Rabéa Ameur-Boulifa and Ludovic Henrio and Eric Madelaine:
\newblock {Compositional equivalences based on Open pNets}.
\newblock arXiv {2007.10770} (2020).
 \newblock \url{https://arxiv.org/abs/2007.10770}
\end{thebibliography}

\end{document}
