\documentclass[11pt,fleqn]{article}
\usepackage{graphicx,url}
\usepackage{subfigure}
\usepackage[T1]{fontenc}
\usepackage[utf8]{inputenc}
\usepackage{amsmath}
\usepackage{amsfonts}

\oddsidemargin 0cm
\evensidemargin 0cm
\topmargin -0.5cm
%\textheight 23cm  %\advance\textheight by \topskip 
%\textwidth 18.5cm
% current ---
\textheight 21cm  %\advance\textheight by \topskip 
\textwidth 16cm
\raggedbottom


%\usepackage{scrpage2,lastpage,tikz,
\usepackage{url,amsmath}

%\graphicspath{{../../logos/}{./figs/}}
%
%\newcommand{\logos}{%
%  \begin{tikzpicture}[remember picture,overlay]
%    \node [yshift=-1cm] at (current page.north) [below] 
%      {\includegraphics[height=15mm]{logo_lip}%\hspace*{2cm}
%%        \includegraphics[height=15mm]{citi}
%
%};
% \end{tikzpicture}}
%\newcommand{\adresse}{%
%  \begin{tikzpicture}[remember picture,overlay]
%    \node [yshift=1cm] at (current page.south) [above] 
%       {\begin{tabular}{c}
%
%        \textsf{\textup{\color{blue}
%          LIP -- UMR CNRS / ENS Lyon / UCB Lyon 1 / INRIA -- 69007
%          Lyon  %et         CEA Tech - LIST
%        }}\\
%        \textsf{\textup{\color{blue}
%         %- +33 (0)3 59 57 78 24
%         % -- Fax. : +33 (0)3 28 77 85 37
%          Contact E-mail : amaury.maille@ens-lyon.fr -- matthieu.moy@univ-lyon1.fr -- ludovic.henrio@ens-lyon.fr}}
%      \end{tabular}};
% \end{tikzpicture}}


\thispagestyle{empty}

\begin{document}
\begin{center}
\resizebox{0.5\textwidth}{!}{\bf Internship subject (Master 2 or Master 1)}
\\\bigskip
\resizebox{\textwidth}{!}{\bf Open Automata meet Session Types}
%  Systolisation d'un réseau de processus polyédrique}
\end{center}
\medskip

\begin{description}
\item[\bf Advisors:] Rabéa Ameur-Boulifa\textsuperscript{(1)} and Ludovic Henrio\textsuperscript{(2)}
\item[\bf Place:] ~\\(1) Eurecom  -- Sophia Antipolis
\\
or\\
(2) Laboratoire de l'Informatique du Parallélisme (LIP) --
  ENS de Lyon

\item[contact:]  \quad \url{rabea.ameur-boulifa@telecom-paris.fr }  \quad \qquad \url{ludovic.henrio@ens-lyon.fr}
\end{description}

\subsection*{Context}




In the previous years, we have studied theoretical foundations for open
systems and we defined  open automata \cite{arxiv-weakbisim,henrio:Forte2016,hou:hal-02406098} that can be seen as labelled transition systems (LTSs) with parameters and holes. The transitions of open
automata are much more complex than transitions of an LTS: they include guards expressing the relations between the parameters of the automaton with the actions of the holes, and assignments encoding their effects.
% as the firing on a transition depends on parameters and actions that are symbolic.
 We proved that our models 
have good properties, namely that composition preserves some properties like bisimulation.



The composition of open automata refers to filling holes, it requires to check the compatibility of their sorts. Nowadays, we simply verify the compatibility of two automata to be composed based on   action labels and  arity (comparing actions exposed by one LTS with actions expected to fill the hole). But, 
 we are interested in enriching holes with description of behaviours; this description should look like a communication protocol. Precisely, the theory of session types and behavioural types  \cite{Fantechi:MLTCDP2019} are intended  to describe dynamic aspects of the behaviour of processes.

 
To tackle this issue, we ask: { \it Can we synthesise a global type from a collection of automata in order to ensure the composability of an open automaton  with a compatible  surrounding automaton? }



\subsection*{Objectives}

The main objective of the internship is to deal with the characterization of interactions  of an automaton with its environment. These interactions can be characterised as behavioural types.
More specifically, the internship would follow the following steps:
\begin{itemize}
\item Familiarization with technical material: the semantics of  open automata and the notion behavioural types \cite{Hans:2016} that allows the description of the dynamic aspects of processes. In particular the notion of session types that focus on process interactions should be studied.

\item Design of an adequate characterisation of  compatibility condition  suitable for the composition open automata, and study the guarantees brought by this condition. 

\item Prove absence of deadlock upon composition.

\item  We  have defined a notion of bisimulation between open automata. 
One additional objective of the internship could be to prove that behavioural types behave well relatively to bisimilarity.

\item Design of examples demonstrating the benefit of this work.

\end{itemize}

The internship can be tackled only theoretically on paper, or formalised in an interactive theorem prover, or integrated with existing tools for open automata based on SMT solving.
 
\begin{thebibliography}{9}

\bibitem{Hans:2016}
Hans H\"uttel, Ivan Lanese, Vasco T. Vasconcelos, Lu\`is Caires, Marco Carbone, Pierre-Malo Deni\'elou, Dimitris Mostrous, Luca Padovani, Ant\`onio Ravara, Emilio Tuosto, Hugo Torres Vieira, Gianluigi Zavattaro:
\newblock {Foundations of Session Types and Behavioural Contracts}.
\newblock ACM Comput. Surv. 49(1): 3:1-3:36 (2016)



\bibitem{henrio:Forte2016}
Henrio, L., Madelaine, E., Zhang, M.:
\newblock {A Theory for the Composition of Concurrent Processes}.
\newblock In Albert, E., Lanese, I., eds.: {36th International Conference on
  Formal Techniques for Distributed Objects, Components, and Systems (FORTE)}.
  Volume LNCS-9688 of Formal Techniques for Distributed Objects, Components,
  and Systems., Heraklion, Greece (June 2016)  175--194. 
\newblock  \url{https://hal.inria.fr/hal-01299562}

\bibitem{Fantechi:MLTCDP2019}
Alessandro Fantechi, Elie Najm, Jean-Bernard Stefani:
\newblock {From Behavioural Contracts to Session Types}.
\newblock Models, Languages, and Tools for Concurrent and Distributed Programming 2019: 278-297



\bibitem{hou:hal-02406098}
Hou, Z., Madelaine, E.:
\newblock {Symbolic Bisimulation for Open and Parameterized Systems}.
\newblock In: {PEPM 2020 - ACM SIGPLAN Workshop on Partial Evaluation and
  Program Manipulation}, New-Orleans, United States (January 2020). 
 \newblock \url{https://hal.inria.fr/hal-02406098}

\bibitem{arxiv-weakbisim}
Rabéa Ameur-Boulifa and Ludovic Henrio and Eric Madelaine:
\newblock {Compositional equivalences based on Open pNets}.
\newblock arXiv {2007.10770} (2021).
 \newblock \url{https://arxiv.org/abs/2007.10770}
\end{thebibliography}

\end{document}
