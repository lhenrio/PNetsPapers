\documentclass[11pt,fleqn]{article}
\usepackage{graphicx,url}
\usepackage{subfigure}
\usepackage[T1]{fontenc}
\usepackage[utf8]{inputenc}
\usepackage{amsmath}
\usepackage{amsfonts}

\oddsidemargin 0cm
\evensidemargin 0cm
\topmargin -0.5cm
%\textheight 23cm  %\advance\textheight by \topskip 
%\textwidth 18.5cm
% current ---
\textheight 21cm  %\advance\textheight by \topskip 
\textwidth 16cm
\raggedbottom


%\usepackage{scrpage2,lastpage,tikz,
\usepackage{url,amsmath}

%\graphicspath{{../../logos/}{./figs/}}
%
%\newcommand{\logos}{%
%  \begin{tikzpicture}[remember picture,overlay]
%    \node [yshift=-1cm] at (current page.north) [below] 
%      {\includegraphics[height=15mm]{logo_lip}%\hspace*{2cm}
%%        \includegraphics[height=15mm]{citi}
%
%};
% \end{tikzpicture}}
%\newcommand{\adresse}{%
%  \begin{tikzpicture}[remember picture,overlay]
%    \node [yshift=1cm] at (current page.south) [above] 
%       {\begin{tabular}{c}
%
%        \textsf{\textup{\color{blue}
%          LIP -- UMR CNRS / ENS Lyon / UCB Lyon 1 / INRIA -- 69007
%          Lyon  %et         CEA Tech - LIST
%        }}\\
%        \textsf{\textup{\color{blue}
%         %- +33 (0)3 59 57 78 24
%         % -- Fax. : +33 (0)3 28 77 85 37
%          Contact E-mail : amaury.maille@ens-lyon.fr -- matthieu.moy@univ-lyon1.fr -- ludovic.henrio@ens-lyon.fr}}
%      \end{tabular}};
% \end{tikzpicture}}


\thispagestyle{empty}

\begin{document}
\begin{center}
\resizebox{0.5\textwidth}{!}{\bf Internship subject (Master 2 or Master 1)}
\\\bigskip
\resizebox{\textwidth}{!}{\bf Refinement for open automata}
%  Systolisation d'un réseau de processus polyédrique}
\end{center}
\medskip

\begin{description}
\item[\bf Main advisor:] Rabéa Ameur Boulifa 
\item[\bf Co-advisors:] Ludovic Henrio and Eric Madelaine
\item[\bf Place:] Eurecom  -- Sophia Antipolis
\\
or\\
Laboratoire de l'Informatique du Parallélisme (LIP) --
  \'Ecole Normale Supérieure de Lyon
\end{description}

\subsection*{Context}


Establishing equivalences or refinement relations between programs or system is crucial both for verifying correctness of programs, by establishing that one implementation is the refinement of a specification. 

In the previous years, we have studied theoretical foundations for open
systems and our formalism, called open automata, is able to represent operators of composition of processes, 
they are represented
as hierarchically composed automata with holes and parameters. 
Our long 
term goal is to 
develop a methodology combining symbolic operational semantic and bisimulation 
equivalences with deductive reasoning on the data part, 
and in practice combining bisimulation algorithms with SMT solvers to get automatic 
procedures proving equational properties
of these open systems. In the last years, we designed a weak bisimulation theory for  open automata and a translation from a specification language that we used, called pNets, to open automata~\cite{arxiv-weakbisim,henrio:Forte2016,hou:hal-02406098}.


Among the properties of our formalisms, we are interested in compositionality: If two systems are proven equivalent they will be undistinguishable by their context, and they will also be undistinguishable when their holes are filled with equivalent systems.
The article is illustrated with a transport protocol running example; it shows the characteristics of our formalism and our bisimulation relations.

\subsection*{Objectives}


The main objective of the internship is to study refinement theory for open automata.  
Our purpose is to define a refinement relation between the symbolic models allowing refinement verification for a class of parameterized systems.

The internship will start with a study of the semantics of a simple language called value passing CCS. The first objective of this internship is to design a semantics value passing CCS in terms of open automata. From this point we will be able to extend the semantics by taking into account the semantics of operators for composing CCS processes.

After this preliminary step, the next objective will be to design a refinement theory for open automata. We will make sure that this relation is specified in a constructive manner, so that an algorithm could be derived from the specification in the future.
The relationship between refinement and automata composition should also play a major role in this definition.

%The purpose is to define the refinement semantics as a relation between open automata that can be classified in  and we will make sure that an algorithm could be designed to verify refinement can also be verified for finite state systems using an algorithmic method.


The next steps in the internship will rely on these two initial works; they will consist in addressing some of the following (independent) objectives:
\begin{itemize}
\item Express and prove compositionality properties for refinement relation over open automata.
\item Study formally the behaviour of CCS operators wrt refinement.
\item Study the classification of the refinement semantics in the van Glabbeek spectrum could also be studied.
\item Design the algorithm that checks refinement between wo open automata.
\end{itemize}

\begin{thebibliography}{9}

\bibitem{Bellegarde:MEMOCODE2003}
Fran{\c{c}}oise Bellegarde,Celina Charlet and Olga Kouchnarenko:
\newblock {How to Compute the Refinement Relation for Parameterized Systems}.
\newblock 1st {ACM} {\&} {IEEE} International Conference on Formal Methods and Models for Co-Design {(MEMOCODE} 2003), 24-26 June 2003, Mont Saint-Michel, France

\bibitem{henrio:Forte2016}
Henrio, L., Madelaine, E., Zhang, M.:
\newblock {A Theory for the Composition of Concurrent Processes}.
\newblock In Albert, E., Lanese, I., eds.: {36th International Conference on
  Formal Techniques for Distributed Objects, Components, and Systems (FORTE)}.
  Volume LNCS-9688 of Formal Techniques for Distributed Objects, Components,
  and Systems., Heraklion, Greece (June 2016)  175--194

\bibitem{hou:hal-02406098}
Hou, Z., Madelaine, E.:
\newblock {Symbolic Bisimulation for Open and Parameterized Systems}.
\newblock In: {PEPM 2020 - ACM SIGPLAN Workshop on Partial Evaluation and
  Program Manipulation}, New-Orleans, United States (January 2020)

\bibitem{arxiv-weakbisim}
Rabéa Ameur-Boulifa and Ludovic Henrio and Eric Madelaine:
\newblock {Compositional equivalences based on Open pNets}.
\newblock arXiv {2007.10770} (2020)
\end{thebibliography}

\end{document}
