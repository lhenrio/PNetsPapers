\documentclass[a4paper]{llncs}

\usepackage[T1]{fontenc}
%\usepackage{geometry}                % See geometry.pdf to learn the layout options. There are lots.
%\geometry{a4paper}                   % ... or a4paper or a5paper or ...
%\geometry{landscape}                % Activate for for rotated page geometry
%\usepackage[parfill]{parskip}    % Activate to begin paragraphs with an empty line rather than an indent
\usepackage{graphicx}

%\usepackage{amsfonts}
%\usepackage{fancyhdr}
%\usepackage{cite}
%\usepackage{ifthen}
%\usepackage{amssymb}
%\usepackage{fancyhdr}
%\usepackage{pifont}
\usepackage{stmaryrd}
\usepackage{mathtools,mathpartir}
\usepackage{proof}
%\usepackage{setspace}
%\usepackage{indentfirst}
\usepackage{amsmath,amssymb,amscd,mathrsfs}
\usepackage{array,booktabs,arydshln,xcolor}
\DeclareGraphicsRule{.tif}{png}{.png}{`convert #1 `dirname #1`/`basename #1 .tif`.png}
\usepackage{epsfig,color,subfigure,enumitem}
\newcommand{\TODO}[1]{\textcolor{red}{\textbf{[TODO:#1]}}}
\newcommand{\NOTE}[1]{\textcolor{blue}{\textbf{[NOTE:#1]}}}
\newcommand{\ERIC}[1]{\textcolor{blue}{#1}}
\newcommand{\LUDO}[1]{\textcolor{green}{#1}}
\definecolor{airforceblue}{rgb}{0.26, 0.44, 0.56}
\newcommand{\QIN}[1]{\textcolor{airforceblue}{#1}}
\newcommand{\coloncolon}{{:\hspace{-.2ex}:}}
\makeatletter
\newcommand{\raisemath}[1]{\mathpalette{\raisem@th{#1}}}
\newcommand{\raisem@th}[3]{\raisebox{#1}{$#2#3$}}
\makeatother

\usepackage{macrospNets}

\def\AlgT{\mathcal{T}}
\def\AlgE{\mathcal{E}}
\def\AlgA{\mathcal{A}}
\def\AlgAS{\mathcal{A}_S}
\def\AlgB{\mathcal{B}}
\def\AlgI{\mathcal{I}}
\newcommand{\set}[1]{\overline{#1}}
\newcommand{\Pred}{\symb{Pred}}
\newcommand{\Post}{\symb{Post}}
%\usepackage[math]{cellspace}
%\setlength\cellspacetoplimit{ 37pt}
%\setlength\cellspacebottomlimit{18pt}

\pagestyle{plain}
% addition to the mathpartir package for red dotted rules,
% that we use for open-transitions
\makeatletter
\def \dotover {\textcolor{red}{\leavevmode\cleaders\hb@xt@ .22em{\hss $\cdot$\hss}\hfill\kern\z@}}
\def \reddottedrule #1#2{\hbox {\advance \hsize by -0.5em
%\sbox0{$\genfrac{}{}{0pt}{0}{#1}{#2}$} \phantom{\copy0} %
 {\ooalign{\vphantom{$\genfrac{}{}{0pt}{0}{#1}{#2}$}\cr\dotover\cr$\genfrac{}{}{0pt}{0}{#1}{#2}$\cr}}}}

 \def \dottedrule #1#2 {
  {\sbox0{$\genfrac{}{}{0pt}{0}{#1}{#2}$}%
    \vphantom{\copy0}%
    \ooalign{%
      \hidewidth
      $\vcenter{\moveright\nulldelimiterspace
        \hbox to\wd0{%
         \xleaders\hbox{\kern.5pt\vrule height 0.4pt width 1.5pt\kern.5pt}\hfill
          \kern-1.5pt
        }%
      }$
      \hidewidth\cr
    \box0\cr}}
}

\let \defaultfraction \mpr@@fraction
\makeatother

%\newtheorem{theorem}{Theorem}[section]
\newtheorem{prop}[theorem]{Proposition}
\newtheorem{cor}[theorem]{Corollary}
%\newtheorem{lemma}[theorem]{Lemma}
\newtheorem{algorithm}[theorem]{Algorithm}
%\newtheorem{remark}[theorem]{Remark}
%\newtheorem{definition}[theorem]{Definition}
%\newtheorem{example}[theorem]{Example}
%\newtheorem{problem}[theorem]{Problem}
%\newtheorem{proof}[theorem]{Proof}

% Macros for the SOS rules and proof trees:
%\newcommand\openrule[2]{\redinfer{#1}{#2}}
\newcommand\openrule[2]{\inferrule*[myfraction=\reddottedrule,center]{#1}{#2}}
%\newcommand\openrule[2]{\inferrule*{#1}{#2}}
%\newcommand\ostate[1]{\triangleleft{\;#1\;}\triangleright}
\newcommand\ostate[1]{\triangleleft{#1}\triangleright}
\newcommand{\sm}[1]{\mbox{\boldmath\small #1}}
\DeclareMathOperator{\card}{card}
\DeclareMathOperator{\Flat}{Flat}
\renewcommand{\P}{\mathcal P}

\begin{document}
\section{Type system}
The SMT solver Z3 already has conservative type rules and checks type safety in Horn logic. When translating the pNets into SMT-LIB language, pNets should also make sure that
the term matching other term in SMT-LIB language will be correct in type checking. In next two section, we give out the presentation of the pNets and the type rules of them to ensure the type safety. 
\subsection{Algebra Presentation}
We have a set of basic algebra presentation for the pNet, including three basic sorts $Bool$, $Action$ and $Int$ and their own operators(constants as operators with no argument).
\begin{table}\caption{\leftline{Algebra Presentation}}
	\begin{tabular}{p{4cm}p{8cm}}
		\hline\specialrule{0em}{1pt}{1pt}
		Sort & Operator \\\specialrule{0em}{1pt}{1pt}
		\hdashline\specialrule{0em}{3pt}{3pt}
		Bool    			& $=,\ \ne,\ \land,\ \lor,\ \texttt{true},\ \texttt{false},\ \texttt{FUN}$ 								\\\specialrule{0em}{1pt}{1pt}
		Action 			& $=,\ \ne,\ \texttt{FUN}$ 																	\\\specialrule{0em}{1pt}{1pt}
		Int 				& $=,\ \ne,\ - \texttt{(unary)},\ +,\ - \texttt{(binary)},\ \times,\ \div,\ \texttt{Nat},\ \texttt{FUN}$ 			\\\specialrule{0em}{1pt}{1pt}
		\hline
	\end{tabular}
\end{table}	


\subsection{Type Rule}
The environment is a set of variables with their own types like $x_1 : A_1, ... , x_n : A_n$. And use dom$(\Gamma)$ dedicate the collection of $x_1, ... , x_n$. Let $\mathcal{P}$ to be the presentation of the pNet.
\begin{table}\caption{\leftline{Judgments for Open pNets}}
	\begin{tabular}{p{5cm}p{7cm}}
		\hline\specialrule{0em}{3pt}{3pt}
		$\Gamma \vdash \diamond$ 					& $\Gamma$ is a well-formed environment 					\\\specialrule{0em}{1pt}{1pt}
		$\Gamma \vdash A$ 							& A is a well-formed type in $\Gamma$	 					\\\specialrule{0em}{1pt}{1pt}
		$\Gamma \vdash M: A$ 						& M is a well-formed term of type A in $\Gamma$			\\\specialrule{0em}{1pt}{1pt}
		\specialrule{0em}{3pt}{3pt}\hline
	\end{tabular}
\end{table}	

\begin{table}\caption{\leftline{Type Rules for Open pNets}}
	\begin{tabular}{p{5cm}p{5cm}p{2.5cm}}
		\hline\specialrule{0em}{3pt}{3pt}
		(Env $\varnothing$) 								
		& 										
		&					\\\specialrule{0em}{1pt}{1pt}
            $\dfrac{ }{\varnothing \vdash \diamond}$			
            & %$\dfrac{\Gamma \vdash A ~~ x\ }{\varnothing \vdash \diamond}$
            &					\\\specialrule{0em}{3pt}{3pt}
		(Type Bool) 										
		&(Type Action) 						
		&(Type Int)			\\\specialrule{0em}{1pt}{1pt}
		$\dfrac{\mathcal{P} \vdash Bool~~\Gamma \vdash \diamond}{\mathcal{P},\Gamma \vdash Bool}$ 
		& $\dfrac{\mathcal{P} \vdash Action~~\Gamma \vdash \diamond}{\mathcal{P},\Gamma \vdash Action}$ 
		& $\dfrac{\mathcal{P} \vdash Int~~\Gamma \vdash \diamond}{\mathcal{P},\Gamma \vdash Int}$        \\\specialrule{0em}{3pt}{3pt}
		(Var x) 										
		& 						
		&			\\\specialrule{0em}{1pt}{1pt}
		$\dfrac{\mathcal{P} \vdash A~~\Gamma \vdash x:A}{\mathcal{P},\Gamma \vdash x:A}$ 
		&  
		&       		\\\specialrule{0em}{5pt}{5pt}
		(Expr Unary)								
		&					
		& 			\\\specialrule{0em}{1pt}{1pt}
		$\dfrac{\mathcal{P} \vdash A ~~\mathcal{P} \vdash op :: A \rightarrow A ~~\Gamma \vdash x:A}{\mathcal{P},\Gamma \vdash op~x:A}$ 
		& 
		&       		\\\specialrule{0em}{5pt}{5pt}
		(Expr Binary)							
		&					
		& 			\\\specialrule{0em}{1pt}{1pt}
		$\dfrac{\mathcal{P} \vdash A ~~\mathcal{P} \vdash op :: A, A \rightarrow A ~~\Gamma \vdash x_1:A ~~\Gamma \vdash x_2:A}{\mathcal{P},\Gamma \vdash x_1~op~ x_2:A}$ 
		& 
		&       		\\\specialrule{0em}{5pt}{5pt}
%		(Expr +)							
%		&					
%		& 			\\\specialrule{0em}{1pt}{1pt}
%		$\dfrac{\mathcal{P} \vdash + :: Int, Int \rightarrow Int ~~\Gamma \vdash x_1:Int ~~\Gamma \vdash x_2:Int}{\mathcal{P},\Gamma \vdash x_1 + x_2:Int}$ 
%		& 
%		&       		\\\specialrule{0em}{5pt}{5pt}
%		(Expr $\times$)							
%		&					
%		& 			\\\specialrule{0em}{1pt}{1pt}
%		$\dfrac{\mathcal{P} \vdash \times :: Int, Int \rightarrow Int ~~\Gamma \vdash x_1:Int ~~\Gamma \vdash x_2:Int}{\mathcal{P},\Gamma \vdash x_1 \times x_2:Int}$ 
%		& 
%		&       		\\\specialrule{0em}{5pt}{5pt}
%		(Expr $\div$)							
%		&					
%		& 			\\\specialrule{0em}{1pt}{1pt}
%		$\dfrac{\mathcal{P} \vdash \div :: Int, Int \rightarrow Int ~~\Gamma \vdash x_1:Int ~~\Gamma \vdash x_2:Int}{\mathcal{P},\Gamma \vdash x_1 \div x_2:Int}$ 
%		& 
%		&       		\\\specialrule{0em}{5pt}{5pt}
		(Polymorphism)							
		&					
		& 			\\\specialrule{0em}{3pt}{3pt}
		(Poly =)							
		&					
		& 			\\\specialrule{0em}{1pt}{1pt}
		$\dfrac{\mathcal{P} \vdash A ~~\mathcal{P} \vdash = :: A, A \rightarrow Bool ~~\Gamma \vdash x_1:A ~~\Gamma \vdash x_2:A}{\mathcal{P},\Gamma \vdash x_1 = x_2:Bool}$ 
		& 
		&       		\\\specialrule{0em}{5pt}{5pt}
		(Poly $\ne$)							
		&					
		& 			\\\specialrule{0em}{1pt}{1pt}
		$\dfrac{\mathcal{P} \vdash A ~~\mathcal{P} \vdash \ne :: A, A \rightarrow Bool ~~\Gamma \vdash x_1:A ~~\Gamma \vdash x_2:A}{\mathcal{P},\Gamma \vdash x_1 \ne x_2:Bool}$ 
		& 
		&       		\\\specialrule{0em}{5pt}{5pt}
%		(Poly -)							
%		&					
%		& 			\\\specialrule{0em}{1pt}{1pt}
%		$\dfrac{\mathcal{P} \vdash - :: Int \rightarrow Int ~~\Gamma \vdash x:Int}{\mathcal{P},\Gamma \vdash -x:Int}$  
%		&$\dfrac{\mathcal{P} \vdash - :: Int, Int \rightarrow Int ~~\Gamma \vdash x_1:Int ~~\Gamma \vdash x_2:Int}{\mathcal{P},\Gamma \vdash x_1-x_2:Int}$ 
%		&       		\\\specialrule{0em}{5pt}{5pt}
		(Poly $\texttt{FUN}$)							
		&					
		& 			\\\specialrule{0em}{1pt}{1pt}
		$\dfrac{\mathcal{P} \vdash \texttt{FUN} :: A_1,...,A_n \rightarrow A ~~\mathcal{P} \vdash A_1~~...~~\mathcal{P} \vdash A_n ~~\mathcal{P} \vdash A ~~\Gamma \vdash x_1:A_1~~...~~\Gamma \vdash x_n:A_n}{\mathcal{P},\Gamma \vdash \texttt{FUN}(x_1,...,x_n):A}$ 
		& 
		&			\\
		\specialrule{0em}{5pt}{5pt}\hline
	\end{tabular}
\end{table}	

\subsection{Map to SMT-LIB language}
The pNet semantics can be full translated into SMT-LIB language, though some difference on defining functions exist.
\begin{table}\caption{\leftline{Mapping}}
	\begin{tabular}{p{3cm}p{9cm}}
		\hline\specialrule{0em}{5pt}{5pt}			
		(Presentation)							
		&								\\\specialrule{0em}{5pt}{5pt}		
		&$Sort \lhook\joinrel\longrightarrow$	declare-datatypes				\\\specialrule{0em}{3pt}{3pt}
		&$Operator \lhook\joinrel\longrightarrow$	declare-function		\\\specialrule{0em}{3pt}{3pt}
		(Checking)							
		&								\\\specialrule{0em}{5pt}{5pt}
		&$dom(\Gamma) \lhook\joinrel\longrightarrow$	declare-const		\\\specialrule{0em}{3pt}{3pt}
		&$Pred \lhook\joinrel\longrightarrow$	 assert		\\\specialrule{0em}{3pt}{3pt}
		(Expressions)							
		&								\\\specialrule{0em}{5pt}{5pt}
		&$\texttt{FUN}(x_1,...,x_n)$		\\\specialrule{0em}{3pt}{3pt}
		&$x_1- x_n$  $~~~~~~~\lhook\joinrel\longrightarrow$ declare-function	\\\specialrule{0em}{3pt}{3pt}
		&$...$	\\\specialrule{0em}{3pt}{3pt}
		\specialrule{0em}{5pt}{5pt}\hline
	\end{tabular}
\end{table}	

\section{Submission to Z3}
\subsection{Z3 Notations}
Our semantic has already got enough informations for Z3 checking. Z3 has a number of notations for various usages. We only list the notations we needed in our algorithm as follows:
\TODO{Finish the description.}
\begin{enumerate}
\item \texttt{declare-datatypes}
\item \texttt{declare-function}
\item \texttt{declare-const}
\item \texttt{assert}
\end{enumerate}

\subsection{Submission of Algebra}
Before the submission of the open transitions to Z3, the user-inputed algebra of the pNet should be known at first. It gives Z3 informations of the sorts of the expression and the operators, corresponding to the  
\texttt{declare-datatypes} and \texttt{declare-function}. We declare the algebra in Z3 logic through Z3 API methods having the same effect as the notations. At the same time, we have several internal lists (actually hash maps) \texttt{exprs}, \texttt{funcDecls}, \texttt{sortDatatypes} to store the generated Z3 objects with their names for later proofs.

\subsection{Submission of Open Transitions}
Each time we submit each open transition to Z3 module, we translate its predicate into Z3 language format and send it for satisfiability checking. Every term of the predicate is declared as an \texttt{assert} in Z3. A constant action or a parameterized expression is easy to get from the internal list storing the objects while all the variables are not declared at the beginning. So we declare them before the submission of a predicate term with the API method conducting \texttt{declare-const}.

\subsection{Other works}
\paragraph{Quantifier}
\paragraph{Filter the State without Precursor}

\end{document}