\documentclass[10pt]{article}
\usepackage[latin1]{inputenc} 
\usepackage{alltt,url}
%% Old ttbox definition without frame:
\newenvironment{ttbox}{\begin{alltt}\small\tt}%
                      {\end{alltt}}
\usepackage{color}
\newcommand{\Eric}[1]{\textcolor{blue}{#1}}
\thispagestyle{empty}
\addtolength{\textheight}{2cm}
\begin{document}
\section*{Cover Letter }
\subsection*{Submission to Logical Methods in Computer Science}
\medskip
\begin{ttbox}
Title: Compositional equivalences based on Open pNets
Authors: R. Ameur-Boulifa, L. Henrio, and E. Madelaine
\end{ttbox}
\bigskip
%\subsubsection*{Letter to Reviewers and Editors}
Dear reviewers and editors,

\medskip
This article defines bisimulation relations  for the comparison of systems specified as parameterised Networks of synchronised automata; it formalises the work we did in the last 4 years and provides a theory for the operators composing concurrent processes. This theory is based on the definition of operators as open processes. We define the operational semantics of open processes, using open transitions that include symbolic hypotheses on the behaviour of the holes. The  semantics of pNets is given by means of a symbolic open transition system. Behavioural equivalences for pNets, like strong and weak bisimilarities, are defined over such a open transition systems. We finally prove that these equivalences are c
ompositional.



\smallskip

We think this article fits with the scope and criteria of ``Logical Methods in Computer Science'' for the following reasons. First, to our knowledge this is the first paper proposing a theoretical framework for analysing the behaviour of open systems modeled as pNets, i.e., open terms of suitable process calculi.
Second, the quite theoretical and fundamental approach adopted in this paper fits well to the topics of LMCS.
Finally, because we already applied pNets and their equivalence relations to the verification of concrete cases and because we illustrate our approach on a small but realistic example, we believe our article illustrates well the applicability of the theory we propose.




\smallskip

Concerning prior publications, in Forte'16, we have already defined a theory for the operators composing concurrent processes. The theory is based on the definition of operators as open pNets. However, the study of compositionability was only partial, and in particular the proof that bisimulation is an equivalence that is one of the new contributions of this article. Additionally, previous works only formalised strong bisimulation, and the formalisation and proofs dealing with weak bisimulation are also new here.




\smallskip

Considering the length of the submitted material,
 we   suggest to
keep the two first appendices in the final version of the paper (leading to a
paper of 73 pages) \Eric{if we keep A and B, we get 66 pages} because they are important to understand the paper.
The other appendices \Eric{there is only one left out in that case. So siingular. Did we change the structure of appendices ? Do you really intend to keep both A and B ?} will be available as a research report, we keep
them in the submission to ease the work of the reviewers. 
Of course, depending on the opinion of the reviewers, we could also move more material to  the body of the paper, or remove more proofs from the published version and only keep them in the report.


\smallskip

Finally, after a brief exchange with Alexandra Silva, we improved the form of the paper as follows. We improved the spacings here and there (e.g. between subscripts and rest of text) and tracked overfull boxes. We also adopted a reduced number of symbols and tried to reduce the use of $<....>$ to the synchronisation vectors where this notation is usual now.

\begin{flushright}
  Best Regards,\\
  Rab\'ea Ameur-Boulifa, Ludovic Henrio, and Eric Madelaine
\end{flushright}
\end{document}

%%% Local Variables: 
%%% mode: latex
%%% TeX-master: t
%%% End: 
