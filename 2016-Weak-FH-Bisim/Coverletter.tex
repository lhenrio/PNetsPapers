\documentclass[10pt]{article}
\usepackage[latin1]{inputenc} 
\usepackage{alltt,url}
%% Old ttbox definition without frame:
\newenvironment{ttbox}{\begin{alltt}\small\tt}%
                      {\end{alltt}}
\usepackage{color}
\newcommand{\Eric}[1]{\textcolor{blue}{#1}}
\thispagestyle{empty}
\addtolength{\textheight}{2cm}
\begin{document}
\section*{Cover Letter }
\subsection*{Submission to Journal of Logical and Algebraic Methods in Programming}
\medskip
\begin{ttbox}
Title: Compositional equivalences based on Open pNets
Authors: R. Ameur-Boulifa, L. Henrio, and E. Madelaine
\end{ttbox}
\bigskip
%\subsubsection*{Letter to Reviewers and Editors}
Dear reviewers and editors,

\medskip
This article defines bisimulation relations  for the comparison of systems specified as parameterised Networks of synchronised automata (called pNets); it formalises the work we did in the last 4 years and provides a theory for the operators composing concurrent processes. This theory is based on the definition of operators as open processes. We define the operational semantics of open processes, using open transitions that include symbolic hypotheses on the behaviour of the holes. The  semantics of pNets is given by means of a symbolic open transition system. Behavioural equivalences for pNets, like strong and weak bisimilarities, are defined over such a open transition systems. We finally prove that these equivalences are compositional.



\smallskip

We think this article fits with the scope and criteria of ``Journal of Logical and Algebraic Methods in Programming'' for the following reasons. First, to our knowledge this is the first paper proposing a theoretical framework for analysing the behaviour of open systems modeled as pNets, i.e., open terms of suitable process calculi.
Second, the quite theoretical and fundamental approach adopted in this paper fits well to the topics of JLAMP.
Also, we believe our article illustrates well the applicability of the theory we propose.
Finally, the semantics of closed pNets and their usage was published in JLAMP in 2017; this submission extends the previous work to open systems and consequently provides a more symbolic semantics.



\smallskip

Concerning prior publications, in Forte'16, we have already defined a theory for the operators composing concurrent processes. The theory is based on the definition of operators as open pNets. However, the study of compositionability was only partial, and in particular the proof that bisimulation is an equivalence that is one of the new contributions of this article. Additionally, previous works only formalised strong bisimulation, and the formalisation and proofs dealing with weak bisimulation are also new here.




\smallskip

Considering the length of the submitted material, the article is 40 pages long but with 50 pages of appendices. The first 2 appendices (approx. 35 pages) are detailed proofs, and the last one (approx 15 pages) is details on the usecase.
Depending on the policy of the journal concerning paper length and on the opinion of the reviewers we could either publish only the body of the paper, or keep the two first appendices, or keep the 90 pages.
In any case, all appendices will appear in a research report.



\begin{flushright}
  Best Regards,\\
  Rab\'ea Ameur-Boulifa, Ludovic Henrio, and Eric Madelaine
\end{flushright}
\end{document}

%%% Local Variables: 
%%% mode: latex
%%% TeX-master: t
%%% End: 
