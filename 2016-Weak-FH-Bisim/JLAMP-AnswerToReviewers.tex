\documentclass[10pt]{article}
\usepackage[latin1]{inputenc} 
\usepackage{alltt,url}
%% Old ttbox definition without frame:
\newenvironment{ttbox}{\begin{alltt}\small\tt}%
                      {\end{alltt}}
\usepackage{color}
\newenvironment{review}{\bgroup\itshape\begin{quote}}{\end{quote}\egroup}

\newcommand{\TODO}[1]{\textcolor{red}{\textbf{[TODO:#1]}}}
\newcommand{\NOTE}[1]{\textcolor{blue}{\textbf{[NOTE:#1]}}}
\newcommand{\ERIC}[1]{\textcolor{blue}{#1}}
\thispagestyle{empty}
\addtolength{\textheight}{2cm}
\begin{document}
\section*{Answer to reviewers }
\subsection*{Submission to Journal of Logical and Algebraic Methods in Programming}
\medskip
\begin{ttbox}
Title: Compositional equivalences based on Open pNets
Authors: R. Ameur-Boulifa, L. Henrio, and E. Madelaine
\end{ttbox}
\bigskip
%\subsubsection*{Letter to Reviewers and Editors}
Dear reviewers and editors,

Please find enclosed the revision of our submission to JLAMP. We improved our article according to the comments made by the
anonymous reviewers.

We would first like to thank the reviewers for their analysis of our
paper, and for their insightful comments and advices that extremely strengthen the quality of our paper.
%and the useful advices they gave us.

We detail below the major changes we made in this revision and explain how we addressed them.  \emph{Comments that are  addressed trivially and exactly as suggested by the reviewers are omitted}.  

After this, we will review in the next pages the comments of the reviewers and explain how we addressed them. 

The major improvements we performed are the following:
\begin{itemize}
\item We added numerous comparison with related works as suggested (mostly by reviewer 1). 
Most of this positioning has been added to Section 6 ``Related works''. 
We also added a few precise comparisons in the text but we felt it was sometimes too  early, some of them are just forward references to the Section 6 at the points where the knowledgeable reader would expect a comparison with related works.
\item We tried to address the comments of both reviewer requiring n early simple example; in practice we moved in Section  2.2 the definition of the choice operator in pNets, and added informal explanations on the definition of pNets before the formal definitions.
\item We are particularly grateful to the reviewer for highlighting the fact that we misused bisimilarity relatively to bisimulation. We corrected this point and checked thoroughly in the paper that we used the right notion at the right place.
\end{itemize}

We thank again the reviewer for their comments and believe the new version of our article  is much better.

\begin{flushright}
  Best Regards,\\
  Rab\'ea Ameur-Boulifa, Ludovic Henrio, and Eric Madelaine
\end{flushright}

\newpage

\section*{Response to reviewer 1}

\begin{review}
blabla de la review
\end{review}
Answer

\begin{review}
suite
\end{review}
bla






As explained above, we quote below and reply to text requiring changes that we did not  address in a straightforward manner. We additionally produced a version of the paper highlighting all the changes we made: \TODO{Si tu arrives a produire le latexdiff} \url{url???}

\ERIC{P4, *say that this is like in Symbolic Bisimulation}
\begin{quote}Pas sur du tout, je bosse encore la-dessus...
\end{quote}

\ERIC{P4, *Any relations with the conditions on SOS rule formats guaranteeing compositionality of weak bisimilarity?}
\begin{quote} 
\TODO{L'ajout est dans l'introduction pargra. avant la structure}
non je n'ai pas repondu ici... voir paragraph page precedente, est-ce suffisant "
\end{quote}

\ERIC{P14, *Is your definition a combination of De Simone FH-bisimulation and Hennessy and Lin's symbolic bisimulation?}
\begin{quote}
\TODO{L'ajout est dans la section 3.2, fin pargraph 1}
Je n'ai pas redige ca... mais l'idee pourrait etre que en gros oui, et en y ajoutant le formalisme synch vecteurs de Arnold et Nivat, 
mais que notre approache est sensiblement differente, puisque plus syntaxique, pour permettre le developpement de representation finies, d'algorithmes, et d'outils...
\end{quote}


\ERIC{P17, *What is the comoplexity of this "brute force approach"?}
\begin{quote}
le paragraph existant etait difficilement comprehensible et loin de notre implementation dans l'outil. J'ai remplace par quelque chose de plus factuel:
"The definition of this predicate is not constructive. In our tool [39], we construct a logical formula encoding the matching and unification conditions involved, and let an SMT engine (in the current implementation Z3 [30]) decide its satisfiability"
\end{quote}

\ERIC{page 36, *It'd be a good idea to give links to software tools implementing your approach and to examples case-studies}
\begin{quote}
\TODO{L'ajout est dans la conclusion [30,31]...}

J'ai ajoute 7 ligns ici: "In this platform [...] different encodings of operators".
C'est un peu faible parceque 1) les outils de sont pas encore accessibles en ligne (ca aurait du etre l'achevement du stage de Slava...), et 2) on n'a pas de papier sur un vrai "use-case', a part le AVOCS sur le satellite, mais qui ne fait pas de bisimulation, juste des proprietes temporelles.
\end{quote}


\end{document}

%%% Local Variables: 
%%% mode: latex
%%% TeX-master: t
%%% End: 
