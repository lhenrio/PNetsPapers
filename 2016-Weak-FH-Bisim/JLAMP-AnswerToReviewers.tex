\documentclass[10pt]{article}
\usepackage[latin1]{inputenc} 
\usepackage{alltt,url}
%% Old ttbox definition without frame:
\newenvironment{ttbox}{\begin{alltt}\small\tt}%
                      {\end{alltt}}
\usepackage{color}
\newenvironment{review}{\bgroup\itshape\begin{quote}}{\end{quote}\egroup}

\newcommand{\TODO}[1]{\textcolor{red}{\textbf{[TODO:#1]}}}
\newcommand{\NOTE}[1]{\textcolor{blue}{\textbf{[NOTE:#1]}}}
\newcommand{\ERIC}[1]{\textcolor{blue}{#1}}
\thispagestyle{empty}
\addtolength{\textheight}{2cm}
\begin{document}
\section*{Answer to reviewers }
\subsection*{Submission to Journal of Logical and Algebraic Methods in Programming}
\medskip
\begin{ttbox}
Title: Compositional equivalences based on Open pNets
Authors: R. Ameur-Boulifa, L. Henrio, and E. Madelaine
\end{ttbox}
\bigskip
%\subsubsection*{Letter to Reviewers and Editors}
Dear reviewers and editors,

Please find enclosed the revision of our submission to JLAMP. We improved our article according to the comments made by the
anonymous reviewers.

We would first like to thank the reviewers for their analysis of our
paper, and for their insightful comments and advices that extremely strengthen the quality of our paper.
%and the useful advices they gave us.

We detail below the major changes we made in this revision and explain how we addressed them.  \emph{Comments that are  addressed trivially and exactly as suggested by the reviewers are omitted}.  

After this, we will review in the next pages the comments of the reviewers and explain how we addressed them. 

The major improvements we performed are the following:
\begin{itemize}
\item We added numerous comparison with related works as suggested (mostly by reviewer 1). 
Most of this positioning has been added to Section 6 ``Related works''. 
We also added a few precise comparisons in the text but we felt it was sometimes too  early, some of them are just forward references to the Section 6 at the points where the knowledgeable reader would expect a comparison with related works.
\item We tried to address the comments of both reviewer requiring n early simple example; in practice we moved in Section  2.2 the definition of the choice operator in pNets, and added informal explanations on the definition of pNets before the formal definitions.
\item We are particularly grateful to the reviewer for highlighting the fact that we misused bisimilarity relatively to bisimulation. We corrected this point and checked thoroughly in the paper that we used the right notion at the right place.
\end{itemize}



As explained above, we quote below and reply to text requiring changes that we did not  address in a straightforward manner. We additionally produced a version of the paper highlighting all the changes we made.

We thank again the reviewer for their comments and believe the new version of our article  is much better.

\begin{flushright}
  Best Regards,\\
  Rab\'ea Ameur-Boulifa, Ludovic Henrio, and Eric Madelaine
\end{flushright}

\newpage

\section*{Response to reviewer 1}
\begin{review}
Reviewer \#1: Synopsis

The article under review presents under one roof work done by some of its authors (and others) over the years on the study of pNets and open pNets as well as on notions of bisimilarity over those models and their compositionality properties. In previous work, the authors proposed pNets as a hierarchical model for describing large concurrent systems and their specifications, studied notions of behavioural equivalence over pNets, provided some tool support for that model and applied it to case studies.

The aim of this article is to present and study the foundational theory of open pNets, namely pNets with holes at some of their leaves, which can be instantiated when refining the resulting system specification. The article develops the semantics of open pNets via translation to open automata, namely LTSs with parameters and holes. It also develops a behavioural theory for open automata (and thus open pNets) adapting de Simone's FH-bisimilarity to that notion of automaton and showing congruence properties for the resulting strong and weak notions of bisimilarity. In particular, the authors identify sufficient conditions guaranteeing the compositionality of strong and weak bisimulation equivalence. The technical contributions are illustrated throughout with a running example that describes a transport protocol.

Evaluation

I enjoyed reading the paper somewhat and appreciate the long-term effort by the authors and their co-workers on using variations on time-honoured models and techniques from process algebra and concurrency theory to good effect in modelling and verification of non-trivial systems. The authors have also done some valuable tool development accompanying their theoretical work. In particular, I am not aware of other uses of FH-bisimulation-like equivalences in practice.

The technical developments in the paper seem, by and large, sound to me. However, as the authors can see from the annotated scan of their paper that accompanies this short textual review, there is still some work that needs to be done in order to iron out some (relatively minor) remaining issues. Moreover, even though the presentation is reasonably clear, it should still be improved, polished and clarified in places. Last, but not least, it seems to me that several of the notions presented by the authors have connections with classic ones in the literature. I feel that the paper would improve substantially if the authors compared some of their contributions more carefully with the ones I point out in the annotated scan. 
\end{review}
Thank you for the references; we believe we added all of them at some point

\begin{review}
For instance, a reader might ask: Why should I use open pNets instead of the mature mCRL2 tool set, which seems to support many of the features open pNets offer? (Note: I am not claiming that mCRL2 and its modelling language offer everything that open pNets
do. I am asking the authors to discuss their work vis-a-vis existing models and tools more thoroughly.) 
\end{review}
We agree that such comparisons were missing and we compared our approach to mCRL2 and some other approaches (see Section 6). We did not insist much on the comparison of the tools as the focus of the article is mostly theoretical, but we believe that, while mCRL2 tools are more stable and user-ready, the general approach we provide for reasoning on composition of systems could be adapted to extend composition and decomposition results on frameworks like mCRL2, this is why we also cite recent results on the decomposition of mCRL2 processes. A deeper comparison with mCRL2constructs  would also be possible of course but we were afraid it would get lengthy and less focused on the main contribution of the paper.

\begin{review}
What are the relationships with context systems and context-dependent bisimilarity as defined by Larsen and colleagues? Could you compare your model with other hierarchical models of concurrent systems and the verification techniques they support?
\end{review}
We also added several points of comparison with the works of Larsen and colleagues

\begin{review}

In summary, I think that the paper offers a good contribution, but that a revision addressing the comments raised in the annotated scan accompanying this review and the above general comments is needed before the paper can be accepted.

\emph{The rest of the review was provided as a pdf, we only mention a few of the points we treated, all comments have been addressed.}
\end{review}

\begin{review}
P4 *say that this is like in Symbolic Bisimulation
\end{review}
Yes this is similar. We added a note.

\begin{review}
P4, *Any relations with the conditions on SOS rule formats guaranteeing compositionality of weak bisimilarity?
\end{review}
 The relation here is quite far, because our rules are not structural in the sense of SOS operators. We added a remark to make this clear, in the previous page.


\begin{review}
P14, *Is your definition a combination of De Simone FH-bisimulation and Hennessy and Lin's symbolic bisimulation?
\end{review}
Yes. We added a sentence to make this explicit.


\begin{review}
{P17, *What is the complexity of this "brute force approach"?}
\end{review}
We changed this paragraph to make it more concrete and related to our implementation.

\begin{review}
{P17, *Are these related to some of the constructs on rule format for weak bisimilarity"?}
\end{review}
Somehow. We answered this in page 5.

\begin{review}
{page 36, *It'd be a good idea to give links to software tools implementing your approach and to examples case-studies}
\end{review}
In the first paragraph of the conclusion, we added citations to papers describing our tools ans case-studies.



\end{document}

%%% Local Variables: 
%%% mode: latex
%%% TeX-master: t
%%% End: 
