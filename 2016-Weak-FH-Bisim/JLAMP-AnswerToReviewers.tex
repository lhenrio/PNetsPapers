\documentclass[10pt]{article}
\usepackage[latin1]{inputenc} 
\usepackage{alltt,url}
%% Old ttbox definition without frame:
\newenvironment{ttbox}{\begin{alltt}\small\tt}%
                      {\end{alltt}}
\usepackage{color}
\newenvironment{review}{\bgroup\itshape\begin{quote}}{\end{quote}\egroup}

\newcommand{\TODO}[1]{\textcolor{red}{\textbf{[TODO:#1]}}}
\newcommand{\NOTE}[1]{\textcolor{blue}{\textbf{[NOTE:#1]}}}
\newcommand{\ERIC}[1]{\textcolor{blue}{#1}}
\thispagestyle{empty}
\addtolength{\textheight}{2cm}
\begin{document}
\section*{Answer to reviewers }
\subsection*{Submission to Journal of Logical and Algebraic Methods in Programming}
\medskip
\begin{ttbox}
Title: Compositional equivalences based on Open pNets
Authors: R. Ameur-Boulifa, L. Henrio, and E. Madelaine
\end{ttbox}
\bigskip
%\subsubsection*{Letter to Reviewers and Editors}
Dear reviewers and editors,

Please find enclosed the revision of our submission to JLAMP. We improved our article according to the comments made by the
anonymous reviewers.

We would first like to thank the reviewers for their analysis of our
paper, and for their insightful comments and advices that extremely strengthen the quality of our paper.
%and the useful advices they gave us.

We detail below the major changes we made in this revision and explain how we addressed them.  \emph{Comments that are  addressed trivially and exactly as suggested by the reviewers are omitted}.  

After this, we will review in the next pages the comments of the reviewers and explain how we addressed them. 

The major improvements we performed are the following:
\begin{itemize}
\item We added numerous comparison with related works as suggested (mostly by reviewer 1). 
Most of this positioning has been added to Section 6 ``Related works''. 
We also added a few precise comparisons in the text but we felt it was sometimes too  early, some of them are just forward references to the Section 6 at the points where the knowledgeable reader would expect a comparison with related works. Therse references include works by Larsen, quotienting techniques, mCRL2, contextual equivalence, more comparison with GSOS, works of de Simone and symbolic bisimulation, and others.
\item We tried to address the comments of both reviewer requiring n early simple example; in practice we moved in Section  2.2 the definition of the choice operator in pNets, and added informal explanations on the definition of pNets before the formal definitions. We also gathered arguments around observability and choice in a single example as it was a bit confusing.
\item We are particularly grateful to the reviewer for highlighting the fact that we misused bisimilarity relatively to bisimulation. We corrected this point and checked thoroughly in the paper that we used the right notion at the right place.
\end{itemize}



As explained above, we quote below and reply to text requiring changes that we did not  address in a straightforward manner. We additionally produced a version of the paper highlighting all the changes we made.

We thank again the reviewer for their comments and believe the new version of our article  is much better.

\begin{flushright}
  Best Regards,\\
  Rab\'ea Ameur-Boulifa, Ludovic Henrio, and Eric Madelaine
\end{flushright}

\newpage

\section*{Response to reviewer 1}
\begin{review}
Reviewer \#1: Synopsis

The article under review presents under one roof work done by some of its authors (and others) over the years on the study of pNets and open pNets as well as on notions of bisimilarity over those models and their compositionality properties. In previous work, the authors proposed pNets as a hierarchical model for describing large concurrent systems and their specifications, studied notions of behavioural equivalence over pNets, provided some tool support for that model and applied it to case studies.

The aim of this article is to present and study the foundational theory of open pNets, namely pNets with holes at some of their leaves, which can be instantiated when refining the resulting system specification. The article develops the semantics of open pNets via translation to open automata, namely LTSs with parameters and holes. It also develops a behavioural theory for open automata (and thus open pNets) adapting de Simone's FH-bisimilarity to that notion of automaton and showing congruence properties for the resulting strong and weak notions of bisimilarity. In particular, the authors identify sufficient conditions guaranteeing the compositionality of strong and weak bisimulation equivalence. The technical contributions are illustrated throughout with a running example that describes a transport protocol.

Evaluation

I enjoyed reading the paper somewhat and appreciate the long-term effort by the authors and their co-workers on using variations on time-honoured models and techniques from process algebra and concurrency theory to good effect in modelling and verification of non-trivial systems. The authors have also done some valuable tool development accompanying their theoretical work. In particular, I am not aware of other uses of FH-bisimulation-like equivalences in practice.

The technical developments in the paper seem, by and large, sound to me. However, as the authors can see from the annotated scan of their paper that accompanies this short textual review, there is still some work that needs to be done in order to iron out some (relatively minor) remaining issues. Moreover, even though the presentation is reasonably clear, it should still be improved, polished and clarified in places. Last, but not least, it seems to me that several of the notions presented by the authors have connections with classic ones in the literature. I feel that the paper would improve substantially if the authors compared some of their contributions more carefully with the ones I point out in the annotated scan. 
\end{review}
Thank you for the references; we believe we added all of them at some point

\begin{review}
For instance, a reader might ask: Why should I use open pNets instead of the mature mCRL2 tool set, which seems to support many of the features open pNets offer? (Note: I am not claiming that mCRL2 and its modelling language offer everything that open pNets
do. I am asking the authors to discuss their work vis-a-vis existing models and tools more thoroughly.) 
\end{review}
We agree that such comparisons were missing and we compared our approach to mCRL2 and some other approaches (see Section 6). We did not insist much on the comparison of the tools as the focus of the article is mostly theoretical, but we believe that, while mCRL2 tools are more stable and user-ready, the general approach we provide for reasoning on composition of systems could be adapted to extend composition and decomposition results on frameworks like mCRL2, this is why we also cite recent results on the decomposition of mCRL2 processes. A deeper comparison with mCRL2constructs  would also be possible of course but we were afraid it would get lengthy and less focused on the main contribution of the paper.

\begin{review}
What are the relationships with context systems and context-dependent bisimilarity as defined by Larsen and colleagues? Could you compare your model with other hierarchical models of concurrent systems and the verification techniques they support?
\end{review}
We also added several points of comparison with the works of Larsen and colleagues

\begin{review}

In summary, I think that the paper offers a good contribution, but that a revision addressing the comments raised in the annotated scan accompanying this review and the above general comments is needed before the paper can be accepted.

\emph{The rest of the review was provided as a pdf, we only mention a few of the points we treated, all comments have been addressed.}
\end{review}

\begin{review}
P2: can you speculate for the lack of applications ...
\end{review}
Done briefly (not sure long speculation would be much useful but a short hint clearly helps.

\begin{review}
P4 *say that this is like in Symbolic Bisimulation
\end{review}
Yes this is similar. We added a note.

\begin{review}
P4, *Any relations with the conditions on SOS rule formats guaranteeing compositionality of weak bisimilarity?
\end{review}
 The relation here is quite far, because our rules are not structural in the sense of SOS operators. We added a remark to make this clear, in the previous page.

\begin{review}
P5: many sorted setting 
\end{review}
we did not mention many-sorted logic as we are not sure it helps a lot here and it might confuse the reader because we already use sorts (for some kind of interface signatures made of actions). Instead we added a footnote about the different kind of terms both in the paper and in the tools.

\begin{review}
P8: Vars(P) defined too late
\end{review}
We indeed have two mutually dependent definitions, this is merely a problem of presentation of two long definitions, we added an informal definition ans a forward reference.

\begin{review}
P10: can you be sure that your theory ``conservatively extends'' ...
\end{review}
No, in the sense that the proof would be long, easy, only partial by nature (closed vs open systems) and not publishable. We added one sentence here however to clarify things.

\begin{review}
P14, *Is your definition a combination of De Simone FH-bisimulation and Hennessy and Lin's symbolic bisimulation?
\end{review}
Yes. We added a sentence to make this explicit.


\begin{review}
{P17, *What is the complexity of this "brute force approach"?}
\end{review}
We changed this paragraph to make it more concrete and related to our implementation.

\begin{review}
{P17, *Are these related to some of the constructs on rule format for weak bisimilarity"?}
\end{review}
Somehow. We answered this in page 5.

\begin{review}
{page 36, *It'd be a good idea to give links to software tools implementing your approach and to examples case-studies}
\end{review}
In the first paragraph of the conclusion, we added citations to papers describing our tools ans case-studies.

\begin{review}
Reviewer \#2: SUMMARY

The paper studies bisimulation equivalences for open systems, here represented by hierarchically structured systems with placeholders, called open pNets.
Open pNets are tree-like structures that combine fully specified parameterized labelled transition systems (pLTS) and some unknown
components represented as (sorted) holes. Here a pLTS is defined by a set of guarded transitions between states together
with variables assignment. Notably, open pNets can be (de)composed by filling their holes with other pNets.
Bisimulation equivalences for open pNets are obtained by introducing open automata as their semantic model:
open transitions in open automata carry some sort of guarded action labels that allow to require some properties about the
behaviour of placeholders and to fix some variables assignment.
First, a notion of strong bisimulation is defined, whose key feature is the possibility to simulate a single open transition
with a set of open transitions. Such bisimulation is called FH-bisimulation after the Formal Hypotheses it imposes
on the unknown components of the system. The key property is that the equivalence induced by FH-bisimulation
is a congruence w.r.t. the composition of open pNets (Th.5, composability).
Second, weak bisimulation is also defined by taking silent actions into account.
Note that the definition of weak transitions is here more involved than usual due to the rich structure of labels.
Under the hypothesis of non-observability of silent actions (Def.11), the congruence result is also extended
to weak FH-bisimilarity. Although this may look surprising, Def.11 is tailored to avoid the usual counterexamples
that arise in process algebra-like frameworks.
A running example of a simple message transport protocol specification vs implementation (adapted from the literature) is used
to illustrate the key definitions and the main findings.
Related work is extensively discussed in Section 6.

EVALUATION

The problem addressed in the paper (bisimulation for open systems) is interesting
and worth investigation. It falls well in the scope of the journal.
The framework of open pNets is semantically rich but technically heavy, so the notation is quite complicated to follow
in several places. Even the small-sized running example gives rise to complex figures and annotations.
This is due to the presence of actions, guards and assignments in a single label, together with the
need to coordinate the transitions of placeholders by means of synchronisation vectors.
Another limitation is that the structure of the modelled system looks quite static, e.g. holes are used
in a linear manner (cannot be created, duplicated or deleted).
As a coarse grain analogy, the definition of open pNets resembles the notion of "context" in a process
algebraic setting, that is a partial composition of operators whose semantics is defined by (SOS) inference
rules in some format that guarantees the linear usage of holes throughout the computation.
Certainly here the analogy is complicated by the presence of variables, (symbolic) guards and assignments,
and by the fact that ad hoc compositions can be considered (as opposed to having a fixed set of operators
as in the case of process algebras).

I wonder if an incremental presentation (e.g. briefly discussing the case of open pNets without variables first)
could improve the readability.
\end{review}
We introduce a choice operator early on without presenting aspects on variables and before the formalisation.

\begin{review}
The main results are correct as far as I was able to check but not very surprising: their statements are
complicated by the complexity of the chosen framework, but the properties agree with intuition.
The most ingenious aspects are the possibility to simulate an open transition using many transitions,
the composition of transitions and the definition of non-observability of silent actions.
Sometimes it seems that the terms bisimulation and bisimilarity are used interchangeably, while
this should not be the case (e.g., formally speaking we cannot say that bisimulation is an equivalence,
because a FH-bisimulation is not necessarily so, while FH-bisimilarity is).
\end{review}
We clarified this, thanks for pointing this out

\begin{review}
OTHER COMMENTS

...

P7, Def.1: give an example of a pLTS (even just a partially specified one).
\end{review}
We now have the choice operator with a pLTS before

\begin{review}
\ldots

P15,Def.7: give an example of a transition that is (bi)simulated by many
\end{review}

We added this, in a partial automaton though to avoid a lengthy example here
\begin{review}
P15,Th.1: FH-bisimulation is an equivalence -> FH-bisimilarity is an equivalence
The statement must be changed accordingly: not any FH-bisimulation is an equivalence
\end{review}
Done, see maini part of the letter; thanks for pointing this out

\begin{review}
\ldots

P24: "The condition J={j} is a bit restrictive ... " I disagree as you should be careful not to introduce sync of non observable actions
\end{review}
 In fact, here we speak about the direction 2 in the definition. Having J={i,j} would ALLOW taus to occur synchronously but condition 1 still ensures that each process can do a tau separately. So yes we could introduce synchronized taus but as taus do not have to be synchronised it is not a problem.
We added explanations at this point, our text was indeed imprecise. 

\begin{review}

P24: the $\cup$ with the dot inside is a weird symbol for concatenation

\end{review}
We chose a union operator as it is a pointwise concatenation so it also does a union on disjoint indexed sets for example ... but this was probably too intricate here.

We tried \texttt{oplus}, we hope it makes more sense.

\begin{review}
\ldots

P52,Lemma 4: having n=-1 and m=-1 in the base case is a bit strange. Can you rephrase the Lemma to link the base case to n=0,m=0?
\end{review}
We did this, but this required a lot of changes and many $1$ instead of $0$ in the rest of the proofs/theorems, to be consistent.

\end{document}

%%% Local Variables: 
%%% mode: latex
%%% TeX-master: t
%%% End: 
