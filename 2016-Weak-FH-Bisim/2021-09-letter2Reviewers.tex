\documentclass{article}
\usepackage{a4wide}
\title{Reply to reviewers\\
LMCS \#\,6784 Manuscript: \\``Compositional equivalences based on Open pNets'' }

\begin{document}

\maketitle
\noindent
Dear Editor, dear reviewers,

Please find enclosed the revision of our submission to LMCS. We improved our article according to the comments made by the
anonymous reviewers.

We would first like to thank the reviewers for their analysis of our
paper, and the useful advices they gave us.

We detail below the major changes we made in this revision. After
this, we will review in the next pages the comments of the reviewers
and explain how we addressed them.  \emph{Comments that could be
  addressed trivially and exactly as suggested by the reviewers are
  omitted}.  \bigskip

In this new version, you will find the major improvements listed below:
\begin{itemize}
\item 
\end{itemize}

\bigskip
\noindent
Best regards,
\newline
\noindent
Rab\'ea Ameur-Boulifa, Ludovic Henrio, Eric Madelaine.

\newpage
As explained above, we quote below and reply to text requiring changes that we did not 
address in a straightforward manner. Major points of the reviews have not been omitted in 
any case.

\section*{Annex 1: response to first reviewers' comments}
- The definition of pNets is cumbersome. I understand that this work extends prior work where the model has been motivated already, but I have my reservations on the utility of the formalisms.    I found the running example (although useful) to be hard to understand.  It would perhaps be useful to provide anecdotal evidence showing that comparable formalisms cannot handle certain features of the example (which would justify the complexity of the model used).

- I was left unconvinced that using open automata as an intermediary structure to define the bisimualtion relations is the best approach to develop this material.  The authors should give more convincing argument that such definitions could not be adequately defined directly on pNets (an already complex framework).  It left me wondering:
1) How do you validate the translation of pNets to Open Automata?  Shouldn't there be some result that states that the translation is semantic preserving?    
2) Why not work directly in terms of Open Automata (and forget all about pNets)?  A prima facie, they appear to be of a comparable complexity to pNets (if not simpler).  There are few people who would be interested in using a framework that relies on the second rule in Def 9 as a modelling language...  

- The definition of bisimulation is slightly non-standard (eg Def 7).  For instance, why does the transfer property require a "set" of matching transitions?  Normally, the existence of one suffices.  Moreover, given that a set is required, shouldn't there be an additional requirement that the matching set is necessarily non-empty? 

- In general, weak bisimulation is *not* a congruence for all types of interacting contexts.  For instance, in the case of CCS, the classsical counter example is that whereas 
tau.a.0  \\bisim  a.0
we do *not* have
tau.a.0 + b.0 \\bisim  a.0 + b.0 
Given the generality of the pNets framework I found the claim stated in Theorem 7 to be suspicious (and hard to ascertain).  The authors should perhaps provide more intuition why this is the case for their framework.


\section*{Annex 2: response to second reviewers' comments}
DETAILED COMMENTS

P1 Abstract L10 *… that includes parameters, and …* parameters for what? Please elaborate more on this in the introduction.

P2 L24 *… provides a cleaner version…* I am afraid such a sentence does not add any value if there is no explanation discussing it in detail somewhere later in the paper.

P3 L-11 What is an action algebra? I believe it is an undefined term in the paper.

P4 L-6 *We distinguish two kinds….* please rewrite this sentence

P4 L1 We additionally *impose*

P4 L6 *negociation* typo

P4 L20 A minor point: Technically, $a^{i\in 3}$ is not allowed by your syntax since indices need to also appear as a subscript.

P4 L22 I am afraid we do not use abusive vocabulary in mathematics, rather we only abuse notation. Please rewrite.

P4 L26 *\\uplus is the disjoint union of sets* Please rewrite this sentence; it is considered to be a bad practice to start the sentence with a symbol (cf. Donald Knuth on mathematical writing available from $https://jmlr.csail.mit.edu/reviewing-papers/knuth_mathematical_writing.pdf$).

P4 L-9 *We denote y <- e a substitution* What do you mean by this? Do you mean that y<-e is a partial function from the set of variables to terms that is only defined for the variable y as the term e?

P4 L-9 *The application of a substitution…* Its strange to denote an application of a substitution without any denotation for its argument.


P4 L-7 The notion of substitutions on indexed sets is more subtle; they may not be a function unlike a substitution in the traditional setting. For e.g., $x_1 <-e, x_2 <-e’ with x_1 = x_2$.

P4 L-6 *\\otimes is the composition operator…* Is this operation totally defined? This even makes it necessary that substitutions are defined properly.

P4 L-2 Missing comma

P5 Def 1 Is vars(s) defined earlier?

P5 Def 2 How is the symbol $l$ quantified in the clause pertaining to synchronisation vectors.

P6 Def 3 In the first clause defining Sort, the notation *?x <- x* is not defined since, technically, a substitution substitutes a variable not a term like ?x.

P6 Def 3 In the third item, *indexes* should be *indices*

P6 L-15 It would be nice to have a sentence motivating synchronisation vectors that they are essentially transitions; here the example from 2.3 was helpful.

P6 Def 4 When is a pNet of the right sort?

P9 Def 5 Here, is $Sort_j$ again a set of parameterised actions?

P9 Def 5 L6 *all variables in the different terms $\beta_j$ and $\alpha$* What do you mean by this phrase?

P9 Def 5 L7 By *assignments* do you mean indexed substitutions?

P9 Def 6 L-4 Please explain *simple logic* and *paper rules* in more detail?

P9 Def 6 L-2 open with capital O

P9 Def 6 L8 *We take in this article …* Doesn’t add anything and it isn’t clear what the writer is trying to convey here? I guess that the intent is to stay that the set of transition is closed under the implication given below in Page 9.

Also, it would be good to add some accompanying texts that explain this implication. My understanding is that this implication says that the set of open transitions is closed under the
refinement of predicates, which is a strong assumption from the modelling point of view. This is because modeller is forced to add new behaviour which (s)he is not interested to capture.
Lastly, I was anticipating a more formal way to derive transitions of an open automaton just like how transitions are derived by a witnessing proof in the context of transition system specifications. For latter, a reference is as follows:
J.F. Groote. Transition system specifications with negative premises. TCS 118, 1993 $https://www.sciencedirect.com/science/article/pii/0304397593901116$
Also, I believe that the induced transitions in T are all meaningful (in the sense of Glabbeek, see below) because there are no negative transitions in the premise of an open transition.
R.J. van Glabbeek. The meaning of negative premises in transition system specifications II. JLAP, 2004 $https://www.sciencedirect.com/science/article/pii/S1567832604000281$
Such a discussion with a possible related work is missing and should be provided at least in a separate remark.

P10 Def 7 L-6 Typo *!J* should be just *J*

P10 Def 7 It was frustrating to not find out what is the type of an FH-bisimulation R, which I think is necessary if you want to compute such an R. Firstly, it should be mentioned that $Pred_{s,t}$ is some predicate over $V_1 \cup V_2$. Second, please give the formal type of conditional relation R before defining the transfer property of an FH-bisimulation. Here, by a conditional relation R on the sets X and Y I mean a function of type $X \times Y ---> L$, where L is some lattice modelling the values that the relation can take. For instance, traditional relation on X and Y can be seen as a function $X \times Y ---> 2$, where 2={0,1} is the obvious Boolean algebra of two point set; so $x R y \iff R(x,y)=1$. Similarly, in your case, the lattice L should be the Boolean algebra P(Pred), where P(X) is the powerset of X, for any set X.
This raises the issue whether FH-bisimulation is an instance of conditional bisimulation (when the lattice L is P(Pred) ) as defined in the following paper:
H.Beohar, B. König, S. Küpper,  A. Silva. Conditional transition systems with upgrades. $https://www.sciencedirect.com/science/article/abs/pii/S0167642319301169$

P11 L1 What is X in the open transition $OT_x^{x\in X}$?

P11 L2 Typo: jx as the subscript of \\beta. You could use $\beta_{j_x}$?

P11 L-8 What do you mean by a finite predicate?

P12 L6 Could you expand on what is bisimulation theory for open pNets?

P13 Def 9 Please exemplify *I\_1 for the others*?

P13 Def 9 In the first line of the premise of Tr2, shouldn’t the occurrences of $\alpha’ in SV_k just be \alpha$?

P13 Def 9 Tr2 by *$fresh(\alpha’_m,\alpha’,\beta_j,\alpha)$* should be understand that the variables occurring in $\alpha’_m,\alpha’, \beta_j$ (for all j), and $\alpha$ are fresh?

P16 L-4 You claim Theorem 5 is quite useful in practice; however, I wonder here about FH-simulation. First, whether a similar result holds for simulation? Second, FH-simulaton would be more relevant if we 
disallow that the set of transitions in an open automaton are closed under the refinement of predicates (Def. 6).

P17 Def 11 Isn’t (1) a special case of (2)?

P18 Def 12 Other than the symbol used for weak transition, is there any semantic difference in Def 12 between a weak open transition and an open transition defined earlier?

P22 L-1 Shouldn’t Pred be $Pred_{s,t}$? Again, please give the type of this conditional relation

P28 L2 *principle* --> *relations*

\end{document}


%%% Local Variables:
%%% mode: latex
%%% TeX-master: t
%%% End:
